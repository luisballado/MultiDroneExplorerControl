\chapter{Introducción}

Los robots de servicio son máquinas autónomas diseñadas con el objetivo de prestar servicio a los humanos fuera del ambiente industrial, convirtiéndose poco a poco en una parte esencial en nuestras vidas. Los podemos encontrar en diversos ámbitos, como en el entretenimiento, limpieza, logística, entre otras soluciones inovadoras.

Los vehículos aéreos no tripulados (VANTS) han evolucionado rápidamente y se han convertido en sistemas versátiles capaces de una amplia gama de aplicaciones, desde vigilancia hasta misiones de búsqueda y rescate. Entre ellas, las tareas de exploración en entornos complejos y dinámicos representan un área interesante y desafiante, donde la coordinación de múltiples vehículos aéreos no tripulados se vuelve primordial. Este tema es de creciente importancia a medida que los vehículos aéreos no tripulados (VANTS) continúan transformando distintas áreas, incluida la agricultura, el monitoreo ambiental, mantenimiento de infraestructuras (puentes, edificios, líneas eléctricas) y la respuesta a desastres naturales, reduciendo los riesgos y costos asociados con las inspecciones manuales.

Los vehículos aéreos no tripulados (VANTS) capaces de realizar tareas con autonomía generalmente cuentan con mayores capacidades de carga y procesamiento computacional, así como sensores capaces de percibir grandes volúmenes de datos en un tiempo reducido. Estos vehículos aéreos no tripulados (VANTS) se centran en realizar tareas sencillas y estáticas en áreas abiertas con rutas predeterminadas, o bien, en contextos de operación por control remoto por un usuario. Sin embargo en donde los espacios son estrechos, se optan por el uso de VANTS reducidos, comúnmente llamados Micro-vehículos aéreos no tripulados (MAVs).

La exploración de áreas desconocidas, a través de la sinergia de sistemas con múltiples vehículos aéreos no tripulados (Multi-VANTS) promete ser una solución innovadora. Al comprender y perfeccionar una estrategia para la coordinación de múltiples vehículos aéreos no tripulados (Multi-VANTS) en tareas de exploración, esperamos descubrir nuevas posibilidades, replicar o romper los límites existentes, y en última instancia, avanzar en los campos de la robótica y la exploración.

\section{Antecedentes y motivación} 

\subsection*{Antecedentes}

La robótica móvil es una rama de la robótica que se enfoca en el diseño, construcción, programación y operación de robots capaces de moverse de manera autónoma o semi-autónoma en entornos diversos.

Uno de los primeros hitos importantes en la robótica móvil fué el desarrollo del robot shakey en la década de 1970, que se considera el primer robot móvil capaz de desplazarse evitando colisiones a su paso, sentando las bases en algoritmos de inteligencia artificial para búsquedas informadas surgiendo el algoritmo A*. 

Durante las décadas siguientes, se produjeron avances significativos en la miniaturización de componentes electrónicos, sensores y actuadores, lo que permitió la creación de robots móviles más pequeños y versátiles. A finales del siglo XX, los robots móviles comenzaron a ser utilizados en una variedad de aplicaciones, como la exploración espacial, la agricultura, la vigilancia y la logística.

En paralelo, los avances en inteligencia artificial, visión por computadora, planificación de trayectorias y sistemas de control contribuyeron al desarrollo de robots móviles más autónomos y adaptables. El surgimiento de algoritmos de aprendizaje automático y técnicas de percepción avanzada ha permitido a los robots móviles interactuar de manera más efectiva con su entorno y tomar decisiones en tiempo real.

En la actualidad, la robótica móvil está experimentando un rápido crecimiento gracias a la convergencia de diversas tecnologías, como la computación en la nube y la robótica colaborativa. Se están desarrollando robots móviles cada vez más sofisticados y capaces de realizar una amplia gama de tareas en entornos dinámicos y no estructurados. Además, se espera que la robótica móvil desempeñe un papel crucial en aplicaciones futuras, como la asistencia en el hogar, la atención médica, la exploración submarina y la entrega de paquetes.

Un sistema autónomo de un vehículo aéreo no tripulado, consta de cuatro algoritmos:
\begin{itemize}\setlength{\itemsep}{-1mm}
\item Generación de una representación del medio ambiente
\item Planificación de trayectorias
\item Evasión de obstáculos
\item Comunicación
\end{itemize}

La computadora embebida para un sistema de navegación usado en vehículos aéreos no tripulados de menor tamaño, son de bajo rendimiento. Pero, su necesidad de autonomía sigue siendo la misma que un VANT de mayor tamaño. Es por ello que es necesario equiparlos con algoritmos de baja complejidad computacional.

La necesidad de coordinación entre múltiples VANTS en tareas de exploración, surge debido a las limitaciones individuales en cuanto a la extensión de terreno que pueden cubrir y su desempeño. La exploración de áreas extensas o peligrosas a menudo exige un enfoque colaborativo, donde los VANTS trabajen juntos para optimizar la asignación de recursos y mejorar la recopilación y el análisis de datos.

Sin embargo, el camino hacia una coordinación entre múltiples vehículos aéreos no tripulados, presenta diversos desafíos. Las complejidades de gestionar un grupo de vehículos aéreos no tripulados, navegar en entornos dinámicos y distribuir tareas de forma inteligente son sólo algunas de las cuestiones que exigen nuestra atención. %El objetivo del documento es explorar y proponer una estrategia para mejorar la coordinación de múltiples VANTS en el contexto de las tareas de exploración.
% usar --> para resaltar cosas \textbf{\emph{}}

\subsection*{Motivación}

El potencial del uso de los vehículos aéreos no tripulados en tareas de búsqueda y rescate, inspección, mapeo, vigilancia, entre otras, es de gran interés a explorar, debido a las habilidades de vuelo que presentan en favor de la realización de estas tareas, y en especial situaciones que podrían poner en riesgo a personas.

Enviar personal de rescate dentro de un edificio parcialmente colapsado en busca de sobrevivientes, es poner a más personas en un gran riesgo, pues no se sabe qué es lo que les espera en el interior del edificio; esto limita la capacidad de tomar buenas decisiones acerca de si es seguro seguir cierto camino.

Operar en ambientes como éste u otros similares requieren de robots con capacidades de navegar sobre terrenos difíciles y evadir obstáculos de forma segura para obtener información del entorno que pueda ser útil al personal de rescate.

Una posible solución consiste en un robot móvil aéreo (VANT) capaz de desplazarse sobre terrenos difíciles y navegar en espacios cerrados de manera segura, que además recabe información del entorno. Esto implica realizar tareas de reconocimiento del ambiente, evasión de obstáculos y seguimiento de trayectorias.

\section{Planteamiento del problema} 

Dado un volumen de interés desconocido en un espacio cerrado que se desea explorar denotado como $\mathcal{W}$, tal que $\mathcal{W} \subset \mathbb{R}^{3}$.
%el volumen se discretiza en un mapa de ocupación M consistente en voxeles cúbicos $m \exists M$ con una longitud de arista r.
\begin{itemize}\setlength{\itemsep}{-1mm}
  \item El volumen se discretiza usando unidades cúbicas tridimencionales (voxel) tomando valores $v_{libre}$, $v_{ocup}$, $v_{desc}$.
    %La representación del volumen a explorar se obtiene dividiendo el volumen de interés en unidades cúbicas tridimencionales (voxel) que puede tomar los valores de libre $v_{libre}$, ocupado $v_{ocup}$ y desconocido $v_{desc}$ con lecturas a partir de los valores de una cámara RGB-D basada en un modelo de ocupación probabilístico.\\
  \item Un conjunto de VANTS con una cámara RGB-D embarcadas denotado como $\mathcal{V} = \{\mathcal{V}_{1},\mathcal{V}_{2},\mathcal{V}_{3},...,\mathcal{V}_{n}\}$, comenzando cada uno en un estado inicial conocido $q = \{q_{1},q_{2},q_{3},...,q_{n}\}$, y terminando en una configuración que maximice la construcción de un mapa.
\end{itemize}

Coordinar el conjunto de VANTs para reducir el tiempo total de exploración.
  
  
\section{Hipótesis y preguntas de investigación}

\subsection*{Hipótesis}
\emph{``Una estrategia que coordine y asigne tareas de exploración para múltiples VANTS de manera descentralizada, en combinación con una arquitectura de software diseñada para resolver problemas de localización, gestión de mapas y planificación de rutas, mejorará la eficiencia y cobertura de la exploración en interiores de un entorno desconocido''}.


\subsection*{Preguntas de investigación}

\begin{enumerate}\setlength{\itemsep}{-1mm}
\item ¿Qué características de la dinámica del VANT son cruciales para lograr trayectorias suaves y continuas?
\item ¿Podría un planificador de trayectorias que aproveche las regiones libres de obstáculos acelerar los desplazamientos de los VANTs y, consecuentemente, reducir los tiempos de exploración?
\item ¿Qué mecanismos de coordinación existen dentro de la literatura que podrían ayudar en resolver el problema de exploración multi-VANT?
\end{enumerate}

\section{Objetivos}

\subsection*{Objetivo general}

%El principal objetivo del trabajo de tesis se define a continuación:
\emph{``Desarrollar una estrategia de exploración descentralizada que permita resolver los problemas de coordinación para múltiples VANTS en ambientes desconocidos.''}.

\subsection*{Objetivos específicos}

Para lograr nuestro objetivo principal, se consideran los siguientes objetivos especificos a cubrir:

\begin{itemize}\setlength{\itemsep}{-1mm}
\item Desarrollar una arquitectura de software que resuelva los problemas de autonomía para un VANT (localización, manejo de mapas y planificación de trayectorias).
\item Implementar un mecanismo de coordinación descentralizado que asigne tareas de exploración.
\item Realizar pruebas y simulaciones de la solución propuesta en diversos entornos, analizando la relación tiempo de exploración y cobertura del área de interés.
\end{itemize}

\newpage

\section{Solución propuesta}

Para resolver el problema de exploración multi-VANT con un enfoque descentralizado considerando restricciones en el rango de la comunicación. Se cuenta como antecedente el trabajo doctoral de \citeauthor{CINVESTAM2013} que propone un algoritmo basado en un proceso de ofertas de mercado, en el cual cada robot calcula las ofertas de manera independiente, buscando alcanzar el mayor aporte posible al equipo en su conjunto. Cuando un robot alcanza su objetivo, el robot toma una decisión para sí mismo, involucrando a cada uno de los miembros del equipo así como el rango de comunicación, bajo un esquema descentralizado y sin la necesidad de un módulo central.

Para validar la propuesta de exploración coordinada se necesita primero resolver la autonomía de un vehículo aéreo no tripulado diseñando una arquitectura que incluya la coordinación multi-robot propuesta en \cite{CINVESTAM2013}.

\begin{enumerate}\setlength{\itemsep}{-1mm}
\item Conocer los fundamentos que nos aproximen a realizar la tarea de exploración autónoma con múltiples VANTS.
\item Profundizar en la comprensión de los comandos de control y la generación de odometría para un VANT tipo cuadricóptero.
\item Obtener y procesar la información proveniente de un sensor de tipo RGB-D dentro del sistema operativo ROS.
\item Integrar un planificador de trayectoria reactivo que, combinado con la percepción recibida por la cámara RGB-D, nos permita evadir obstáculos en su paso y construir una representación tridimensional del medio ambiente.
\item Elaborar la exploración con un VANT de tipo cuadrotor.
\item Implementar la estrategia de exploración bajo los conceptos de cohesión, exploración y explotación.
\end{enumerate}



%\section{Resumen}
%\lipsum[2-4]

%The rest of this document is organized as follows: Chapter~\ref{chapter2} presents basic concepts and background in the field of optimization. Then, Chapter~\ref{ch:PSODE} introduces particle swarm optimization and differential evolution which are the two metaheuristics on which this thesis work focusses. In order to introduce these two metaheuristics, EAs is general are also described in this chapter. Afterwards, Chapter~\ref{experiments} presents a series of experiments that were developed and that allowed to obtain further information about the search performed by PSO and DE in multi-objective optimization. This knowledge was used to develop two new MOEAs which are presented and evaluted in Chapter~\ref{proposals}. Finally, Chapter~\ref{conclusion} concludes this thesis work.
