\chapter{Introducción}

\lipsum[2-4]

\section{Antecedentes y motivación} 

% usar --> para resaltar cosas \textbf{\emph{}}

\lipsum[2-4]

\section{Planteamiento del problema} 

\lipsum[2-4]

\section{Hipótesis y preguntas de investigación}

\lipsum[1]

\section{Objetivos}

The main objective of this thesis work can be defined as follows: \emph{``To contribute to the state of the art by identifying and understanding the mechanisms that promote good performance on multi-objective particle swarm optimization and multi-objective differential evolution and that allow a multi-objective evolutionary algorithm to reduce the number of objective function evaluations needed to solve an arbitrary MOP''}. 

This main goal has been divided in the following specific objectives: 
\begin{itemize}\setlength{\itemsep}{-1mm}
	\item To evaluate different strategies to generate solutions in DE and to move particles in PSO on multi-objective problems.
	\item To perform a series of experiments that allow to better understand the behavior of PSO and DE on multi-objective problems.
	\item To identify the mechanisms that impact (either enhance it or deter it) in the search of the metaheuristics particle swarm optimization and differential evolution when attacking multi-objective optimization problems. 
	\item To contribute to the state of the art with at least one multi-objective evolutionary algorithm that utilizes knowledge derived from this work to enhance the search process on diversity and convergence. 
	\item To validate the performed experiments using performance measures and test problems taken from the specialized literature.
\end{itemize}

\section{Solución propuesta}

\lipsum[2-4]

\section{Resumen}

\lipsum[2-4]

%The rest of this document is organized as follows: Chapter~\ref{chapter2} presents basic concepts and background in the field of optimization. Then, Chapter~\ref{ch:PSODE} introduces particle swarm optimization and differential evolution which are the two metaheuristics on which this thesis work focusses. In order to introduce these two metaheuristics, EAs is general are also described in this chapter. Afterwards, Chapter~\ref{experiments} presents a series of experiments that were developed and that allowed to obtain further information about the search performed by PSO and DE in multi-objective optimization. This knowledge was used to develop two new MOEAs which are presented and evaluted in Chapter~\ref{proposals}. Finally, Chapter~\ref{conclusion} concludes this thesis work. 
