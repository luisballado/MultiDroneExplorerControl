\chapter{Marco Teórico}

La exploración coordinada con múltiples vehículos aéreos no tripulados es un área de investigación actual en la robótica móvil, con aplicaciones en campos como la vigilancia, la exploración de áreas de difícil acceso y la respuesta a emergencias. Para comprender plenamente este tema, es crucial tener una sólida comprensión de los fundamentos teóricos que sustentan tanto la representación del movimiento de los vehículos aéreos no tripulados como los algoritmos para planificación de trayectorias y coordinación que facilitan su trabajo conjunto para alcanzar objetivos comunes mientras evitan colisiones y construyen una representación del medio ambiente.

%El marco teórico de esta tesis abarca desde los conceptos básicos de transformaciones homogéneas, que son fundamentales para describir la cinemática y la posición de los VANT en un espacio tridimensional, hasta los algoritmos de coordinación, que permiten a múltiples VANT colaborar de manera efectiva para alcanzar objetivos comunes mientras evitan colisiones y optimizan la eficiencia.

%Comenzaremos explorando las transformaciones homogéneas, que son herramientas matemáticas poderosas para representar la posición y orientación de los VANT en un sistema de coordenadas global. Estas transformaciones son esenciales para la planificación de trayectorias, la navegación y la coordinación de múltiples VANT en entornos complejos y dinámicos.

%A continuación, examinaremos los algoritmos de coordinación, que abarcan desde enfoques simples basados en reglas hasta métodos más avanzados basados en técnicas de optimización y aprendizaje automático. Estos algoritmos permiten a los VANT colaborar de manera inteligente, compartiendo información sobre el entorno, planificando rutas eficientes y ajustando su comportamiento en tiempo real para adaptarse a cambios en las condiciones del entorno.

Al comprender estos aspectos teóricos fundamentales, estaremos preparados para abordar los desafíos de la exploración coordinada con múltiples vehículos aéreos no tripulados.

%%%%%%%%%%%%%%%%%%%%%%%
%%%%%%%%%%%%%%%%%%%%%%%
%%%%%%%%%%%%%%%%%%%%%%%
%%%%%%%%%%%%%%%%%%%%%%%
%%%%%%%%%%%%%%%%%%%%%%%

\section{Conceptos fundamentales} \label{}

\subsection*{Exploración o localización y mapeo simultáneo}

Una diferencia clave entre el problema de localización y mapeo simultáneo (SLAM) y la exploración radica en sus objetivos y enfoques:

El \textbf{objetivo principal del SLAM} es estimar simultáneamente la trayectoria del robot (o vehículo) y construir un mapa del entorno desconocido en el que se encuentra.

El \textbf{objetivo principal de la exploración} es recorrer y mapear un área desconocida de manera eficiente y completa, sin necesariamente enfocarse en la localización precisa del robot en tiempo real.

Mientras el \textbf{enfoque del SLAM} se centra en la fusión de datos de sensores (como cámaras, LIDAR, etc.) con algoritmos de estimación de estado (como filtros de Kalman extendidos o métodos basados en grafos) para actualizar la estimación de la posición del robot y construir un mapa del entorno.

Por otra parte, el \textbf{enfoque de la exploración} se centra en la planificación de trayectorias y la toma de decisiones para guiar al robot a través del entorno de manera que se maximice la cobertura del área a explorar y se minimice el tiempo y los recursos necesarios.

%En resumen, mientras que el SLAM se enfoca en la estimación simultánea de la posición del robot y la construcción del mapa del entorno, la exploración se centra en recorrer y mapear un área desconocida de manera eficiente, sin necesidad de una localización precisa en tiempo real.

\subsection*{Sistema de ejes coordenados}

El sistema de ejes coordenados en tres dimensiones (3D) es una herramienta fundamental en la representación y comprensión del espacio tridimensional. Consiste en tres ejes: x, y, z, que se intersectan en un punto de origen común. Estos ejes proporcionan un marco de referencia para describir la posición y orientación de objetos en el espacio 3D, facilitando el análisis de información espacial. Un entendimiento de este sistema es importante para la planificación de trayectorias, la navegación, la representación del medio ambiente para el diseño de sistemas autónomos, incluyendo los vehículos aéreos no tripulados (VANTS).

\textit{\textbf{Desarrollar..}}

\subsection*{Coordenadas homogéneas}

Las coordenadas homogéneas son una herramienta matemática esencial en la representación y manipulación de la posición y orientación de vehículos aéreos no tripulados (VANTS) en un entorno tridimensional. Este enfoque permite expresar las transformaciones geométricas de manera compacta. Al utilizar coordenadas homogéneas, se pueden representar tanto las traslaciones como las rotaciones en un único marco de referencia, lo que simplifica los cálculos. Este enfoque es fundamental para el desarrollo de sistemas autónomos con habilidades de exploración coordinada, ya que proporciona una base matemática sólida para la comprensión y manipulación del movimiento de los vehículos aéreos no tripulados (VANTS).

\textit{\textbf{Desarrollar..}}

\subsection*{Transformaciones geométricas}

En el contexto de los vehículos aéreos no tripulados (VANTS), las transformaciones geométricas son herramientas fundamentales para comprender y manipular la posición y orientación de los vehículos en un espacio tridimensional. Estas transformaciones se utilizan para describir cambios en la posición, la rotación y el tamaño de los objetos en un sistema de coordenadas. Comprenden operaciones como la traslación y la rotación, que permiten modelar y simular el movimiento de los VANTS. 

\textit{\textbf{Desarrollar..}}

\section{Funcionamiento de un VANT}

Un vehículo aéreo no tripulado (VANT) es un robot móvil aéreo que gracias a su privilegiada perspectiva aérea han captado la atención a más de una persona.

Su funcionamiento se basa en una serie de componentes clave, que incluyen una plataforma de vuelo equipada con sistemas de propulsión, sensores, actuadores y una unidad de procesamiento central. Estos dispositivos pueden variar en tamaño, forma y capacidad.

%El VANT utiliza sus sensores para recopilar información del entorno, como datos de posición, imágenes, video y mediciones de distancia. Esta información se procesa a bordo o en tierra mediante algoritmos de percepción y planificación de rutas, permitiendo al VANT tomar decisiones autónomas sobre su navegación, evasión de obstáculos y cumplimiento de objetivos de misión.

El vuelo de un VANT se controla mediante comandos enviados desde una estación terrestre o mediante algoritmos de control autónomo a bordo. Estos comandos ajustan la velocidad, dirección y altitud del VANT, permitiéndole moverse de manera precisa y eficiente en el espacio aéreo.

El funcionamiento de un VANT se ve influenciado por una serie de factores, incluyendo las condiciones meteorológicas, las regulaciones gubernamentales, la autonomía de la batería y la calidad de los datos de los sensores. La seguridad y la fiabilidad son consideraciones críticas en el diseño y operación de estos sistemas, lo que requiere un enfoque cuidadoso en la planificación, desarrollo y prueba de cada componente.

\textit{\textbf{Desarrollar..}}

%En resumen, el funcionamiento de un VANT implica una integración compleja de hardware, software y sistemas de control para permitir operaciones autónomas en entornos aéreos variados. El entendimiento detallado de estos principios es fundamental para el diseño efectivo, la operación segura y el aprovechamiento máximo del potencial de los VANT en una amplia gama de aplicaciones.

\section{Control de un VANT}

El control es un aspecto crucial en la investigación y desarrollo de sistemas autónomos aéreos. Este proceso implica el diseño y la implementación de algoritmos y estrategias que permiten a los VANTs realizar tareas específicas de manera eficiente y segura. Se abordan diversos aspectos del control, como la estabilización, la navegación, la planificación de trayectorias y la interacción con el entorno. La investigación en este campo se centra en mejorar la precisión, y la capacidad de adaptación de los algoritmos de control, así como en la integración de tecnologías de la inteligencia artificial clásica. El control efectivo de los VANTs es fundamental para una amplia gama de aplicaciones.

\textit{\textbf{Desarrollar..}}

\section{Estimación de posición}

El proceso de estimación de posición en vehículos aéreos no tripulados (VANTs) es fundamental para su operación autónoma. Este proceso implica el uso de sensores y algoritmos para determinar la posición y orientación del VANT en relación con un sistema de coordenadas global. Los VANTs suelen utilizar una combinación de sensores, como IMU, cámaras y LIDAR, para recopilar información sobre su entorno y su movimiento. A partir de estos datos, se aplican algoritmos de estimación, como el filtro de Kalman extendido (EKF) o el filtro de partículas, para integrar la información de los sensores y calcular la posición estimada del VANT con respecto al tiempo.

\textit{\textbf{Desarrollar..}}

\subsection{Odometría visual inercial (Visual Inertial Odometry - VIO)}

Es una tecnología avanzada utilizada en vehículos aéreos no tripulados (VANTs) que combina datos visuales y de sensores inerciales para estimar la posición y orientación del vehículo en tiempo real. Este enfoque aprovecha la información visual de cámaras a bordo y la información de acelerómetros y giroscopios para calcular con precisión la trayectoria del VANT en entornos dinámicos. Al fusionar la información de múltiples fuentes sensoriales, la odometría visual inercial (VIO por sus siglas en inglés) supera las limitaciones individuales de cada sensor y proporciona una estimación confiable del movimiento del VANT.

\textit{\textbf{Desarrollar..}}

\section{Planificación de trayectoria}

El tema de la planificación de trayectorias es de vital importancia, abarcando un conjunto diverso de técnicas y enfoques para garantizar la navegación segura y eficiente de estos dispositivos en entornos complejos y dinámicos. Este proceso implica la generación de rutas que permitan a los VANTs alcanzar sus destinos mientras evitan obstáculos y cumplen con restricciones operativas, como minimizar el consumo de energía o maximizar la cobertura de área explorada. Desde métodos clásicos basados en grafos, hasta enfoques más avanzados que integran técnicas de aprendizaje automático y optimización.

\textit{\textbf{Desarrollar..}}

\subsection{Evasión de obstáculos}

El desarrollo de estrategias efectivas para la evasión de obstáculos es un componente crítico en el diseño de sistemas autónomos, especialmente en el contexto de los vehículos aéreos no tripulados (VANTs).

\textit{\textbf{Desarrollar..}}
%Además, se investigan estrategias de planificación de trayectorias y control de vuelo que permiten a los VANTs calcular rutas alternativas y tomar decisiones evasivas en función de la información del entorno.

\section{Representación medio ambiente}

%El mapeo del entorno es una tarea crítica en la operación de Vehículos Aéreos No Tripulados (VANTs), ya que proporciona una representación precisa y detallada del medio ambiente. Estos mapas son esenciales para la planificación de rutas, la navegación autónoma y la toma de decisiones inteligentes durante las misiones de exploración.

Estas representaciones espaciales permiten a los robots construir y mantener mapas detallados y precisos del entorno que los rodea, lo que les permite tomar decisiones inteligentes y seguras en tiempo real. Los Octomaps, que se basan en estructuras de datos de tipo octree, permiten una representación eficiente de la información de obstáculos y espacio libre en 3D, lo que facilita la planificación de trayectorias y la evasión de obstáculos. Por otro lado, los HGrids son una alternativa que utiliza una cuadrícula híbrida para representar el entorno, combinando una cuadrícula regular con una estructura de árbol para adaptarse a la complejidad del terreno. Ambos enfoques son ampliamente utilizados en robótica móvil y sistemas autónomos para mejorar la percepción del entorno y la navegación en entornos 3D, lo que resulta crucial en aplicaciones como la exploración de terrenos desconocidos.

%En la tesis se examinan diferentes enfoques y técnicas para la generación y actualización de mapas, que van desde métodos basados en sensores como cámaras y LIDAR hasta técnicas de fusión de datos y aprendizaje automático. Se analizan tanto las ventajas como las limitaciones de cada método en términos de precisión, eficiencia computacional y robustez en diferentes entornos y condiciones. Además, se exploran estrategias para la representación eficiente y la gestión de la incertidumbre en los mapas, con el objetivo de mejorar la fiabilidad y la utilidad de los datos para la planificación de misiones y la toma de decisiones. El estudio contribuye al avance en el campo de la robótica móvil y la exploración autónoma, ofreciendo nuevas perspectivas y soluciones innovadoras para el mapeo del entorno con VANTs.

\subsection{Ray casting}

El raycasting es una técnica fundamental en la percepción que permite detectar obstáculos y mapear el entorno tridimensional de manera eficiente. Consiste en emitir rayos desde el robot móvil en direcciones específicas y detectar las intersecciones con objetos o superficies en el entorno. Estas intersecciones proporcionan información crucial sobre la geometría del entorno, que se utiliza para generar mapas tridimensionales y planificar trayectorias seguras y eficientes para la navegación autónoma.

\textit{\textbf{Desarrollar..}}

\subsection{Point clouds}

El uso de vehículos aéreos no tripulados (VANTs) para la captura y procesamiento de nubes de puntos (point clouds) ha emergido como una herramienta poderosa en una amplia gama de aplicaciones, desde la cartografía y la topografía hasta la inspección de infraestructuras y la monitorización medioambiental. Las nubes de puntos, que consisten en conjuntos masivos de puntos tridimensionales que representan la superficie de objetos y entornos, proporcionan información detallada y precisa sobre la estructura y la geometría de los mismos.

\textit{\textbf{Desarrollar..}}

\subsection{Octomapas vs. HGrid}

\textit{\textbf{Desarrollar..}}

%En esta tesis, exploramos los avances en la adquisición y procesamiento de nubes de puntos utilizando VANTs, centrándonos en el desarrollo de algoritmos y técnicas para la planificación de trayectorias, la navegación y la captura eficiente de datos. Además, investigamos estrategias para la fusión y la interpretación de múltiples nubes de puntos obtenidas de diferentes perspectivas y momentos temporales, con el objetivo de generar modelos tridimensionales precisos y completos del entorno. Nuestro trabajo contribuye al avance en la capacidad de los VANTs para la recopilación de datos geoespaciales y la generación de información útil para una variedad de aplicaciones, impulsando así el potencial de esta tecnología en el ámbito de la percepción y la inteligencia espacial.

%\section{Exploración}

%El uso de Vehículos Aéreos No Tripulados (VANTs) para la exploración ha ganado prominencia en una variedad de campos, desde la vigilancia ambiental hasta la respuesta a desastres. Esta tesis se centra en el desarrollo y la implementación de técnicas avanzadas para la exploración coordinada con múltiples VANTs. Se investigan y se proponen algoritmos de coordinación eficientes que permiten a los VANTs colaborar de manera inteligente, compartiendo información del entorno y optimizando rutas para una exploración efectiva. Además, se aborda la planificación de trayectorias y la navegación autónoma para garantizar la eficacia y seguridad de las misiones de exploración. A través de experimentos simulados y pruebas de campo, se evalúan y validan las soluciones propuestas, con el objetivo de avanzar en el estado del arte en la exploración con VANTs y contribuir al desarrollo de sistemas autónomos más robustos y eficientes.

\section{Exploración multi-robot}

El concepto de exploración multi-robot con vehículos aéreos no tripulados (VANTs) es una área de investigación que aprovecha las capacidades colectivas de múltiples robots para explorar y mapear entornos desconocidos de manera eficiente y precisa. Esta estrategia ofrece ventajas significativas sobre la exploración con un solo robot, incluyendo una mayor cobertura del área, tiempos de exploración reducidos y una mayor robustez ante fallos individuales.

%En esta tesis, se investiga y se propone un conjunto de algoritmos y técnicas para la coordinación efectiva de múltiples VANTs en entornos complejos y dinámicos. Se explora el diseño de sistemas de planificación de trayectorias, la asignación de tareas, la comunicación entre robots y la toma de decisiones distribuida para lograr una exploración colaborativa y eficiente. Además, se llevan a cabo experimentos simulados y en entornos reales para validar y evaluar el desempeño de los algoritmos propuestos. Los resultados obtenidos demuestran el potencial de la exploración multirobot con VANTs para aplicaciones prácticas en campos como la vigilancia, la cartografía de áreas de difícil acceso y la respuesta a emergencias. Este trabajo contribuye al avance del conocimiento en el campo de la robótica móvil y sienta las bases para futuras investigaciones en el desarrollo de sistemas autónomos colaborativos para la exploración de entornos desconocidos.

\subsection{Asignación de tareas}

El desarrollo de sistemas multi-robot ha generado un interés creciente en la coordinación y asignación de tareas, especialmente en aplicaciones que requieren la colaboración de múltiples VANT para lograr objetivos comunes de la mejor manera posible.

%En esta tesis, se examina en detalle el diseño y la implementación de algoritmos de coordinación y asignación de tareas para equipos de VANT. Se abordan aspectos teóricos y prácticos relacionados con la planificación de rutas, la optimización de recursos, la comunicación entre VANT y la adaptación a cambios en el entorno. Se exploran diferentes enfoques, desde métodos basados en reglas simples hasta algoritmos más sofisticados que incorporan técnicas de aprendizaje automático y optimización. Además, se presentan estudios de caso y experimentos para validar la efectividad y la escalabilidad de los enfoques propuestos en una variedad de escenarios y aplicaciones prácticas. Esta investigación contribuye al avance del campo de la coordinación y asignación de tareas en sistemas de VANT, proporcionando herramientas y metodologías para mejorar la eficiencia y el rendimiento de los equipos de VANT en una amplia gama de aplicaciones, desde la vigilancia y la exploración hasta la logística y el transporte.

\section{Arquitectura de un robot}

La arquitectura de un robot integra diversos componentes y subsistemas que trabajan en conjunto para permitir al robot percibir su entorno, planificar y ejecutar acciones, y comunicarse con su entorno y otros sistemas, lo que le permite realizar tareas específicas de manera autónoma.

\begin{itemize}\setlength{\itemsep}{-1mm}
\item \textbf{Arquitectura de control:} La arquitectura de control consiste en la implementación de tres capas que permiten organizar los algoritmos en grupos específicos de acuerdo al tipo de acceso requerido. Las capas se listan a continuación.
  \begin{itemize}\setlength{\itemsep}{-1mm}
  \item Capa de planificación: Es la capa de más alto nivel, contiene algoritmos que no son de tiempo real y que pueden demandar altos recursos computacionales. Módulos de planificación de trayectoria, representación del medio ambiente son los incorporados en esta capa. Su función es evaluar los planes para alcanzar objetivos establecidos por la exploración.
  \item Capa de secuenciador: Es la capa mediadora entre la capa de planificación y la capa de habilidades. Activa los módulos solicitados por el planificador proporcionando los argumentos necesarios para cada habilidad y reporta a la capa superior eventos que pueden generar cambios en los planes establecidos.
  \item Capa de habilidades: Es la capa que ejecutará las instrucciones que modifican el estado del vehículo accediendo a su hardware, se conoce como capa de ejecución reactiva y se encarga de las respuestas en tiempo real de los eventos en el medio ambiente. En esta capa se localizan las habilidades reactivas como los módulos de evasión de obstáculos, seguir contrornos entre otros. 
  \end{itemize}

%\item \textbf{Comportamientos VANT:} La arquitectura integra diversos comportamientos que incrementan la funcionalidad del VANT en tareas de exploración.
%  \begin{itemize}\setlength{\itemsep}{-1mm}
%  \item Navegación hacia ciertos puntos.
%  \item Evitar colisones.
%  \item Construcción y mantenimiento de un mapa de ocupación 3D.
%  \end{itemize}
%\item \textbf{Coordinación entre múltiples VANTS:} La coordinación descentralizada se realizará siguiendo el enfoque de auto-ofertas presentada por \citeauthor{CINVESTAM2013}, ya que su portabilidad hace que pueda ejecutarse sin necesidad de un módulo central.
\end{itemize}

\section{El uso del sistema operativo robótico ROS}

El desarrollo de soluciones en robótica no es una tarea fácil, la escalabilidad y las capacidades de los robot son cada vez mayores, debido a la miniaturización de componentes que han permitido el desarrollo en robots con recursos limitados.

El uso de ROS (Robot Operating System) emerge como un componente esencial en el diseño y desarrollo de robótica, incluidos los vehículos aéreos no tripulados (VANTs). ROS proporciona una plataforma flexible y modular que facilita la integración de hardware, la implementación de algoritmos de control y la comunicación entre componentes en un entorno distribuido. Este sistema operativo robótico ofrece una amplia gama de herramientas y recursos que agilizan el desarrollo de aplicaciones, permitiendo concentrarse en la implementación de soluciones innovadoras en lugar de abordar problemas de infraestructura.

%En esta tesis, se destaca la importancia y los beneficios del uso de ROS en el diseño, implementación y evaluación de sistemas autónomos, subrayando su papel fundamental en la mejora de la interoperabilidad de estos diversos componentes que componen a un sistema autónomo.

El Sistema Operativo Robótico (ROS) se basa en programas modulares o procesos tratados como grafos de un nodo. Cada nodo en ejecución procesa información de forma paralela y la comparte con otros procesos con los que se comunica dentro del sistema.

A continuación se describen las principales características del Sistema Operativo Robótico (ROS):

\begin{itemize}\setlength{\itemsep}{-1mm}
\item Capaz de construir sistemas basados en múltiples procesos en diferentes equipos anfitriones conectados a través de una red, permitiendo realizar ejecuciones paralelas en distintas computadoras a bordo.
\item La codificación se puede llevar en diversos lenguajes de programación como: C/C++ y python.
\end{itemize}

Los principales conceptos en que ROS trabaja es bajo el paradigma publicador-suscriptor. A continuación se listan diversos tecnicismos usados en ROS.

\begin{itemize}\setlength{\itemsep}{-1mm}
\item \textbf{rosmaster}: Módulo de control, en el que cada nodo existente en el sistema debe registrarse, este módulo mantiene una lista de todos los nodos y tópicos en el sistema.
\item \textbf{Nodo}: Son módulos de programa que efectúan funciones específicas en la aplicación y comparten información a través de tópicos, servicios o por medio del servidor de parámetros. Su desarrollo es modular, cada nodo es un proceso independiente permitiendo que sea tolerante a fallos. La información generada por cada nodo puede ser publicada a través del intercambio de mensajes y otros nodos pueden suscribirse a ese tópico.
\item \textbf{Tópico}: Es el canal de comunicación usado para el intercambio de información entre los nodos. Un nodo recibirá los mensajes de un tópico previo a una suscripción al tópico.
\item \textbf{Mensaje}: Estructura de datos definida que puede contener tipos datos como enteros, flotantes, booleanos, puntos, entre otros.
\item \textbf{Servicios}: Responde a peticiones de un cliente, los servicios son contenidos dentro de un nodo (servicio) que responde con información de respuesta a la solicitud de otros nodos (clientes). 
\item \textbf{Paquetes}: Contiene los archivos de configuración, y los programas (nodos) para la funcionalidad del robot.
  
\end{itemize}

\textit{\textbf{Desarrollar..}}

