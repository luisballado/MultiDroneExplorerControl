\chapter{Enfoque propuesto}

El enfoque propuesto para la tesis de exploración coordinada multi-VANT se centra en el desarrollo de un sistema autónomo que emplea varios Vehículos Aéreos No Tripulados (VANTS) para explorar y mapear entornos desconocidos de manera coordinada. Después de identificar técnicas y algoritmos relevantes, seguido del diseño detallado del sistema y su implementación en un entorno simulado.

El diseño e implementación de un sistema autónomo que integre estas tecnologías avanzadas permite abordar los desafíos inherentes a la exploración coordinada con múltiples VANT y complementado con una distrubución de tareas descentralizado.

En resumen, el enfoque propuesto aborda el desafío de la exploración coordinada multi-VANT mediante la integración de algoritmos avanzadas y la evaluación práctica de su desempeño en diferentes ambientes.


\section{Revisión de la literatura}

Realizar una revisión exhaustiva de la literatura sobre técnicas y algoritmos existentes para la exploración coordinada con múltiples VANT. Esto incluiría investigar enfoques de planificación de trayectorias, técnicas de mapeo y localización simultáneos (SLAM), y métodos de coordinación y comunicación entre VANT.

Dentro de la revisión expuesta en el estado del arte se localizaron los siguientes algoritmos que tomaron nuestro interes.

A*, E-scaling, diagramas de voronoi, ?? 

\section{Diseño del sistema}

Desarrollar un diseño detallado del sistema que incluya la arquitectura general, los componentes individuales y los algoritmos específicos que se utilizarán para la planificación de trayectorias, el mapeo del entorno y la coordinación entre los VANT.

\subsection*{Odometría VANT}
