\chapter{Enfoque propuesto}

El enfoque propuesto para la tesis de exploración coordinada multi-VANT se centra en el desarrollo de un sistema autónomo que emplea varios Vehículos Aéreos No Tripulados (VANT) para explorar y mapear entornos desconocidos de manera eficiente y coordinada. Este enfoque implica una revisión exhaustiva de la literatura para identificar técnicas y algoritmos relevantes, seguido del diseño detallado del sistema y su implementación en un entorno simulado. Las pruebas en el mundo real permitirán validar el rendimiento del sistema en condiciones más realistas, mientras que el análisis de resultados proporcionará información sobre la eficiencia, precisión y robustez del sistema. En resumen, el enfoque propuesto aborda el desafío de la exploración coordinada multi-VANT mediante la integración de tecnologías avanzadas y la evaluación práctica de su desempeño en diferentes contextos.


\section{Revisión de la literatura}

Realizar una revisión exhaustiva de la literatura sobre técnicas y algoritmos existentes para la exploración coordinada con múltiples VANT. Esto incluiría investigar enfoques de planificación de trayectorias, técnicas de mapeo y localización simultáneos (SLAM), y métodos de coordinación y comunicación entre VANT.

\section{Diseño del sistema}

Desarrollar un diseño detallado del sistema que incluya la arquitectura general, los componentes individuales y los algoritmos específicos que se utilizarán para la planificación de trayectorias, el mapeo del entorno y la coordinación entre los VANT.
