%% LyX 1.6.7 created this file.  For more info, see http://www.lyx.org/.
%% Do not edit unless you really know what you are doing.
\documentclass[spanish,ESP,MSc]{cinvestav} %si quieren en inglés, usen english, ENG
\usepackage{cite}
\usepackage{slashbox}
\usepackage[T1]{fontenc}
\usepackage[utf8]{inputenc}
\setcounter{secnumdepth}{3}
\setcounter{tocdepth}{3}
\usepackage{graphics,epsfig,latexsym,amssymb}
%\usepackage{graphics,epsfig,latexsym,amssymb}
\usepackage{tabularx}
\usepackage{titlesec}
\usepackage{amsmath}
%\usepackage[hypertex]{hyperref}
\usepackage{array}
\usepackage{float}
\usepackage{rotfloat}
\usepackage{textcomp}
\usepackage{graphicx}
\usepackage{color}
\usepackage{xcolor}
\usepackage{setspace}
\usepackage[TABBOTCAP]{subfigure}
\usepackage{algorithm}
\usepackage{captcont}
\usepackage{afterpage}
\usepackage{lscape}
\usepackage{algorithmic}
\usepackage{longtable}
%\usepackage{caption}
\usepackage{caption}[=v1]
\usepackage[authoryear,square]{natbib}

%\renewcommand{\thesubfigure}{\thefigure.\arabic{subfigure}} 
%\makeatletter 
%\renewcommand{\p@subfigure}{} 
%\renewcommand{\@thesubfigure}{\thesubfigure:\hskip\subfiglabelskip} 
%\makeatother 

%\usepackage[nonamebreak]{natbib}


\makeatletter

%%%%%%%%%%%%%%%%%%%%%%%%%%%%%% LyX specific LaTeX commands.
%% Because html converters don't know tabularnewline
\providecommand{\tabularnewline}{\\}
%% A simple dot to overcome graphicx limitations
\newcommand{\lyxdot}{.}

\floatstyle{ruled}
\newfloat{algorithm}{tbp}{loa}
\floatname{algorithm}{Algorithm}

%%%%%%%%%%%%%%%%%%%%%%%%%%%%%% User specified LaTeX commands.

%\usepackage{apacite}
%\usepackage{natbib}
%\let\citep=\citeA
%\let\cite=\citeA
\bibpunct{[}{]}{,}{n}{,}{,} 

%\usepackage[lined,boxed]{algorithm2e}
%\decimalpoint 

\title{Estrategias para la exploración coordinada multi-VANT}
\titleen        {A Study on the Mechanisms of two Multi-objective Evolutionary Algorithms}
\author         {Luis Alberto Ballado Aradias}
\department     {Cinvestav Unidad Tamaulipas}
\departmenten   {Cinvestav Unidad Tamaulipas}
\degreein       {Ciencias en Ingeniería y Tecnologías Computacionales}
\degreeinen     {Computer engineering and technology}
\city           {Cd. Victoria, Tamaulipas, M\'{e}xico.}
\date           {\today}
\degreeday      {8}
\degreeyear     {2024}
\degreemonth  {Septiembre}
\degreemonthen  {September}

%A partir de 2015, este fondo ya no se utiliza
%\acknowledgmenttoproject{
%This research was partially funded by project number 51623 from ``Fondo Mixto
%Conacyt-Gobierno del Estado de Tamaulipas''
%}

%%%%%%%%%%%%%%%%%%%%%%%%%%%%%%%%%%%%%%%%%%%%%%%%%%%%%%%%%%%%%%%%%%%%%%%%

\chair          {Dr. José Gabriel Ramirez-Torres}
\chair          {Dr. Eduardo Arturo Rodriguez-Tello}
\member         {Dr. Ricardo Landa-Becerra}
\member         {Dr. Mario Garza-Fabre}

%%%%%%%%%%%%%%%%%%%%%%%%%%%%%%%%%%%%%%%%%%%%%%%%%%%%%%%%%%%%%%%%%%%%%%%%

\dedication     {A mi familia}

%%%%%%%%%%%%%%%%%%%%%%%%%%%%%%%%%%%%%%%%%%%%%%%%%%%%%%%%%%%%%%%%%%%%%%%%

\abstract{
\vspace*{-7mm}
La \emph{Optimización Mediante Cúmulos de Partículas (PSO)} y la \emph{Evolución Diferencial (DE)} son dos \emph{Algoritmos Evolutivos (EAs)} simples de conceptualizar que han sido exitosamente utilizados para resolver problemas mono-objetivo. Dicha simplicidad y éxito han promovido su uso en problemas multi-objetivo. Aun cuando a la fecha existen varias propuestas \emph{PSO Multi-objetivo (MOPSO)} y \emph{DE Multi-objetivo (MODE)}, el conocimiento sobre el proceso de búsqueda que realizan estas dos metaheurísticas es escaso dado que solo existen algunos trabajos teóricos y únicamente mono-objetivo. Como resultado, no se conoce claramente el comportamiento de estos algoritmos evolutivos en problemas multi-objetivo. Esta tesis presenta un estudio empírico sobre estos dos \emph{Algoritmos Evolutivos Multi-objetivo (MOEAs)}. Dicho estudio consta de una serie de experimentos que comparan diferentes variantes en DE y fórmulas de vuelo en PSO. Después, se evalúa la manera en que estos dos MOEAs generan nuevas soluciones y se identifican características de dichas soluciones y su relación con los mecanismos presentes en ambos enfoques. Estos experimentos permitieron concluir que MOPSO se mueve agresivamente hacia regiones prometedoras lo cual puede deteriorar la búsqueda. MODE por otro lado, realiza una búsqueda pasiva basada en pasos pequeños que a la larga le permiten seguir moviéndose hacia el frente de Pareto. El conocimiento obtenido fue usado para diseñar dos nuevos MOEAs que mostraron ser competitivos al ser comparados con tres algoritmos (OMOPSO, NSGA-II y DEMO) representativos del estado del arte. 
}

\abstracten{
\vspace*{-7mm}
\emph{Particle Swarm Optimization (PSO)} and \emph{Differential Evolution (DE)} are two \emph{Evolutionary Algorithms (EAs)} which are very simple to conceptualize and have shown excellent results on single-objective optimization problems. As expected, this simplicity and success have promoted their migration to multi-objective optimization. Even when several \emph{Multi-Objective Particle Swarm Optimizers (MOPSOs)} and \emph{Multi-objective Differential Evolution (MODE)} algorithms are available to this date, knowledge about the search performed by these two metaheuristics is limited in regards to multi-objective optimization since only some theoretical single-objective studies have been developed. As a result, there is uncertainty in regards to the search behavior of these \emph{Multi-objective Evolutionary Algorithms (MOEAS)}. This thesis work performs an empirical study about the search of these two MOEAs. The performed analysis develops a series of experiments that compare several DE variants and PSO flight formulas. Thereafter, the manner in which these two MOEAs generate new solutions is evaluated and certain characteristics of these solutions and their relationship to the mechanisms found on a MOEA are identified. These experiments allowed to conclude that MOPSO performs an aggressive search towards promissory regions which might result in stagnation of the search. On the other hand, MODE performs a more passive search taking small steps which on the long run allowed it to continue improving towards the true Pareto front. This knowledge was further used to design two new MOEAs. Results indicate that both algorithms are very competitive with respect to three algorithms (OMOPSO, NSGA-II, and DEMO) representative of the state of the art. 
}

%Las publicaciones generadas por este trabajo de investigación deben publicarse en la sección de 'Anexos'

\acknowledgments {
\begin{itemize}
\item { Le agradezco a spiderman y pongan m\'as agradecimientos como los siguientes:}
\item { I also thank the administrative personnel at CINVESTAV-Tamaulipas for their help during my stay}
\item { I thank CONACyT for the provided economic support which allowed me to concentrate in my studies and CINVESTAV-Tamaulipas for the opportunity to pursue graduate studies}
\item { I also acknowledge support from CONACyT through project 105060 ``Uso de técnicas evolutivas híbridas para resolver problemas de optimización multiobjetivo dinámicos y con más de tres objetivos'' under the lead of Dr. Gregorio Toscano Pulido}
\end{itemize}
}

\nomenclature {
\begin{longtable}{ll}
\textbf{BBDE} & Bare Bones Differential Evolution\tabularnewline
\textbf{BBPSO} & Bare Bones Particle Swarm Optimization\tabularnewline
\textbf{DelMiDE} & Delayed Micro Differential Evolution\tabularnewline
\textbf{DTLZ} & Deb-Thiele-Leumman-Zitzler test suite\tabularnewline
\textbf{EA} & Evolutionary Algorithm\tabularnewline
\textbf{EC} & Evolutionary Computation\tabularnewline
\textbf{ES} & Evolutionary Strategy\tabularnewline
\textbf{EP} & Evolutionary Programming\tabularnewline
\textbf{DE} & Differential Evolution\tabularnewline
\textbf{DEMO} & An specific implementation of Multi-objective Differential Evolution\tabularnewline
\textbf{GA} & Genetic Algorithm\tabularnewline
\textbf{GD} & Generational Distance\tabularnewline
\textbf{IFF} & If and only if\tabularnewline
\textbf{IGD} & Inverted Generational Distance\tabularnewline
\textbf{MODE} & Multi-objective Differential Evolution\tabularnewline
\textbf{MOEA} & Multi-objective Evolutionary Algorithm\tabularnewline
\textbf{MOP} & Multi-objective Optimization Problem\tabularnewline
\textbf{MOPEDS} & Multi-objective Particle Swarm Optimization Enhanced with a Differential Evolution Scheme\tabularnewline
\textbf{MOPSO} & Multi-objective Particle Swarm Optimizer\tabularnewline
\textbf{NSGA-II} & Non-dominated Sorting Genetic Algorithm\tabularnewline
\textbf{PF} & Pareto Optimal Front\tabularnewline
\textbf{PS} & Pareto Optimal Set\tabularnewline
\textbf{PSO} & Particle Swarm Optimization\tabularnewline
\textbf{SMPSO} & Speed constrained Multi-objective Particle Swarm Optimization\tabularnewline
\textbf{WRT} & With Respect To\tabularnewline
\textbf{ZDT} & Zitzler-Deb-Thiele test suite\tabularnewline
\tabularnewline
\tabularnewline
\end{longtable}
}

 \floatname{algorithm}{Algorithm}
%\renewcommand{\algorithmicrequire}{\textbf{Entrada:}}
%\renewcommand{\algorithmicensure}{\textbf{Salida:}}

%\@ifundefined{showcaptionsetup}{}{%
 %\PassOptionsToPackage{caption=false}{subfig}}
%\usepackage{subfig}
\makeatother

%\usepackage{babel}
%\addto\shorthandsspanish{\spanishdeactivate{~<>}}

\setcounter{lofdepth}{1} 
\setcounter{lotdepth}{1} 

\begin{document}
%\maketitle
\makeintropages

%\renewcommand{\tablename}{Tabla}

\begin{doublespace}
\chapter{Introducción}

Los robots de servicio son máquinas autónomas diseñadas con el objetivo de prestar servicio a los humanos fuera del ambiente industrial, convirtiéndose poco a poco en una parte esencial en nuestras vidas. Los podemos encontrar en diversos ámbitos, como en el entretenimiento, limpieza, logística, entre otras soluciones inovadoras.

Los vehículos aéreos no tripulados (VANTS) han evolucionado rápidamente y se han convertido en sistemas versátiles capaces de una amplia gama de aplicaciones, desde vigilancia hasta misiones de búsqueda y rescate. Entre ellas, las tareas de exploración en entornos complejos y dinámicos representan un área interesante y desafiante, donde la coordinación de múltiples vehículos aéreos no tripulados se vuelve primordial. Este tema es de creciente importancia a medida que los vehículos aéreos no tripulados (VANTS) continúan transformando distintas áreas, incluida la agricultura, el monitoreo ambiental, mantenimiento de infraestructuras (puentes, edificios, líneas eléctricas) y la respuesta a desastres naturales, reduciendo los riesgos y costos asociados con las inspecciones manuales.

Los vehículos aéreos no tripulados (VANTS) capaces de realizar tareas con autonomía generalmente cuentan con mayores capacidades de carga y procesamiento computacional, así como sensores capaces de percibir grandes volúmenes de datos en un tiempo reducido. Estos vehículos aéreos no tripulados (VANTS) se centran en realizar tareas sencillas y estáticas en áreas abiertas con rutas predeterminadas, o bien, en contextos de operación por control remoto por un usuario. Sin embargo en donde los espacios son estrechos, se optan por el uso de VANTS reducidos, comúnmente llamados Micro-vehículos aéreos no tripulados (MAVs).

La exploración de áreas desconocidas, a través de la sinergia de sistemas con múltiples vehículos aéreos no tripulados (Multi-VANTS) promete ser una solución innovadora. Al comprender y perfeccionar una estrategia para la coordinación de múltiples vehículos aéreos no tripulados (Multi-VANTS) en tareas de exploración, esperamos descubrir nuevas posibilidades, replicar o romper los límites existentes, y en última instancia, avanzar en los campos de la robótica y la exploración.

\section{Antecedentes y motivación} 

\subsection*{Antecedentes}

La robótica móvil es una rama de la robótica que se enfoca en el diseño, construcción, programación y operación de robots capaces de moverse de manera autónoma o semi-autónoma en entornos diversos.

Uno de los primeros hitos importantes en la robótica móvil fué el desarrollo del robot shakey en la década de 1970, que se considera el primer robot móvil capaz de desplazarse evitando colisiones a su paso, sentando las bases en algoritmos de inteligencia artificial para búsquedas informadas surgiendo el algoritmo A*. 

Durante las décadas siguientes, se produjeron avances significativos en la miniaturización de componentes electrónicos, sensores y actuadores, lo que permitió la creación de robots móviles más pequeños y versátiles. A finales del siglo XX, los robots móviles comenzaron a ser utilizados en una variedad de aplicaciones, como la exploración espacial, la agricultura, la vigilancia y la logística.

En paralelo, los avances en inteligencia artificial, visión por computadora, planificación de trayectorias y sistemas de control contribuyeron al desarrollo de robots móviles más autónomos y adaptables. El surgimiento de algoritmos de aprendizaje automático y técnicas de percepción avanzada ha permitido a los robots móviles interactuar de manera más efectiva con su entorno y tomar decisiones en tiempo real.

En la actualidad, la robótica móvil está experimentando un rápido crecimiento gracias a la convergencia de diversas tecnologías, como la computación en la nube y la robótica colaborativa. Se están desarrollando robots móviles cada vez más sofisticados y capaces de realizar una amplia gama de tareas en entornos dinámicos y no estructurados. Además, se espera que la robótica móvil desempeñe un papel crucial en aplicaciones futuras, como la asistencia en el hogar, la atención médica, la exploración submarina y la entrega de paquetes.

Un sistema autónomo de un vehículo aéreo no tripulado, consta de cuatro algoritmos:
\begin{itemize}\setlength{\itemsep}{-1mm}
\item Generación de una representación del medio ambiente
\item Planificación de trayectorias
\item Evasión de obstáculos
\item Comunicación
\end{itemize}

La computadora embebida para un sistema de navegación usado en vehículos aéreos no tripulados de menor tamaño, son de bajo rendimiento. Pero, su necesidad de autonomía sigue siendo la misma que un VANT de mayor tamaño. Es por ello que es necesario equiparlos con algoritmos de baja complejidad computacional.

La necesidad de coordinación entre múltiples VANTS en tareas de exploración, surge debido a las limitaciones individuales en cuanto a la extensión de terreno que pueden cubrir y su desempeño. La exploración de áreas extensas o peligrosas a menudo exige un enfoque colaborativo, donde los VANTS trabajen juntos para optimizar la asignación de recursos y mejorar la recopilación y el análisis de datos.

Sin embargo, el camino hacia una coordinación entre múltiples vehículos aéreos no tripulados, presenta diversos desafíos. Las complejidades de gestionar un grupo de vehículos aéreos no tripulados, navegar en entornos dinámicos y distribuir tareas de forma inteligente son sólo algunas de las cuestiones que exigen nuestra atención. %El objetivo del documento es explorar y proponer una estrategia para mejorar la coordinación de múltiples VANTS en el contexto de las tareas de exploración.
% usar --> para resaltar cosas \textbf{\emph{}}

\subsection*{Motivación}

El potencial del uso de los vehículos aéreos no tripulados en tareas de búsqueda y rescate, inspección, mapeo, vigilancia, entre otras, es de gran interés a explorar, debido a las habilidades de vuelo que presentan en favor de la realización de estas tareas, y en especial situaciones que podrían poner en riesgo a personas.

Enviar personal de rescate dentro de un edificio parcialmente colapsado en busca de sobrevivientes, es poner a más personas en un gran riesgo, pues no se sabe qué es lo que les espera en el interior del edificio; esto limita la capacidad de tomar buenas decisiones acerca de si es seguro seguir cierto camino.

Operar en ambientes como éste u otros similares requieren de robots con capacidades de navegar sobre terrenos difíciles y evadir obstáculos de forma segura para obtener información del entorno que pueda ser útil al personal de rescate.

Una posible solución consiste en un robot móvil aéreo (VANT) capaz de desplazarse sobre terrenos difíciles y navegar en espacios cerrados de manera segura, que además recabe información del entorno. Esto implica realizar tareas de reconocimiento del ambiente, evasión de obstáculos y seguimiento de trayectorias.

\section{Planteamiento del problema} 

Dado un volumen de interés desconocido en un espacio cerrado que se desea explorar denotado como $\mathcal{W}$, tal que $\mathcal{W} \subset \mathbb{R}^{3}$.
%el volumen se discretiza en un mapa de ocupación M consistente en voxeles cúbicos $m \exists M$ con una longitud de arista r.
\begin{itemize}\setlength{\itemsep}{-1mm}
  \item El volumen se discretiza usando unidades cúbicas tridimencionales (voxel) tomando valores $v_{libre}$, $v_{ocup}$, $v_{desc}$.
    %La representación del volumen a explorar se obtiene dividiendo el volumen de interés en unidades cúbicas tridimencionales (voxel) que puede tomar los valores de libre $v_{libre}$, ocupado $v_{ocup}$ y desconocido $v_{desc}$ con lecturas a partir de los valores de una cámara RGB-D basada en un modelo de ocupación probabilístico.\\
  \item Un conjunto de VANTS con una cámara RGB-D embarcadas denotado como $\mathcal{V} = \{\mathcal{V}_{1},\mathcal{V}_{2},\mathcal{V}_{3},...,\mathcal{V}_{n}\}$, comenzando cada uno en un estado inicial conocido $q = \{q_{1},q_{2},q_{3},...,q_{n}\}$, y terminando en una configuración que maximice la construcción de un mapa.
\end{itemize}

Coordinar el conjunto de VANTs para reducir el tiempo total de exploración.
  
  
\section{Hipótesis y preguntas de investigación}

\subsection*{Hipótesis}
\emph{``Una estrategia que coordine y asigne tareas de exploración para múltiples VANTS de manera descentralizada, en combinación con una arquitectura de software diseñada para resolver problemas de localización, gestión de mapas y planificación de rutas, mejorará la eficiencia y cobertura de la exploración en interiores de un entorno desconocido''}.


\subsection*{Preguntas de investigación}

\begin{enumerate}\setlength{\itemsep}{-1mm}
\item ¿Qué características de la dinámica del VANT son cruciales para lograr trayectorias suaves y continuas?
\item ¿Podría un planificador de trayectorias que aproveche las regiones libres de obstáculos acelerar los desplazamientos de los VANTs y, consecuentemente, reducir los tiempos de exploración?
\item ¿Qué mecanismos de coordinación existen dentro de la literatura que podrían ayudar en resolver el problema de exploración multi-VANT?
\end{enumerate}

\section{Objetivos}

\subsection*{Objetivo general}

%El principal objetivo del trabajo de tesis se define a continuación:
\emph{``Desarrollar una estrategia de exploración descentralizada que permita resolver los problemas de coordinación para múltiples VANTS en ambientes desconocidos.''}.

\subsection*{Objetivos específicos}

Para lograr nuestro objetivo principal, se consideran los siguientes objetivos especificos a cubrir:

\begin{itemize}\setlength{\itemsep}{-1mm}
\item Desarrollar una arquitectura de software que resuelva los problemas de autonomía para un VANT (localización, manejo de mapas y planificación de trayectorias).
\item Implementar un mecanismo de coordinación descentralizado que asigne tareas de exploración.
\item Realizar pruebas y simulaciones de la solución propuesta en diversos entornos, analizando la relación tiempo de exploración y cobertura del área de interés.
\end{itemize}

\newpage

\section{Solución propuesta}

Para resolver el problema de exploración multi-VANT con un enfoque descentralizado considerando restricciones en el rango de la comunicación. Se cuenta como antecedente el trabajo doctoral de \citeauthor{CINVESTAM2013} que propone un algoritmo basado en un proceso de ofertas de mercado, en el cual cada robot calcula las ofertas de manera independiente, buscando alcanzar el mayor aporte posible al equipo en su conjunto. Cuando un robot alcanza su objetivo, el robot toma una decisión para sí mismo, involucrando a cada uno de los miembros del equipo así como el rango de comunicación, bajo un esquema descentralizado y sin la necesidad de un módulo central.

Para validar la propuesta de exploración coordinada se necesita primero resolver la autonomía de un vehículo aéreo no tripulado diseñando una arquitectura que incluya la coordinación multi-robot propuesta en \cite{CINVESTAM2013}.

\begin{enumerate}\setlength{\itemsep}{-1mm}
\item Conocer los fundamentos que nos aproximen a realizar la tarea de exploración autónoma con múltiples VANTS.
\item Profundizar en la comprensión de los comandos de control y la generación de odometría para un VANT tipo cuadricóptero.
\item Obtener y procesar la información proveniente de un sensor de tipo RGB-D dentro del sistema operativo ROS.
\item Integrar un planificador de trayectoria reactivo que, combinado con la percepción recibida por la cámara RGB-D, nos permita evadir obstáculos en su paso y construir una representación tridimensional del medio ambiente.
\item Elaborar la exploración con un VANT de tipo cuadrotor.
\item Implementar la estrategia de exploración bajo los conceptos de cohesión, exploración y explotación.
\end{enumerate}



%\section{Resumen}
%\lipsum[2-4]

%The rest of this document is organized as follows: Chapter~\ref{chapter2} presents basic concepts and background in the field of optimization. Then, Chapter~\ref{ch:PSODE} introduces particle swarm optimization and differential evolution which are the two metaheuristics on which this thesis work focusses. In order to introduce these two metaheuristics, EAs is general are also described in this chapter. Afterwards, Chapter~\ref{experiments} presents a series of experiments that were developed and that allowed to obtain further information about the search performed by PSO and DE in multi-objective optimization. This knowledge was used to develop two new MOEAs which are presented and evaluted in Chapter~\ref{proposals}. Finally, Chapter~\ref{conclusion} concludes this thesis work.


%\chapter{Marco Teórico}

\lipsum[2-4]

%%%%%%%%%%%%%%%%%%%%%%%
%%%%%%%%%%%%%%%%%%%%%%%
%%%%%%%%%%%%%%%%%%%%%%%
%%%%%%%%%%%%%%%%%%%%%%%
%%%%%%%%%%%%%%%%%%%%%%%

\lipsum[2-4]

\section{Conceptos fundamentales} \label{}

\lipsum[2-4]

\subsection{Sistema de ejes coordenados}

\lipsum[2-4]

\subsection{Coordenadas homogéneas}

\lipsum[2-4]

\subsection{Transformaciones geométricas}

\lipsum[2-4]

\section{Funcionamiento de un VANT}

\lipsum[2-4]

\section{Control de un VANT}

\lipsum[2-4]

\section{Estimación de posición}

\lipsum[2-4]



%\chapter{Enfoque propuesto}

El enfoque propuesto para la tesis de exploración coordinada multi-VANT se centra en el desarrollo de un sistema autónomo que emplea varios Vehículos Aéreos No Tripulados (VANTS) para explorar y mapear entornos desconocidos de manera coordinada. Después de identificar técnicas y algoritmos relevantes, seguido del diseño detallado del sistema y su implementación en un entorno simulado.

El diseño e implementación de un sistema autónomo que integre estas tecnologías avanzadas permite abordar los desafíos inherentes a la exploración coordinada con múltiples VANT y complementado con una distrubución de tareas descentralizado.

En resumen, el enfoque propuesto aborda el desafío de la exploración coordinada multi-VANT mediante la integración de algoritmos avanzadas y la evaluación práctica de su desempeño en diferentes ambientes.


\section{Revisión de la literatura}

Realizar una revisión exhaustiva de la literatura sobre técnicas y algoritmos existentes para la exploración coordinada con múltiples VANT. Esto incluiría investigar enfoques de planificación de trayectorias, técnicas de mapeo y localización simultáneos (SLAM), y métodos de coordinación y comunicación entre VANT.

Dentro de la revisión expuesta en el estado del arte se localizaron los siguientes algoritmos que tomaron nuestro interes.

A*, E-scaling, diagramas de voronoi, ?? 

\section{Diseño del sistema}

Desarrollar un diseño detallado del sistema que incluya la arquitectura general, los componentes individuales y los algoritmos específicos que se utilizarán para la planificación de trayectorias, el mapeo del entorno y la coordinación entre los VANT.

\subsection*{Odometría VANT}


%\include{chapter4}

%\include{chapter5}

%\include{chapter6}

%\include{chapter7}

\end{doublespace}

\appendix

\begin{doublespace}
\chapter{Publicaciones}
%Dar el formato que considere apropiado. Presentar, por ejemplo: Conferencias, Revistas internacionales, etc...
\begin{itemize}
\item Jorge Sebastian Hernández Domínguez and Gregorio Toscano Pulido. \emph{A Comparison on the Search of Particle Swarm Optimization and Differential Evolution on Multi-Objective Optimization}, in IEEE Congress on Evolutionary Computation (CEC 2011), New Orleans, LA, USA, June 2011.
\item 
Jorge Sebastian Hernández Domínguez, Gregorio Toscano Pulido, and Carlos A. Coello Coello, \emph{A Multi-objective Particle Swarm Optimizer Enhanced with a Differential Evolution Scheme}, in International Conference on Artificial Evolution (EA 2011), Angers, France, October 2011.

\end{itemize}



%Las publicaciones generadas por este trabajo de investigación deben publicarse en la sección de 'Anexos'

%\include{appendixA}

%\include{appendixB}

%\bibliographystyle{apalike}
\bibliography{EMOO,protocolo}
\end{doublespace}

\end{document}
