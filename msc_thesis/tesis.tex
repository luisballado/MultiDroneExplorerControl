%% LyX 1.6.7 created this file.  For more info, see http://www.lyx.org/.
%% Do not edit unless you really know what you are doing.

\documentclass[spanish,ESP,MSc]{cinvestav} %si quieren en inglés, usen english, ENG
\usepackage{cite}
\usepackage{slashbox}
\usepackage[T1]{fontenc}
\usepackage[utf8]{inputenc}
\setcounter{secnumdepth}{3}
\setcounter{tocdepth}{3}
\usepackage{graphics,epsfig,latexsym,amssymb}
\usepackage{tabularx,booktabs}
\usepackage{titlesec}
\usepackage{amsmath}
\usepackage{blindtext}
\usepackage{hyperref}
%\usepackage[hypertex]{hyperref}
\usepackage{array}
\usepackage{float}
\usepackage{rotfloat}
\usepackage{textcomp}
\usepackage{graphicx}
\usepackage{color}
\usepackage{xcolor,colortbl}
%%
\usepackage{bbding} %palomitas checkmark
\usepackage{pifont}
\usepackage{multicol}
\usepackage{array} % needed for \arraybackslash
\usepackage{adjustbox} % for \adjincludegraphics
\usepackage{enumitem}% http://ctan.org/pkg/enumitem
\usepackage{forloop}
\usepackage{pdflscape}
\newcounter{loopcntr}
%%
\usepackage{setspace}
\usepackage[TABBOTCAP]{subfigure}
\usepackage{algorithm}
\usepackage{captcont}
\usepackage{afterpage}
\usepackage{lscape}
\usepackage{algorithmic}
\usepackage{longtable}
%\usepackage{caption}
\usepackage{caption}[=v1]
\usepackage[authoryear,square]{natbib}

\usepackage{lipsum}

%%% -()-
\newcommand{\rpt}[2][1]{%
  \forloop{loopcntr}{0}{\value{loopcntr}<#1}{#2}%
}
\newcommand{\on}[1][1]{
  \forloop{loopcntr}{0}{\value{loopcntr}<#1}{&\cellcolor{gray}}
}
\newcommand{\off}[1][1]{
  \forloop{loopcntr}{0}{\value{loopcntr}<#1}{&}
}

%\addtolength{\textheight}{90pt}

\newcommand{\specialcell}[2][c]{%
  \begin{tabular}[#1]{@{}c@{}}#2\end{tabular}}
%%% -()-

%\renewcommand{\thesubfigure}{\thefigure.\arabic{subfigure}} 
%\makeatletter 
%\renewcommand{\p@subfigure}{} 
%\renewcommand{\@thesubfigure}{\thesubfigure:\hskip\subfiglabelskip} 
%\makeatother 

%\usepackage[nonamebreak]{natbib}


\makeatletter

%%%%%%%%%%%%%%%%%%%%%%%%%%%%%% LyX specific LaTeX commands.
%% Because html converters don't know tabularnewline
\providecommand{\tabularnewline}{\\}
%% A simple dot to overcome graphicx limitations
\newcommand{\lyxdot}{.}

\floatstyle{ruled}
\newfloat{algorithm}{tbp}{loa}
\floatname{algorithm}{Algorithm}

%%%%%%%%%%%%%%%%%%%%%%%%%%%%%% User specified LaTeX commands.

%\usepackage{apacite}
%\usepackage{natbib}
%\let\citep=\citeA
%\let\cite=\citeA
\bibpunct{[}{]}{,}{n}{,}{,} 

%\usepackage[lined,boxed]{algorithm2e}
%\decimalpoint 

\title{Estrategias para la exploración coordinada multi-VANT}
\titleen        {Strategies for coordinated multi-UAV exploration}
\author         {Luis Alberto Ballado Aradias}
\department     {Cinvestav Unidad Tamaulipas}
\departmenten   {Cinvestav Unidad Tamaulipas}
\degreein       {Ciencias en Ingeniería y Tecnologías Computacionales}
\degreeinen     {Computer engineering and technology}
\city           {Cd. Victoria, Tamaulipas, México.}
\date           {\today}
\degreeday      {8}
\degreeyear     {2024}
\degreemonth    {Septiembre}
\degreemonthen  {September}

%A partir de 2015, este fondo ya no se utiliza
\acknowledgmenttoproject{}

%%%%%%%%%%%%%%%%%%%%%%%%%%%%%%%%%%%%%%%%%%%%%%%%%%%%%%%%%%%%%%%%%%%%%%%%

\chair          {Dr. José Gabriel Ramirez-Torres}
\chair          {Dr. Eduardo Arturo Rodriguez-Tello}
\member         {Dr. Mario Garza-Fabre}
\member         {Dr. Ricardo Landa-Becerra}

%%%%%%%%%%%%%%%%%%%%%%%%%%%%%%%%%%%%%%%%%%%%%%%%%%%%%%%%%%%%%%%%%%%%%%%%

\dedication     {A mi familia}

%%%%%%%%%%%%%%%%%%%%%%%%%%%%%%%%%%%%%%%%%%%%%%%%%%%%%%%%%%%%%%%%%%%%%%%%

\abstract{
\vspace*{-7mm}
En la última década se ha tenido un aumento en la investigación y el desarrollo en el campo de los vehículos aéreos no tripulados (VANTS), lo que ha dado lugar a importantes avances e innovaciones en esta área. Los sistemas con múltiples vehículos aéreos no tripulados (multi-VANT) permiten la adquisición simultánea de datos desde múltiples puntos de vista privilegiados, en comparación con robots terrestres, lo que permite mejorar la generación de mapas de entornos desconocidos. El uso de algoritmos de coordinación inteligente y la toma descentralizada de decisiones proporciona una mayor confiabilidad a estos sistemas, ya que cualquier falla o mal funcionamiento de un VANT puede ser compensado por los VANTS restantes. Además, un buen protocolo de comunicación permite una colaboración fluida entre los robots, mejorando su capacidad para moverse de un lugar a otro evitando obstáculos a su paso (navegar), descubrir un entorno desconocido adquiriendo información a su paso (explorar). Asimismo, la integración de sensores de última generación los combierte en herramientas útiles en diversos dominios que van desde el entretenimiento, hasta aplicaciones de vigilancia. La mayor parte de las aplicaciones que hacen uso de VANTS suelen carecer de autonomía, requiriendo la asistencia y vigilancia constantes de un usuario humano. Para que un robot se considere autónomo deberá tomar decisiones y realizar tareas sin necesidad de que alguien le diga qué hacer o guiarlo paso a paso.
%Tener la capacidad de percibir su entorno y usar la información para decidir cómo moverse son considerados altos niveles de autonomía. Para llegar a ello, el robot debe resolver primero problemas como su localización, construir el mapa de su entorno y posteriormente usarlo para navegar dentro de él.\\

  %La exploración multi-robot ha surgido como un enfoque prometedor para la generación eficiente del mapa de un medio ambiente desconocido. Un enfoque colaborativo ofrece mejores resultados de exploración con una rápida obtención de información, logrando sus objetivos con un alto grado de consistencia y resilencia a fallos en comparación con implementaciones donde se emplea un único robot. Sin embargo, la exploración multi-robot plantea diversos desafíos que deben abordarse para su correcta implementación, como la comunicación, la colaboración y la fusión de datos.\\
  
  El enfoque de este trabajo es la propuesta de una estrategia descentralizada, capaz de coordinar múltiples vehículos aéreos no tripulados con habilidades para la exploración, generación de mapas y planificación de trayectorias para explorar eficientemente un área de interés.
  
%La \emph{Optimización Mediante Cúmulos de Partículas (PSO)} y la \emph{Evolución Diferencial (DE)} son dos \emph{Algoritmos Evolutivos (EAs)} simples de conceptualizar que han sido exitosamente utilizados para resolver problemas mono-objetivo. Dicha simplicidad y éxito han promovido su uso en problemas multi-objetivo. Aun cuando a la fecha existen varias propuestas \emph{PSO Multi-objetivo (MOPSO)} y \emph{DE Multi-objetivo (MODE)}, el conocimiento sobre el proceso de búsqueda que realizan estas dos metaheurísticas es escaso dado que solo existen algunos trabajos teóricos y únicamente mono-objetivo. Como resultado, no se conoce claramente el comportamiento de estos algoritmos evolutivos en problemas multi-objetivo. Esta tesis presenta un estudio empírico sobre estos dos \emph{Algoritmos Evolutivos Multi-objetivo (MOEAs)}. Dicho estudio consta de una serie de experimentos que comparan diferentes variantes en DE y fórmulas de vuelo en PSO. Después, se evalúa la manera en que estos dos MOEAs generan nuevas soluciones y se identifican características de dichas soluciones y su relación con los mecanismos presentes en ambos enfoques. Estos experimentos permitieron concluir que MOPSO se mueve agresivamente hacia regiones prometedoras lo cual puede deteriorar la búsqueda. MODE por otro lado, realiza una búsqueda pasiva basada en pasos pequeños que a la larga le permiten seguir moviéndose hacia el frente de Pareto. El conocimiento obtenido fue usado para diseñar dos nuevos MOEAs que mostraron ser competitivos al ser comparados con tres algoritmos (OMOPSO, NSGA-II y DEMO) representativos del estado del arte. 
}

\abstracten{
\vspace*{-7mm}
In the last decade, there has been an increase in research and development in the field of Unmanned Aerial Vehicles (UAVs), leading to significant advancements and innovations in this area. Systems with multiple Unmanned Aerial Vehicles (multi-UAV) allow for the simultaneous acquisition of data from multiple vantage points, compared to ground robots, thus enhancing the generation of maps of unknown environments. The use of intelligent coordination algorithms and decentralized decision-making provides greater reliability to these systems, as any failure or malfunction of one UAV can be compensated for by the remaining UAVs. Additionally, effective communication protocols enable seamless collaboration among the robots, enhancing their ability to navigate from one place to another while avoiding obstacles (navigation) and to discover an unknown environment by acquiring information along the way (exploration). Likewise, the integration of state-of-the-art sensors turns them into useful tools in various domains like drones shows to vigilance and monitoring. Most applications that use UAVs often lack autonomy, requiring constant assistance and supervision from a human user. For a robot to be considered autonomous, it must make decisions and perform tasks without needing someone to tell it what to do or guide it step by step.
%Having the ability to perceive its environment and use the information to decide how to move are considered high levels of autonomy. To achieve this, the robot must first solve problems such as its localization, build the map of its environment, and then use it to navigate within it.\\

%Multi-robot exploration has emerged as a promising approach for efficiently generating a map of an unknown environment. A collaborative approach yields better exploration results with rapid information gathering, achieving its objectives with a high degree of consistency and resilience to failures compared to single-robot implementations. However, multi-robot exploration poses various challenges that must be addressed for its proper implementation, such as communication, collaboration, and data fusion.\\

The focus of this work is on proposing a decentralized strategy capable of coordinating multiple unmanned aerial vehicles with exploration, mapping, and trajectory planning capabilities to efficiently explore an area of interest.
%\emph{Particle Swarm Optimization (PSO)} and \emph{Differential Evolution (DE)} are two \emph{Evolutionary Algorithms (EAs)} which are very simple to conceptualize and have shown excellent results on single-objective optimization problems. As expected, this simplicity and success have promoted their migration to multi-objective optimization. Even when several \emph{Multi-Objective Particle Swarm Optimizers (MOPSOs)} and \emph{Multi-objective Differential Evolution (MODE)} algorithms are available to this date, knowledge about the search performed by these two metaheuristics is limited in regards to multi-objective optimization since only some theoretical single-objective studies have been developed. As a result, there is uncertainty in regards to the search behavior of these \emph{Multi-objective Evolutionary Algorithms (MOEAS)}. This thesis work performs an empirical study about the search of these two MOEAs. The performed analysis develops a series of experiments that compare several DE variants and PSO flight formulas. Thereafter, the manner in which these two MOEAs generate new solutions is evaluated and certain characteristics of these solutions and their relationship to the mechanisms found on a MOEA are identified. These experiments allowed to conclude that MOPSO performs an aggressive search towards promissory regions which might result in stagnation of the search. On the other hand, MODE performs a more passive search taking small steps which on the long run allowed it to continue improving towards the true Pareto front. This knowledge was further used to design two new MOEAs. Results indicate that both algorithms are very competitive with respect to three algorithms (OMOPSO, NSGA-II, and DEMO) representative of the state of the art. 
}

%Las publicaciones generadas por este trabajo de investigación deben publicarse en la sección de 'Anexos'

\acknowledgments {
\begin{itemize}
\item {}
\item {}
\item {} 
\end{itemize}
}

\nomenclature {
  \begin{longtable}{ll}
    \textbf{VANT} & Vehículo Aéreo No Trupulado\tabularnewline
    \textbf{MAVs} & Micro-Vehículos Aéreos no tripulados\tabularnewline
    \textbf{SLAM} & Simultaneous Localization and Mapping\tabularnewline
    \textbf{ESC} & POTENCIA MOTORES\tabularnewline
    \tabularnewline
    \tabularnewline
  \end{longtable}
}

\floatname{algorithm}{Algorithm}
%\renewcommand{\algorithmicrequire}{\textbf{Entrada:}}
%\renewcommand{\algorithmicensure}{\textbf{Salida:}}

%\@ifundefined{showcaptionsetup}{}{%
 %\PassOptionsToPackage{caption=false}{subfig}}
%\usepackage{subfig}
\makeatother

%\usepackage{babel}
%\addto\shorthandsspanish{\spanishdeactivate{~<>}}

\setcounter{lofdepth}{1} 
\setcounter{lotdepth}{1} 

\begin{document}
%\maketitle
\makeintropages

%\renewcommand{\tablename}{Tabla}

\begin{doublespace}

  \chapter{Introducción}

Los robots de servicio son máquinas autónomas diseñadas con el objetivo de prestar servicio a los humanos fuera del ambiente industrial, convirtiéndose poco a poco en una parte esencial en nuestras vidas. Los podemos encontrar en diversos ámbitos, como en el entretenimiento, limpieza, logística, entre otras soluciones inovadoras.

Los vehículos aéreos no tripulados (VANTS) han evolucionado rápidamente y se han convertido en sistemas versátiles capaces de una amplia gama de aplicaciones, desde vigilancia hasta misiones de búsqueda y rescate. Entre ellas, las tareas de exploración en entornos complejos y dinámicos representan un área interesante y desafiante, donde la coordinación de múltiples vehículos aéreos no tripulados se vuelve primordial. Este tema es de creciente importancia a medida que los vehículos aéreos no tripulados (VANTS) continúan transformando distintas áreas, incluida la agricultura, el monitoreo ambiental, mantenimiento de infraestructuras (puentes, edificios, líneas eléctricas) y la respuesta a desastres naturales, reduciendo los riesgos y costos asociados con las inspecciones manuales.

Los vehículos aéreos no tripulados (VANTS) capaces de realizar tareas con autonomía generalmente cuentan con mayores capacidades de carga y procesamiento computacional, así como sensores capaces de percibir grandes volúmenes de datos en un tiempo reducido. Estos vehículos aéreos no tripulados (VANTS) se centran en realizar tareas sencillas y estáticas en áreas abiertas con rutas predeterminadas, o bien, en contextos de operación por control remoto por un usuario. Sin embargo en donde los espacios son estrechos, se optan por el uso de VANTS reducidos, comúnmente llamados Micro-vehículos aéreos no tripulados (MAVs).

La exploración de áreas desconocidas, a través de la sinergia de sistemas con múltiples vehículos aéreos no tripulados (Multi-VANTS) promete ser una solución innovadora. Al comprender y perfeccionar una estrategia para la coordinación de múltiples vehículos aéreos no tripulados (Multi-VANTS) en tareas de exploración, esperamos descubrir nuevas posibilidades, replicar o romper los límites existentes, y en última instancia, avanzar en los campos de la robótica y la exploración.

\section{Antecedentes y motivación} 

\subsection*{Antecedentes}

La robótica móvil es una rama de la robótica que se enfoca en el diseño, construcción, programación y operación de robots capaces de moverse de manera autónoma o semi-autónoma en entornos diversos.

Uno de los primeros hitos importantes en la robótica móvil fué el desarrollo del robot shakey en la década de 1970, que se considera el primer robot móvil capaz de desplazarse evitando colisiones a su paso, sentando las bases en algoritmos de inteligencia artificial para búsquedas informadas surgiendo el algoritmo A*. 

Durante las décadas siguientes, se produjeron avances significativos en la miniaturización de componentes electrónicos, sensores y actuadores, lo que permitió la creación de robots móviles más pequeños y versátiles. A finales del siglo XX, los robots móviles comenzaron a ser utilizados en una variedad de aplicaciones, como la exploración espacial, la agricultura, la vigilancia y la logística.

En paralelo, los avances en inteligencia artificial, visión por computadora, planificación de trayectorias y sistemas de control contribuyeron al desarrollo de robots móviles más autónomos y adaptables. El surgimiento de algoritmos de aprendizaje automático y técnicas de percepción avanzada ha permitido a los robots móviles interactuar de manera más efectiva con su entorno y tomar decisiones en tiempo real.

En la actualidad, la robótica móvil está experimentando un rápido crecimiento gracias a la convergencia de diversas tecnologías, como la computación en la nube y la robótica colaborativa. Se están desarrollando robots móviles cada vez más sofisticados y capaces de realizar una amplia gama de tareas en entornos dinámicos y no estructurados. Además, se espera que la robótica móvil desempeñe un papel crucial en aplicaciones futuras, como la asistencia en el hogar, la atención médica, la exploración submarina y la entrega de paquetes.

Un sistema autónomo de un vehículo aéreo no tripulado, consta de cuatro algoritmos:
\begin{itemize}\setlength{\itemsep}{-1mm}
\item Generación de una representación del medio ambiente
\item Planificación de trayectorias
\item Evasión de obstáculos
\item Comunicación
\end{itemize}

La computadora embebida para un sistema de navegación usado en vehículos aéreos no tripulados de menor tamaño, son de bajo rendimiento. Pero, su necesidad de autonomía sigue siendo la misma que un VANT de mayor tamaño. Es por ello que es necesario equiparlos con algoritmos de baja complejidad computacional.

La necesidad de coordinación entre múltiples VANTS en tareas de exploración, surge debido a las limitaciones individuales en cuanto a la extensión de terreno que pueden cubrir y su desempeño. La exploración de áreas extensas o peligrosas a menudo exige un enfoque colaborativo, donde los VANTS trabajen juntos para optimizar la asignación de recursos y mejorar la recopilación y el análisis de datos.

Sin embargo, el camino hacia una coordinación entre múltiples vehículos aéreos no tripulados, presenta diversos desafíos. Las complejidades de gestionar un grupo de vehículos aéreos no tripulados, navegar en entornos dinámicos y distribuir tareas de forma inteligente son sólo algunas de las cuestiones que exigen nuestra atención. %El objetivo del documento es explorar y proponer una estrategia para mejorar la coordinación de múltiples VANTS en el contexto de las tareas de exploración.
% usar --> para resaltar cosas \textbf{\emph{}}

\subsection*{Motivación}

El potencial del uso de los vehículos aéreos no tripulados en tareas de búsqueda y rescate, inspección, mapeo, vigilancia, entre otras, es de gran interés a explorar, debido a las habilidades de vuelo que presentan en favor de la realización de estas tareas, y en especial situaciones que podrían poner en riesgo a personas.

Enviar personal de rescate dentro de un edificio parcialmente colapsado en busca de sobrevivientes, es poner a más personas en un gran riesgo, pues no se sabe qué es lo que les espera en el interior del edificio; esto limita la capacidad de tomar buenas decisiones acerca de si es seguro seguir cierto camino.

Operar en ambientes como éste u otros similares requieren de robots con capacidades de navegar sobre terrenos difíciles y evadir obstáculos de forma segura para obtener información del entorno que pueda ser útil al personal de rescate.

Una posible solución consiste en un robot móvil aéreo (VANT) capaz de desplazarse sobre terrenos difíciles y navegar en espacios cerrados de manera segura, que además recabe información del entorno. Esto implica realizar tareas de reconocimiento del ambiente, evasión de obstáculos y seguimiento de trayectorias.

\section{Planteamiento del problema} 

Dado un volumen de interés desconocido en un espacio cerrado que se desea explorar denotado como $\mathcal{W}$, tal que $\mathcal{W} \subset \mathbb{R}^{3}$.
%el volumen se discretiza en un mapa de ocupación M consistente en voxeles cúbicos $m \exists M$ con una longitud de arista r.
\begin{itemize}\setlength{\itemsep}{-1mm}
  \item El volumen se discretiza usando unidades cúbicas tridimencionales (voxel) tomando valores $v_{libre}$, $v_{ocup}$, $v_{desc}$.
    %La representación del volumen a explorar se obtiene dividiendo el volumen de interés en unidades cúbicas tridimencionales (voxel) que puede tomar los valores de libre $v_{libre}$, ocupado $v_{ocup}$ y desconocido $v_{desc}$ con lecturas a partir de los valores de una cámara RGB-D basada en un modelo de ocupación probabilístico.\\
  \item Un conjunto de VANTS con una cámara RGB-D embarcadas denotado como $\mathcal{V} = \{\mathcal{V}_{1},\mathcal{V}_{2},\mathcal{V}_{3},...,\mathcal{V}_{n}\}$, comenzando cada uno en un estado inicial conocido $q = \{q_{1},q_{2},q_{3},...,q_{n}\}$, y terminando en una configuración que maximice la construcción de un mapa.
\end{itemize}

Coordinar el conjunto de VANTs para reducir el tiempo total de exploración.
  
  
\section{Hipótesis y preguntas de investigación}

\subsection*{Hipótesis}
\emph{``Una estrategia que coordine y asigne tareas de exploración para múltiples VANTS de manera descentralizada, en combinación con una arquitectura de software diseñada para resolver problemas de localización, gestión de mapas y planificación de rutas, mejorará la eficiencia y cobertura de la exploración en interiores de un entorno desconocido''}.


\subsection*{Preguntas de investigación}

\begin{enumerate}\setlength{\itemsep}{-1mm}
\item ¿Qué características de la dinámica del VANT son cruciales para lograr trayectorias suaves y continuas?
\item ¿Podría un planificador de trayectorias que aproveche las regiones libres de obstáculos acelerar los desplazamientos de los VANTs y, consecuentemente, reducir los tiempos de exploración?
\item ¿Qué mecanismos de coordinación existen dentro de la literatura que podrían ayudar en resolver el problema de exploración multi-VANT?
\end{enumerate}

\section{Objetivos}

\subsection*{Objetivo general}

%El principal objetivo del trabajo de tesis se define a continuación:
\emph{``Desarrollar una estrategia de exploración descentralizada que permita resolver los problemas de coordinación para múltiples VANTS en ambientes desconocidos.''}.

\subsection*{Objetivos específicos}

Para lograr nuestro objetivo principal, se consideran los siguientes objetivos especificos a cubrir:

\begin{itemize}\setlength{\itemsep}{-1mm}
\item Desarrollar una arquitectura de software que resuelva los problemas de autonomía para un VANT (localización, manejo de mapas y planificación de trayectorias).
\item Implementar un mecanismo de coordinación descentralizado que asigne tareas de exploración.
\item Realizar pruebas y simulaciones de la solución propuesta en diversos entornos, analizando la relación tiempo de exploración y cobertura del área de interés.
\end{itemize}

\newpage

\section{Solución propuesta}

Para resolver el problema de exploración multi-VANT con un enfoque descentralizado considerando restricciones en el rango de la comunicación. Se cuenta como antecedente el trabajo doctoral de \citeauthor{CINVESTAM2013} que propone un algoritmo basado en un proceso de ofertas de mercado, en el cual cada robot calcula las ofertas de manera independiente, buscando alcanzar el mayor aporte posible al equipo en su conjunto. Cuando un robot alcanza su objetivo, el robot toma una decisión para sí mismo, involucrando a cada uno de los miembros del equipo así como el rango de comunicación, bajo un esquema descentralizado y sin la necesidad de un módulo central.

Para validar la propuesta de exploración coordinada se necesita primero resolver la autonomía de un vehículo aéreo no tripulado diseñando una arquitectura que incluya la coordinación multi-robot propuesta en \cite{CINVESTAM2013}.

\begin{enumerate}\setlength{\itemsep}{-1mm}
\item Conocer los fundamentos que nos aproximen a realizar la tarea de exploración autónoma con múltiples VANTS.
\item Profundizar en la comprensión de los comandos de control y la generación de odometría para un VANT tipo cuadricóptero.
\item Obtener y procesar la información proveniente de un sensor de tipo RGB-D dentro del sistema operativo ROS.
\item Integrar un planificador de trayectoria reactivo que, combinado con la percepción recibida por la cámara RGB-D, nos permita evadir obstáculos en su paso y construir una representación tridimensional del medio ambiente.
\item Elaborar la exploración con un VANT de tipo cuadrotor.
\item Implementar la estrategia de exploración bajo los conceptos de cohesión, exploración y explotación.
\end{enumerate}



%\section{Resumen}
%\lipsum[2-4]

%The rest of this document is organized as follows: Chapter~\ref{chapter2} presents basic concepts and background in the field of optimization. Then, Chapter~\ref{ch:PSODE} introduces particle swarm optimization and differential evolution which are the two metaheuristics on which this thesis work focusses. In order to introduce these two metaheuristics, EAs is general are also described in this chapter. Afterwards, Chapter~\ref{experiments} presents a series of experiments that were developed and that allowed to obtain further information about the search performed by PSO and DE in multi-objective optimization. This knowledge was used to develop two new MOEAs which are presented and evaluted in Chapter~\ref{proposals}. Finally, Chapter~\ref{conclusion} concludes this thesis work.


  \chapter{Estado del Arte}

En este capítulo, se presenta un panorama de las distintas propuestas en exploración que involucran el uso de múltiples vehículos aéreos no tripulados (VANTs), evaluando tanto sus ventajas como sus limitaciones. Nuestra atención se centra especialmente en las propuestas relacionadas con la exploración multi-robot. % y la exploración multi-robot en entornos con incertidumbre en la posición.

%Este capítulo ofrece un panorama general de las principales propuestas en exploración empleando múltiples vehículos aéreos no tripulados (VANTS), discutiendo sus fortalezas y sus debilidades. Nos enfococamos principalmente en las propuestas de exploración multi-robot, SLAM multi-robot y exploración multi-robot con incertidumbre en la posición.

La exploración multi-robot se enfoca principalmente en guiar a cada robot de forma eficiente con la finalidad de obtener una representación del medio ambiente asumiendo, que la localización es perfecta y conocida en todo momento. El problema de localización y mapeo simultáneo (SLAM) integra la localización y el aprendizaje de mapas como el problema de construir un mapa mientras se trata de localizar al robot dentro del mismo.

Para lograr una exploración efectiva, los robots deben coordinar sus acciones, compartir información sobre el entorno y planificar sus movimientos de manera inteligente, teniendo en cuenta la distribución de tareas, la comunicación entre robots y la detección de obstáculos.%Finalmente, el problema de exploración multi-robot con incertidumbre en la posición representa el problema de generar un mapa mientras se estima la localización y se guía al robot hacia objetivos definidos con la finalidad de generar el mapa de forma eficiente.

%A continuación se analizan las principales propuestas para la generación de mapas con múltiples robots, que comparten rasgos comunes con el enfoque desarrollado en este trabajo de tesis.

\section{Introducción}

Las aplicaciones de la rob\'{o}tica industrial se han centrado en realizar tareas simples y repetitivas. La necesidad de robots con capacidad de identificar cambios en su entorno y reaccionar sin la intervenci\'{o}n humana, da origen a los robots inteligentes. Aunado a ello, si deseamos que el robot se mueva libremente, los cambios en su entorno pueden aumentar r\'{a}pidamente y complicar el problema de desarrollar un robot que muestre un comportamiento inteligente.

El despliegue r\'{a}pido de robots en situaciones de riesgo, b\'{u}squeda y rescate ha sido un \'{a}rea ampliamente estudiada en la rob\'{o}tica m\'{o}vil. Recientemente, concursos como el DARPA Subterranean (SubT) Challenge, buscan acelerar la investigación y desarrollo de tecnología en escenarios complejos subterráneos, donde factores como exploraci\'{o}n, planeaci\'{o}n y coordinaci\'{o}n son clave para lograr los objetivos del Reto [\citenum{DARPA2022}], la creación de una estategia de exploración es pieza clave para el cumplimiento del reto.

%%%%%%%%%%%%%%%%%%%%%%%
%%%%%%%%%%%%%%%%%%%%%%%
%%%%%%%%%%%%%%%%%%%%%%%
%%%%%%%%%%%%%%%%%%%%%%%
%%%%%%%%%%%%%%%%%%%%%%%

La exploraci\'{o}n de un ambiente desconocido empleando multi-VANT es un \'{a}rea relativamente nueva y con mucho crecimiento en los \'{u}ltimos a\~{n}os. Se han abordado una variedad de temas para lograr la exploraci\'{o}n aut\'{o}noma, desde la planificaci\'{o}n de rutas para m\'{u}ltiples robots terrestres en tareas de exploraci\'{o}n [\citenum{Sharma2016}], estrategias para la coordinaci\'{o}n y protocolos de comunicaci\'{o}n [\citenum{Gielis2022}]. Diversos estudios de aplicaciones multi-VANT se han realizado para resolver tareas como el monitoreo ambiental [\citenum{SurveyCollab2019}], la agricultura de precisi\'{o}n [\citenum{10011226}] y operaciones de b\'{u}squeda y rescate [\citenum{SHAKHATREH2019}]. A pesar de que los VANTS comerciales modernos cuenten con avanzadas integraciones en los problemas de navegación. Siguen sin incorporar una solución completa de autonomía.

La direcci\'{o}n en que apunta el estado del arte actualmente se puede atribuir a los avances en tecnolog\'{i}a en la \'{u}ltima d\'{e}cada que nos han permitido dotar de autonomía a VANTS de menor tamaño. Investigadores de diversas \'{a}reas, que incluyen las ciencias computacionales y la ingeniería, han contribuido al progreso de este campo.

Sin embargo, la necesidad de una arquitectura de software descentralizada que engloble los retos de exploración, junto con un mecanismo de coordinación para múltiples VANTS, surge como una área de investigación prometedora que garantice el funcionamiento correcto de los componentes necesarios para realizar tareas de exploración coordinada multi-VANT.

En esta revisión del estado del arte se citan trabajos relacionados a la coordinación multi-VANT para tareas de exploración, que sirven de guía para la realización de este trabajo.

%\section{Conceptos básicos} \label{}

La exploración es una tarea fundamental en robótica, el objetivo es construir un mapa de un espacio desconocido con ayuda de un algoritmo de planificación de trayectoria que guiará al robot donde moverse. Una representación del ambiente (mapa) es necesaria para tomar la decisión de la mejor ruta hacia un mayor conocimiento del espacio que lo rodea.

%Para poder construir una buena representación del ambiente, debemos comprender como realizar una exploración. En una planificación para una exploración, después de una observación, el robot debe decidir donde moverse.

%Buscando acelerar la exploración, el robot debe percibir el ambiente desconocido lo más rápido posible. En la literatura existen dos principales técnicas:

\section{Trabajos relacionados} \label{}

El presente texto se centra en el estado del arte de las estrategias para la exploración coordinada con múltiples Vehículos Aéreos No Tripulados (VANT), abordando diversos temas fundamentales en este campo de investigación. La exploración, la representación del medio ambiente en 3D, la planificación de trayectorias, la generación de trayectorias y la coordinación multirobot son aspectos cruciales que influyen en el diseño y la implementación de sistemas autónomos para la exploración coordinada. En este contexto, se analizan las tendencias actuales, los enfoques más relevantes y los desafíos existentes.


\subsection{Exploración}

Para poder construir una buena representación del ambiente, debemos comprender como realizar una exploración. En una planificación para una exploración, después de una observación, el robot debe decidir donde moverse.

Buscando acelerar la exploración, el robot debe percibir el ambiente desconocido lo más rápido posible. En la literatura existen dos principales técnicas:

\begin{itemize}\setlength{\itemsep}{-1mm}
\item Next Best View (NBV) planning - El planificador elige la siguiente posición en base a la percepción y una función de utilidad que maximice la ganancia de información. 
\item Basado en fronteras - La estrategia es simple y eficiente al asignar rutas al robot en base a la frontera más cercana.
\end{itemize}

%Las bases para la exploraci\'{o}n aut\'{o}noma e innovaciones son heredadas de algoritmos ya empleados en la rob\'{o}tica m\'{o}vil (tabla \ref{pathsummary}). Uno de los primeros trabajos en la exploraci\'{o}n con robots, es la propuesta de fronteras por \citeauthor{YAMAUCHI1997}[\citenum{YAMAUCHI1997}], donde define como una frontera a la l\'{i}nea entre las zonas exploradas y las no exploradas de un \'{a}rea de inter\'{e}s. Durante la navegaci\'{o}n la informaci\'{o}n percibida por el robot crece, moviendo las fronteras hasta que no existan m\'{a}s fronteras.

%En el trabajo de \citeauthor{Faria2019}[\citenum{Faria2019}], se combina la estrategia basada en fronteras con t\'{e}cnicas de planificaci\'{o}n de trayectorias Lazy Theta* en un VANT.\\  

%\cite{FRAUNDORFER2012}[\citenum{FRAUNDORFER2012}] hacen uso de una exploraci\'{o}n con fronteras a partir de una navegaci\'{o}n aut\'{o}noma aplicando un algoritmo tipo bug para el seguimiento de una pared, empleando campos de potencial artificial para una planificaci\'{o}n local en un mapa de ocupaci\'{o}n tipo grid. Estos trabajos demostraron la navegaci\'{o}n aut\'{o}noma de veh\'{i}culos a\'{e}reos no tripulados y que estos pueden seguir puntos de referencia en el mapa, evitar obst\'{a}culos y llevar a cabo tareas de exploraci\'{o}n en entornos complejos.\\

Los trabajos de \citeauthor{CIESLEWSKI2017}[\citenum{CIESLEWSKI2017}], \citeauthor{BARTOLOMEI2023}[\citenum{BARTOLOMEI2023}] se enfocan en la exploración haciendo uso de estrategias basadas en fronteras.

Mientras que los trabajos de \citeauthor{PAPACHRISTOS2017}[\citenum{PAPACHRISTOS2017}],\citeauthor{SELIN2019}[\citenum{SELIN2019}], \citeauthor{RACER2022}[\citenum{RACER2022}], que también trabajan en el problema de exploración autónoma. Utilizan la estrategia de Next Best View Planner guiando al robot donde puedan aportar más información en la construcción del medio ambiente (mapa). Por el contrario \citeauthor{BUG2019}[\citenum{BUG2019}], al tener un comportamiento tipo bug hace uso de su propuesta con Swarm Gradient Bug Algorithm.

\subsection{Representación medio ambiente 3D}

%Los primeros trabajos de VANT autónomos se encuentran en las aportaciones de \citeauthor{SHEN2011}[\citenum{SHEN2011}], que hacen uso de un VANT con la propuesta de dos planificadores de trayectorias con un control proporcional con retroalimentaci\'{o}n y basados en RRT*, para conseguir una representaci\'{o}n del mundo en 2D empleando un sensor tipo LiDAR. Por otra parte, los trabajos de \citeauthor{GRZONKA2012}[\citenum{GRZONKA2012}] tambi\'{e}n hacen una representaci\'{o}n del entorno en 2D, haciendo uso de algoritmos que trabajan en mapas densos tipo grid, y del algoritmo D* lite para la planificaci\'{o}n de trayectorias.\\

Con la llegada de las primeras c\'{a}maras capaces de obtener valores de profundidad (RGB-D), y con mayores capacidades de almacenamiento en menos espacio, nos permiten tratar el medio ambiente a través de representaciones tridimensionales. Podemos citar por ejemplo la propuesta de estructura de datos basada en grafos octrees por \citeauthor{DONALD1982}[\citenum{DONALD1982}] con una baja complejidad en el orden logarítmico, por consiguiente en el año 2013 se introdujo un nuevo concepto para la representaci\'{o}n de mapas 3D basados en esos principios, haciendo que la representaci\'{o}n de entornos 3D se realice de manera eficiente para aplicaciones en rob\'{o}tica donde se necesitan algoritmos r\'{a}pidos. Los trabajos de \citeauthor{ARMIN2013}[\citenum{ARMIN2013}] introducen los Octomaps, que se utilizan para representar mapas tridimensionales como subdivisiones marcadas como ocupadas, desocupadas y desconocidas, para aplicaciones de navegaci\'{o}n.

En recientes trabajos \citeauthor{min2020accelerating}[\citenum{min2020accelerating}] proponen dar soluci\'{o}n a los cuellos de botella que se presentan en el Octomap buscando acelerar los tiempos de cómputo en la construcci\'{o}n de mapas a partir de la implementaci\'{o}n de Aceleradores Gr\'{a}ficos GPU.

Los trabajos de \citeauthor{CIESLEWSKI2017}[\citenum{CIESLEWSKI2017}] hacen uso de la representaci\'{o}n del entorno por medio de una rejilla tridimensional (voxel grids), la versión tridimensional de la rejilla de ocupación empleada en robótica móvil terrestre, para planificar trayectorias de exploraci\'{o}n. \citeauthor{USENKO2017}[\citenum{USENKO2017}] proponen el uso del mapa centrando al robot en un círculo tridimensional de tama\~{n}o fijo, por su parte \citeauthor{MOHTA2017}[\citenum{MOHTA2017}] hacen uso de un mapa h\'{i}brido formado con la combinaci\'{o}n de un mapa local 3D con un mapa global en 2D. \citeauthor{FLORENCE2018}[\citenum{FLORENCE2018}] propone un framework para el manejo de datos para mapas 3D a partir de la información de una cámara de profundidad (RGB-D). La propuesta hace uso de una estructura con cuadrículas de ocupación y datos de profundidad.

Diversos trabajos en los que se incluyen a \citeauthor{GAO2018}[\citenum{GAO2018}], \citeauthor{LIN2017}[\citenum{LIN2017}], \cite{OLEYNIKOVA2018}[\citenum{OLEYNIKOVA2018}], añaden una estructura adicional a su representación del medio ambiente, usando distancias que permiten la evasión de obstáculos de forma segura. Sin embargo estas soluciones pueden ser costosas requiriendo un mayor procesamiento de computo.

\citeauthor{COLLINS2019}[\citenum{COLLINS2019}] usan una representaci\'{o}n local del mapa con ayuda de una estructura de datos KD-Tree. Usa un mapa representado en voxels, mientras que un grafo topol\'{o}gico representa todo el entorno explorado. Por otra parte el trabajo de \citeauthor{BUG2019}[\citenum{BUG2019}] no hace uso de un mapa. Al generar movimientos reactivos tipo bug, logra generar una navegación autónoma con odometría visual.

Los trabajos de \citeauthor{PAPACHRISTOS2017}[\citenum{PAPACHRISTOS2017}], \citeauthor{SELIN2019}[\citenum{SELIN2019}], \citeauthor{CINVES2021}[\citenum{CINVES2021}], \citeauthor{RACER2022}[\citenum{RACER2022}], \citeauthor{BARTOLOMEI2023}[\citenum{BARTOLOMEI2023}], optan por el uso de OctoMaps. Haciendo de la representación de ambientes en 3D con octomaps un estandard en la robótica moderna.

\subsection{Planificación de trayectorias}

Uno de los desaf\'{i}os clave en la colaboraci\'{o}n de m\'{u}ltiples VANTS es la planificaci\'{o}n de rutas. Se han desarrollado diversos algoritmos para optimizar la planificaci\'{o}n de rutas dentro de la rob\'{o}tica m\'{o}vil, minimizando los riesgos de colisi\'{o}n y mejorando la eficiencia en sus misiones. Estos algoritmos tienen en cuenta varios factores como las restricciones del robot y la ubicación del objetivo, para generar trayectorias seguras.

El objetivo principal de los algoritmos de planificación de trayectorias, es el de guiar al robot desde el punto de inicio al punto destino. Los trabajos por \citeauthor{Lumelsky1987}[\citenum{Lumelsky1987}], dieron respuesta a problemáticas de navegaci\'{o}n eficiente, que no requieren de una representación del medio ambiente y emplean, por lo tanto, pocos recursos computacionales y de memoria (algoritmos tipo bug).

%Siendo el v\'{e}rtice  la posici\'{o}n del robot y las aristas un camino donde encontramos los trabajos de

Matem\'{a}ticamente, el problema de planificación de trayectorias es resuelto a través del modelado del medio ambiente utilizando grafos, siendo un grafo una representaci\'{o}n matem\'{a}tica de v\'{e}rtices y aristas. \citeauthor{4082128}[\citenum{4082128}], al mejorar el algoritmo de Dijkstra para el robot Shakey, logr\'{o} navegar en una habitaci\'{o}n que conten\'{i}a obst\'{a}culos fijos. El objetivo principal del algoritmo A* es la eficiencia en la planificaci\'{o}n de rutas al incorporar una heurística. A su vez, el algoritmo D*, propuesto por \citeauthor{351061}[\citenum{351061}], ha demostrado operar de manera eficiente ante obst\'{a}culos din\'{a}micos; en comparaci\'{o}n con el algoritmo A* que vuelve a ejecutarse al encontrarse con un obst\'{a}culo no previsto inicialmente, el algoritmo D* usa la informaci\'{o}n previa para buscar una nueva ruta hacia el objetivo.

%\begin{table}[h!]
%\centering
%\scalebox{0.65}{
%\begin{tabular}[t]{l|c|c|c|p{9cm}}
%\hline\hline
%M\'{e}todo&Completez&\'{O}ptimo&Escalable&Notas\\
%\hline\hline
%Grafo de visibilidad [\citenum{8798322}]& \ding{51} & \ding{51} & ding{55} & \begin{itemize}[left=0pt,topsep=0pt]
%  \setlength\itemsep{0.1em}
%\item Mucho espacio libre
%\item Mala escalabilidad
%\item El robot pasa cerca de obstáculos
%\end{itemize}\nointerlineskip\\
%\hline
%Diagramas de Voronoi [\citenum{7978644}] & \ding{51} & \ding{55} &\ding{55} & \begin{itemize}[left=0pt,topsep=0pt]
%\item Espacio libre m\'{a}ximo
%\item Rutas conservadoras
%\item Mala escalabilidad
%\end{itemize}\nointerlineskip\\
%\hline
%Campos de potencial artificial [\citenum{9836159}]& \ding{51} & \dng{55} & Depende del ambiente & \begin{itemize}[left=0pt,topsep=0p]
%\item F\'{a}cil de implementar
%\item Susceptible a m\'{i}nimos locales
%\end{itemize}\nointerlineskip\\
%\hline
%Dijkstra/A* [\citenum{Wang2022}]& \ding{51} & Grafo & \ding{55} & begin{itemize}[left=0pt,topsep=0pt]
%\item M\'{a}s r\'{a}pido que la b\'{u}squeda desinformada
%\item A* usa una funci\'{o}n heur\'{i}stica para impulsar la b\'{usqueda de manera eficiente
%\item Mala escalabilidad
%\end{itemize}\nointerlineskip\\
%\hline
%PRM [\citenum{6491122}]& \ding{51} & Grafo & \ding{51} & \begin{itmize}[left=0pt,topsep=0pt]
%\item Eficiente para problemas con consultas m\'{u}ltiples
%\item Completez probabil\'{i}stica
%\item  Camino irregular
%\end{itemize}\nointerlineskip\\
%\hline
%RRT [\citenum{9018916}]& \ding{51} & \ding{55} & \ding{51} & \begi{itemize}[left=0pt,topsep=0pt]
%\item Eficiente para problemas de consulta \'{u}nica
%\item Completez probabil\'{i}stica
%\item Camino irregular
%\end{itemize}\nointerlineskip\\
%\hline
%\end{tabular}
%}
%\setlength\tabcolsep{0pt}
%\caption{\label{pathsummary}M\'{e}todos para planificaci\'{o}n de rayectorias usados en rob\'{o}tica m\'{o}vil}
%\end{table}%

Por otra parte, el algoritmo RRT (Rapidly Exploring Random Trees), propuesto por \citeauthor{LaValle1998RapidlyexploringRT}[\citenum{LaValle1998RapidlyexploringRT}], es ampliamente usado para la planificaci\'{o}n de rutas en robots modernos. El algoritmo construye de forma incremental una estructura de \'{a}rbol mediante un muestreo aleatorio en el espacio de configuraciones, uniendo aleatoriamente nuevas posiciones al \'{a}rbol existente hasta alcanzar la posición final. Las modificaciones realizadas al algoritmo RRT por \citeauthor{Karaman2011}[\citenum{Karaman2011}], incorporando una heur\'{i}stica de costo por recorrer, permite encontrar rutas casi \'{o}ptimas de manera eficiente.

En recientes trabajos de \citeauthor{yang2022far}[\citenum{yang2022far}], muestran la capacidad de implementaci\'{o}n de algoritmos clásicos de planificación de trayectorias, como los grafos de visibilidad, para tareas en entornos conocidos y no conocidos, utilizando una representaci\'{o}n del ambiente a base de polígonos, logrando un r\'{a}pido planificador que tambi\'{e}n resuelve obst\'{a}culos nuevos en el ambiente, logrando resultados comparables a las estrategias más recientes como D* e inclusive RRT*.

%Trabajos como el de \citeauthor{CERBERUS2022}[\citenum{CERBERUS2022}] en el que han logrado optimizar problemas de alta dimensionalidad como el control de navegaci\'{o}n para un robot con cuatro patas, haciendo uso de aprendizaje por refuerzo y con ayuda de simulaciones, logrando obtener un esquema de control que le permiten al robot resolver el problema de navegaci\'{o}n. Sin embargo, al momento de probar el esquema en un robot real, el robot no pudo efectuar un paso correcto. Este problema se debe a la distancia que existe entre la simulaci\'{o}n y la realidad, en particular al no considerar las incertidumbres en las lecturas de los sensores. \\

%Las simulaciones permiten demostrar el correcto funcionamiento de los esquemas de control. A través de simulaciones híbridas e introduciendo ruidos estocásticos en las simulaciones, es posible lograr resultados muy prometedores, como en el caso de éxito en el DARPA Subterranean Challenge [\citenum{DARPA2022}], que utiliza una exploraci\'{o}n basada en grafos y un mapa de ocupaci\'{o}n (Octomap) para simular el entorno tridimensional.\\

Un enfoque muy recurrente para abordar el problema de planificación de trayectorias, es el uso de las metaheur\'{i}sticas bio-inspiradas (Algoritmos Genéticos (GA), Particle Swarm Optimization (PSO), Ant Colony Optimization (ACO), Firefly Algorithm (FA)). Estas estrategias se inspiran en sistemas y procesos biol\'{o}gicos para resolver problemas complejos de optimizaci\'{o}n.
%Existen varios tipos de metaheur\'{i}sticas bio-inspiradas:

%\begin{enumerate}
%  \item \textbf{Algoritmos Gen\'{e}ticos (GA)}. Propuestos por J. Holland, se basan en los principios de selecci\'{o}n natural, usando operadores como la cruza, mutaci\'{o}n y selecci\'{o}n. Manteniendo una poblaci\'{o}n de las posibles soluciones iterando para encontrar la soluci\'{o}n cercana a la soluci\'{o}n \'{o}ptima.
%  \item \textbf{Particle Swarm Optimization (PSO)}. Propuestos por Eberhart y Kennedy, inspirado en el comportamiento de parvadas de p\'{a}jaros y cardumen de peces, el algoritmo involucra una poblaci\'{o}n de part\'{i}culas que se mueven en un espacio de b\'{u}squeda. Cada part\'{i}cula ajusta su posici\'{o}n seg\'{u}n su propia soluci\'{o}n y la soluci\'{o}n de toda la poblaci\'{o}n.
%  \item \textbf{Ant Colony Optimization (ACO)}. Propuesto por M. Dorigo, inspirado en el comportamiento de b\'{u}squeda de alimento de las hormigas, imita la comunicaci\'{o}n y toma de decisiones colectiva de las hormigas, puede ser usado para encontrar caminos dentro de un grafo. 
%  \item \textbf{Firefly Algorithm (FA)}. Propuesto por X. Yang, sigue el modelo de los patrones intermitentes de las luci\'{e}rnagas, el algoritmo emula el comportamiento de atracci\'{o}n y repulsi\'{o}n de las luci\'{e}rnagas.
%  \end{enumerate}

  Las metaheur\'{i}sticas han demostrado ser efectivas para resolver una amplia gama de problemas de optimizaci\'{o}n; sin embargo, su adopci\'{o}n en el campo de la rob\'{o}tica móvil se ve limitada por las restricciones de ejecución en tiempo real. La rob\'{o}tica a menudo implica la toma de decisiones en tiempo real, donde los robots deben responder r\'{a}pidamente a entornos cambiantes. Las metaheur\'{i}sticas suelen requerir extensos recursos computacionales y temporales para converger en una soluci\'{o}n \'{o}ptima, lo que puede no ser factible en aplicaciones de rob\'{o}tica en tiempo real, particularmente en vehículos aéreos con limitado poder de cómputo. El control y la planificaci\'{o}n en tiempo real en rob\'{o}tica a menudo requieren algoritmos de baja complejidad computacional, como la planificaci\'{o}n cl\'{a}sica o los enfoques de control reactivo.
  
  %\begin{itemize}
  %\item \textbf{Complejidad y restricciones en tiempo real:} la rob\'{o}tica a menudo implica la toma de decisiones en tiempo real, donde los robots deben responder r\'{a}pidamente a entornos cambiantes. Las metaheur\'{i}sticas suelen requerir extensos recursos computacionales y temporales para converger en una soluci\'{o}n \'{o}ptima, lo que puede no ser factible en aplicaciones de rob\'{o}tica en tiempo real, particularmente en vehículos aéreos con limitado poder de cómputo. El control y la planificaci\'{o}n en tiempo real en rob\'{o}tica a menudo requieren algoritmos de baja complejidad computacional, como la planificaci\'{o}n cl\'{a}sica o los enfoques de control reactivo.
    
  %\item \textbf{Soluciones rápidas:} en aplicaciones de rob\'{o}tica, especialmente las que involucran tareas cr\'{i}ticas para la seguridad o control preciso, se prefieren las soluciones deterministas y predecibles a las soluciones estoc\'{a}sticas que ofrecen las metaheur\'{i}sticas. Las metaheur\'{i}sticas brindan soluciones aproximadas con diversos grados de optimizaci\'{o}n, que pueden no ser adecuadas para tareas que requieren un control preciso o garant\'{i}as de seguridad.

%  \item \textbf{Optimizaci\'{o}n basada en modelos:} muchos problemas de rob\'{o}tica se pueden resolver de manera efectiva utilizando t\'{e}cnicas de optimizaci\'{o}n basadas en modelos. Con modelos din\'{a}micos conocidos y restricciones ambientales, los m\'{e}todos basados en modelos, como el control \'{o}ptimo o la optimizaci\'{o}n de la trayectoria, pueden proporcionar soluciones anal\'{i}ticas o num\'{e}ricas con un rendimiento garantizado. Estos enfoques pueden explotar la estructura del problema y las restricciones espec\'{i}ficas, lo que lleva a soluciones eficientes y confiables en comparaci\'{o}n con las metaheur\'{i}sticas de prop\'{o}sito general.

  %\item \textbf{Algoritmos de tareas específicas:} la rob\'{o}tica a menudo implica tareas y dominios espec\'{i}ficos que se han estudiado ampliamente, lo que da como resultado algoritmos espec\'{i}ficos de tareas adaptados a esos dominios. Estos enfoques personalizados a menudo son m\'{a}s eficientes y efectivos para resolver los problemas espec\'{i}ficos abordados en rob\'{o}tica, lo que hace que las metaheur\'{i}sticas de prop\'{o}sito general sean menos atractivas.
    
  %\item \textbf{Limitaciones de hardware y energ\'{i}a}: los sistemas de rob\'{o}tica suelen tener recursos de hardware limitados y, a menudo, est\'{a}n limitados por el consumo de energ\'{i}a. Las metaheur\'{i}sticas, que a menudo requieren una mayor cantidad de recursos o extensos tiempos de ejecución para alcanzar la convergencia, pueden no ser adecuadas para plataformas rob\'{o}ticas con recursos limitados.
  %\end{itemize}

Sin embargo, es importante tener en cuenta que ciertamente hay \'{a}reas dentro de la rob\'{o}tica donde las metaheur\'{i}sticas se han aplicado con \'{e}xito, como la planificaci\'{o}n de rutas de robots en entornos complejos, la rob\'{o}tica de enjambres o la asignaci\'{o}n de tareas en sistemas de m\'{u}ltiples robots. Los enfoques h\'{i}bridos que combinan metaheur\'{i}sticas con optimizaci\'{o}n basada en modelos o algoritmos espec\'{i}ficos de tareas pueden aprovechar las fortalezas de ambos y proporcionar soluciones efectivas para aplicaciones en la rob\'{o}tica.

En lugar de planificar trayectorias, los autores en \citeauthor{CIESLEWSKI2017}[\citenum{CIESLEWSKI2017}], emplean un enfoque reactivo que genera comandos de velocidad instantáneos hacia las fronteras descubiertas. \citeauthor{PAPACHRISTOS2017}[\citenum{PAPACHRISTOS2017}], \citeauthor{CINVES2021}[\citenum{CINVES2021}] construye un árbol de exploración rápido y aleatorio (RRT) con un costo relacionado al número de nuevos voxels para identificar el próximo objetivo, y un segundo RRT para trazar una ruta hacia el punto de vista seleccionado minimizando la incertidumbre en la posición y puntos de referencia del robot. Por su parte, \citeauthor{SELIN2019}[\citenum{SELIN2019}] introduce nodos con un alto potencial de ganancia de información en un RRT para utilizarlos como objetivos de planificación después de que el agente ha explorado su entorno cercano.

\citeauthor{OLEYNIKOVA2018}[\citenum{OLEYNIKOVA2018}] que también se ocupa del problema de exploración, incorpora un objetivo adicional de alcanzar una meta para abordar de manera explícita el problema de quedarse atrapado en mínimos locales. Eligen la próxima meta al seleccionarla con cierta probabilidad desde el objetivo global.

\citeauthor{MOHTA2017}[\citenum{MOHTA2017}], \citeauthor{GAO2018}[\citenum{GAO2018}] y \citeauthor{LIN2017}[\citenum{LIN2017}] emplean la información generada en la etapa de planificación para establecer un problema de optimización que produzca trayectorias seguras y dinámicamente viables. Todos buscan generar un corredor seguro para restringir la optimización. \citeauthor{MOHTA2017}[\citenum{MOHTA2017}], \citeauthor{LIN2017}[\citenum{LIN2017}], y \citeauthor{FLORENCE2018}[\citenum{FLORENCE2018}] emplean un algoritmo A* para planificar una trayectoria para buscar una distancia mínima hacia la siguiente frontera.

\cite{BUG2019}[\citenum{BUG2019}] presenta una soluci\'{o}n de navegaci\'{o}n para enjambres de peque\~{n}os multi-VANTS que exploran entornos desconocidos sin se\~{n}al de GPS de forma centralizada. \'{E}ste trabajo propone el algoritmo Swarm Gradient Bug (SGBA), que maximiza la cobertura al hacer que los robots se muevan en diferentes direcciones lejos del punto de partida. Los robots navegan por el entorno y enfrentan obst\'{a}culos est\'{a}ticos sobre la marcha mediante la odometr\'{i}a visual y algoritmos tipo BUG para el seguimiento de paredes. Adem\'{a}s, se comunican entre s\'{i} para evitar colisiones y maximizar la eficiencia de la b\'{u}squeda. Para regresar al punto de partida, los robots realizan una b\'{u}squeda de gradiente hacia una se\~{n}al Bluetooth de baja potencia.

%Se estudiaron los aspectos colectivos de SGBA, demostrando que permite que un grupo de cuadric\'{o}pteros comerciales est\'{a}ndar de 33 gramos explore con \'{e}xito un entorno del mundo real. El potencial de aplicaci\'{o}n se ilustra mediante una misi\'{o}n de b\'{u}squeda y rescate de prueba en la que los robots capturaron im\'{a}genes para encontrar v\'{i}ctimas en un entorno de oficina. Los algoritmos desarrollados se generalizan a otros tipos de robots y sientan las bases para abordar misiones igualmente complejas con enjambres de robots en el futuro.\cite{BUG2019}.\\

Independientemente del enfoque utilizado para generar la ruta, un aspecto crítico para la navegación libre de colisiones en entornos desconocidos es imponer restricciones en el plan de movimiento para navegar dentro del campo de visión actual hacia el siguiente punto de referencia, utilizada por los métodos en \citeauthor{RACER2022}[\citenum{RACER2022}] y \citeauthor{BARTOLOMEI2023}[\citenum{BARTOLOMEI2023}]

%En el trabajo de \citeauthor{MOHTA2017}[\citenum{MOHTA2017}] usan un planificador A* en un grafo h\'{i}brido con la informaci\'{o}n 3D y 2D, formulan un problema de programaci\'{o}n cuadrática para la generaci\'{o}n de trayectorias agregando un t\'{e}rmino en la funci\'{o}n de costo sobre el error entre la trayectoria y los segmentos de l\'{i}nea del camino. La trayectoria se representa como un polinomio de s\'{e}ptimo orden, utilizando un perfil de velocidad trapezoidal. \\

%\cite{LIN2017}[\citenum{LIN2017}] hacen uso de un planificador global offline para generar rutas, en la navegaci\'{o}n usan un planificador local seleccionando las nuevas rutas y un algoritmo A* para buscar la distancia m\'{i}nima hacia esas nuevas rutas. Utilizan un polinomio por partes de octavo orden para la representaci\'{o}n de la trayectoria.\\

%\cite{PAPACHRISTOS2017}[\citenum{PAPACHRISTOS2017}] presentan algoritmos para la exploraci\'{o}n aut\'{o}noma, construyendo un \'{a}rbol aleatorio de exploraci\'{o}n r\'{a}pida (RRT), buscando el camino que minimice la incertidumbre del robot con los puntos de referencia del mapa, mientras una segunda ejecuci\'{o}n del algoritmo RRT encuentra el camino hacia el punto de vista seleccionado minimizando la incertidumbre del robot y los puntos de referencia.\\

%\cite{OLEYNIKOVA2018}[\citenum{OLEYNIKOVA2018}] aborda el problema de m\'{i}nimos locales de la función de potencial empleada para la navegación, agregando objetivos secundarios para escapar de dichos mínimos.\\ %Los autores hacen uso de tablas hash que proporcionan una representaci\'{o}n del entorno con r\'{a}pidos tiempos de inserci\'{o}n y consulta de complejidad constante.\\

%Reescribir. No entendí.
%\cite{GAO2018}[\citenum{GAO2018}] propone un algoritmo para la generación de trayectorias utilizando marcha rápida y los polinomios en la base de Bernstein. Al considerar la dinámica del VANT, garantizan la prevención de colisiones manteniendo movimientos suaves y continuos.\\

%\cite{SELIN2019}[\citenum{SELIN2019}] presenta un enfoque para la planificación de una exploración autónoma en entornos tridimensionales a gran escala. Hacen uso del algoritmo RRT insertando valores altos a los vertices con mayores ganancias de informaci\'{o}n y que son usados como objetivos de planificaci\'{o}n de rutas. \\
%La planificación combina la estrategia basada en fronteras con el enfoque de ganancia de información para dirigir al robot a áreas inexploradas, logrando evitar obstáculos a su paso. Mediante un ajuste del plan de exploración en función del tamaño y complejidad del ambiente, muestran su eficiencia maximizando la cobertura de la exploración a medida que minimizan el tiempo de procesamiento.

%\cite{WESTHEIDER2023}\\

%\cite{BARTOLOMEI2023}\\

\subsection{Generación de trayectoria}
%\citeauthor{}[\citenum{}]

La propuesta de \citeauthor{GAO2018}[\citenum{GAO2018}] busca confinar toda la trayectoria dentro del espacio libre. Plantean un programa cuadrático (QP, por sus siglas en inglés) con restricciones, representando la trayectoria en forma de curvas de Bezier por tramos. \citeauthor{MOHTA2017}[\citenum{MOHTA2017}] formulan un QP para la generación de trayectorias donde, además de las restricciones habituales de velocidad, aceleración y jerk, agregan un término en la función de costo proporcional al cuadrado de la distancia entre la trayectoria y los segmentos de línea de la trayectoria modificada. Para asignar tiempo a cada segmento de spline, lo cual es crucial para la viabilidad del QP y la calidad de la trayectoria resultante, utilizan los tiempos obtenidos ajustando un perfil de velocidad trapezoidal a través de los segmentos.

\citeauthor{USENKO2017}[\citenum{USENKO2017}] plantean un problema de replanificación local como la optimización de una función de costo compuesta por un término que penaliza las desviaciones de posición y velocidad al final de la trayectoria, así como un costo por colisión. La trayectoria local se representa a través de un B-spline cúbico uniforme, lo cual simplifica el cálculo de los términos de costo. Por otro lado, \citeauthor{LIN2017}[\citenum{LIN2017}] formulan un problema de optimización no lineal utilizando polinomios de octavo orden para representar la trayectoria.

\citeauthor{CIESLEWSKI2017}[\citenum{CIESLEWSKI2017}] emplea un modo reactivo para generar comandos de velocidad instantáneos basados en las fronteras que se observan en el momento. Para aquellas fronteras dentro del alcance del sensor de profundidad, la velocidad deseada será la máxima y estará orientada hacia el volumen desconocido. En contraste, para las fronteras más cercanas al robot, la velocidad deseada será menor. Por otro lado, el algoritmo presentado por \citeauthor{FLORENCE2018}[\citenum{FLORENCE2018}], busca un movimiento primitivo 3D que maximice el progreso euclidiano hacia el objetivo global, teniendo en cuenta las probabilidades de colisión para trayectorias completas en entornos con obstáculos.

%Un aspecto importante de la navegación en el campo de visión es asegurar que el eje de la cámara esté alineado con la dirección del movimiento. Oleynikova, Taylor, Siegwart y Nieto (2018), y Cieslewski et al. (2017a) intentaron abordar esta restricción implementando un enfoque de seguimiento de velocidad en dirección de cabeceo, pero no garantizan que las trayectorias de cabeceo generadas cumplan con las restricciones dinámicas angulares.

Adicionalmente, es crucial abordar de manera explícita las restricciones dinámicas del movimiento lineal del robot para garantizar su permanencia en áreas seguras. En específico, los trabajos de \citeauthor{CIESLEWSKI2017}[\citenum{CIESLEWSKI2017}], \citeauthor{USENKO2017}[\citenum{USENKO2017}], \citeauthor{SELIN2019}[\citenum{SELIN2019}], \citeauthor{LIN2017}[\citenum{LIN2017}], \citeauthor{COLLINS2019}[\citenum{COLLINS2019}] y \citeauthor{BUG2019}[\citenum{BUG2019}] no gestionan las restricciones dinámicas de forma explícita. En contraste, los algoritmos propuestos en \citeauthor{FLORENCE2018}[\citenum{FLORENCE2018}], \citeauthor{GAO2018}[\citenum{GAO2018}], \citeauthor{OLEYNIKOVA2018}[\citenum{OLEYNIKOVA2018}], \citeauthor{PAPACHRISTOS2017}[\citenum{PAPACHRISTOS2017}], \citeauthor{MOHTA2017}[\citenum{MOHTA2017}], \citeauthor{CINVES2021}[\citenum{CINVES2021}], \citeauthor{RACER2022}[\citenum{RACER2022}] y \citeauthor{BARTOLOMEI2023}[\citenum{BARTOLOMEI2023}] que abordan de manera explícita las restricciones dinámicas del robot.

\subsection{Coordinación multi-robot}

Adem\'{a}s de la planificaci\'{o}n de rutas, la coordinaci\'{o}n de m\'{u}ltiples robots para la exploración requiere de una estrategia de comunicaci\'{o}n efectiva, que garantice la coherencia del mapa que se va generando. Se han investigado diferentes protocolos de comunicaci\'{o}n y estrategias de intercambio de informaci\'{o}n para permitir la colaboraci\'{o}n. Algunos enfoques utilizan comunicaci\'{o}n directa entre los robots, mientras que otros emplean una arquitectura de red donde los m\'{u}ltiples robots se comunican a trav\'{e}s de una infraestructura descentralizada \citeauthor{10120943}[\citenum{10120943}], \citeauthor{BARTOLOMEI2023}[\citenum{BARTOLOMEI2023}], \citeauthor{RACER2022}[\citenum{RACER2022}], \citeauthor{BARTOLOMEI2023}[\citenum{BARTOLOMEI2023}], mostrando la tolerancia a fallas en equipos para tareas de b\'{u}squeda y rescate.

En recientes trabajos \cite{CIESLEWSKI2021}[\citenum{CIESLEWSKI2021}] ha demostrado descentralizar la tarea de SLAM para la creaci\'{o}n de mapas en tareas de exploraci\'{o}n multi-VANT eliminando el bloque de optimizaci\'{o}n, haciendo uso de t\'{e}cnicas de machine learning (teach and repeat).

La dirección que apunta el estado del arte, es en la repartición inteligente de tareas para un problema multi-agente en tareas de exploración.

%La elecci\'{o}n del enfoque depende de las caracter\'{i}sticas de la aplicaci\'{o}n y las restricciones del sistema.\\

%La colaboraci\'{o}n de m\'{u}ltiples VANTS tambi\'{e}n puede implicar la formaci\'{o}n de formaciones o la realizaci\'{o}n de tareas coordinadas. Para ello, se han desarrollado algoritmos de control distribuido que permiten a los VANTS mantener posiciones relativas estables y realizar movimientos coordinados. 

%En t\'{e}rminos de validaci\'{o}n y evaluaci\'{o}n, se utilizan simulaciones y pruebas reales para verificar el rendimiento y la eficacia de los sistemas de colaboraci\'{o}n de m\'{u}ltiples VANTS. Las simulaciones permiten evaluar diferentes escenarios y ajustar los par\'{a}metros del sistema antes de las pruebas reales. Los casos de prueba reales proporcionan informaci\'{o}n sobre la implementaci\'{o}n y la eficiencia en situaciones del mundo real, y pueden ayudar a identificar desaf\'{i}os adicionales que deben abordarse.\\

%\noindent\hrulefill ((IDEAS INICIO)) \noindent\hrulefill\\
  
%\textcolor{red}{Estos problemas ya se han resulto en varios robots terrestres llegando a tener soluciones distribuidas o resultos los problemas de colisi\'{o}n, navegaci\'{o}n, mapeo y se han propuestos buenos algoritmos que formar\'{a}n parte de la arquitectura de software para resolver.}\\
 
%\noindent\hrulefill ((IDEAS FIN)) \noindent\hrulefill\\

%Multirobot\\
%Multi-robot exploration is a popular area of research in robotics, with applications in search and rescue, planetary exploration, and more. The papers present different algorithms and approaches to multi-robot exploration. Pandey 2012 presents an algorithm that takes into account communication constraints between robots and allocates target points to maximize the area explored while minimizing time and distance traveled. Pal 2011 proposes a modification to the A* algorithm for optimal path planning and target allocation strategy. Zlot 2002 presents an approach that uses a market architecture to maximize information gain while minimizing costs, which is reliable and robust to dynamic changes in team members and communication interruptions. Overall, the papers collectively suggest that multi-robot exploration is a complex problem that requires careful consideration of communication constraints, path planning, and target allocation strategies.\\

En el Centro de Investigaci\'{o}n y Estudios Avanzados del Instituto Polit\'{e}cnico Nacional Unidad Tamaulipas se han realizado investigaciones en el \'{a}rea de exploraci\'{o}n multi-robot y dise\~{n}o de prototipos de VANTS, lo cual sirve como antecedente para este trabajo. Este relevante desarrollo, propuesto por \citeauthor{CINVESTAM2013}[\citenum{CINVESTAM2013}], tiene como objetivo principal el despliegue de una estrategia de coordinaci\'{o}n para m\'{u}ltiples robots m\'{o}viles basado en un enfoque de auto-ofertas (método húngaro).

\newpage

\begin{landscape}
  \vspace{1cm}
  \begin{table*}[htbp]
    \centering
    \caption[Trabajos relacionados]{Trabajos relacionados}\label{tab:summary}
    \vspace{0.5cm}
    \scalebox{1.20}{
      \begin{tabular}{ | p{2.5cm} | p{1.5cm} | p{2.3cm} | p{3cm} | p{3cm} | p{0.8cm} | p{1cm} | p{0.7cm} | }
        \hline    
        \tiny REFERENCIA&
        \tiny APLICACIÓN&
        \tiny GENERACIÓN MAPA&
        \tiny PLANIFICACIÓN DE TRAYECTORIA&
        \tiny GENERACIÓN TRAYECTORIA&
        \tiny SENSOR RGB-D&
        \tiny DINÁMICA VANT&
        \tiny multi-VANT\\
        \hline
        %--------------------------
        %\tiny \cellcolor{gray!20}\cite{cinvestav2016}[\citenum{cinvestav2016}]&
        %\tiny \cellcolor{gray!20}Exploración&
        %\tiny \cellcolor{gray!20}Octomap&
        %\tiny \cellcolor{gray!20}Basado en fronteras&
        %\tiny \cellcolor{gray!20}Control directo de velocidad&
        %\tiny \cellcolor{gray!20}\ding{51} &
        %\tiny \cellcolor{gray!20}\ding{55} &
        %\tiny \cellcolor{gray!20}\ding{51} \\ \hline
        %--------------------------
        \tiny \cellcolor{gray!20}\citeauthor{CIESLEWSKI2017}[\citenum{CIESLEWSKI2017}]&
        \tiny \cellcolor{gray!20}Exploración&
        \tiny \cellcolor{gray!20}3D Grid&
        \tiny \cellcolor{gray!20}Basado en fronteras&
        \tiny \cellcolor{gray!20}Control directo de velocidad&
        \tiny \cellcolor{gray!20}\ding{51} &
        \tiny \cellcolor{gray!20}\ding{55} &
        \tiny \cellcolor{gray!20}\ding{55} \\ \hline
        %--------------------------
        \tiny \citeauthor{USENKO2017}[\citenum{USENKO2017}]&
        \tiny Punto Objetivo&
        \tiny Cuadr\'{i}cula egoc\'{e}ntrica Voxel 3D&
        \tiny Offline RRT*&
        \tiny Curvas de Bezier&
        \tiny \ding{51} &
        \tiny \ding{55} &
        \tiny \ding{55} \\ \hline
        %--------------------------
        \tiny \citeauthor{MOHTA2017}[\citenum{MOHTA2017}]&
        \tiny Punto Objetivo&
        \tiny 3D-Local y 2D-Global&
        \tiny A*&
        \tiny Programaci\'{o}n cuadr\'{a}tica&
        \tiny \ding{55} &
        \tiny \ding{51} &
        \tiny \ding{55} \\ \hline
        %--------------------------
        \tiny \citeauthor{LIN2017}[\citenum{LIN2017}]&
        \tiny Punto Objetivo&
        \tiny 3D voxel array TSDF&
        \tiny A*&
        \tiny Optimizaci\'{o}n cuadr\'{a}tica&
        \tiny \ding{55}&
        \tiny \ding{55} &
        \tiny \ding{55} \\ \hline
        %--------------------------
        \tiny \cellcolor{gray!20}\citeauthor{PAPACHRISTOS2017}[\citenum{PAPACHRISTOS2017}]&
        \tiny \cellcolor{gray!20}Exploración&
        \tiny \cellcolor{gray!20}Octomap&
        \tiny \cellcolor{gray!20}Next Best View Planner (NBVP)&
        \tiny \cellcolor{gray!20}Control directo de velocidad&
        \tiny \cellcolor{gray!20}\ding{55}&
        \tiny \cellcolor{gray!20}\ding{51}&
        \tiny \cellcolor{gray!20}\ding{55} \\ \hline
        %--------------------------
        \tiny \citeauthor{OLEYNIKOVA2018}[\citenum{OLEYNIKOVA2018}]&
        \tiny Punto Objetivo&
        \tiny Voxel Hashing TSDF&
        \tiny Next Best View Planner (NBVP)&
        \tiny Optimizaci\'{o}n cuadr\'{a}tica&
        \tiny \ding{51}&
        \tiny \ding{51}&
        \tiny \ding{55} \\ \hline
        %--------------------------
        \tiny \citeauthor{GAO2018}[\citenum{GAO2018}]&
        \tiny Punto Objetivo&
        \tiny Mapa de cuadr\'{i}cula&
        \tiny M\'{e}todo de marcha r\'{a}pida&
        \tiny Optimizaci\'{o}n cuadr\'{a}tica&
        \tiny \ding{55}&
        \tiny \ding{51}&
        \tiny \ding{55} \\ \hline
        %--------------------------
        \tiny \citeauthor{FLORENCE2018}[\citenum{FLORENCE2018}]&
        \tiny Punto Objetivo&
        \tiny Busqueda basada en visibilidad&
        \tiny 2D A*&
        \tiny Control predictivo por modelo (MPC)&
        \tiny \ding{51}&
        \tiny \ding{51}&
        \tiny \ding{55} \\ \hline
        %--------------------------
        \tiny \cellcolor{gray!20}\citeauthor{SELIN2019}[\citenum{SELIN2019}]&
        \tiny \cellcolor{gray!20}Exploración&
        \tiny \cellcolor{gray!20}Octomap&
        \tiny \cellcolor{gray!20}Next Best View Planner (NBVP)&
        \tiny \cellcolor{gray!20}Control directo de velocidad&
        \tiny \cellcolor{gray!20}\ding{55}&
        \tiny \cellcolor{gray!20}\ding{55}&
        \tiny \cellcolor{gray!20}\ding{55} \\ \hline
        %--------------------------
        \tiny \cellcolor{gray!20}\citeauthor{BUG2019}[\citenum{BUG2019}]&
        \tiny \cellcolor{gray!20}Exploración&
        \tiny \cellcolor{gray!20}NA&
        \tiny \cellcolor{gray!20}Swarm Gradient Bug Algorithm (SGBA)&
        \tiny \cellcolor{gray!20}Control directo de velocidad&
        \tiny \cellcolor{gray!20}\ding{55}&
        \tiny \cellcolor{gray!20}\ding{55}&
        \tiny \cellcolor{gray!20}\ding{51} \\ \hline
        %--------------------------
        \tiny \citeauthor{COLLINS2019}[\citenum{COLLINS2019}]&
        \tiny Punto Objetivo&
        \tiny KD Tree $+$ Mapa en Voxel&
        \tiny B\'{u}squeda en Grafo&
        \tiny Movimientos suaves&
        \tiny \ding{51}&
        \tiny \ding{55}&
        \tiny \ding{55} \\ \hline
        %--------------------------
        %\tiny \cite{Zhang2020}[\citenum{Zhang2020}]&
        %\tiny Punto Objetivo&
        %\tiny N/A&
        %\tiny Rapidly Exploring Random Trees (RRT)&
        %\tiny Movimientos suaves&
        %\tiny \ding{51}&
        %\tiny \ding{51}&
        %\tiny \ding{51} \\ \hline
        %--------------------------
        \tiny \citeauthor{CINVES2021}[\citenum{CINVES2021}]&
        \tiny Punto Objetivo&
        \tiny Octree&
        \tiny Rapidly Exploring Random Trees (RRT)&
        \tiny Basado en contornos&
        \tiny \ding{51}&
        \tiny \ding{51}&
        \tiny \ding{55} \\ \hline
        %--------------------------
        \tiny \cellcolor{gray!20}\citeauthor{CIESLEWSKI2021}[\citenum{CIESLEWSKI2021}]&
        \tiny \cellcolor{gray!20}Exploración&
        \tiny \cellcolor{gray!20}Octomap&
        \tiny \cellcolor{gray!20}Basado en fronteras&
        \tiny \cellcolor{gray!20}Control directo de velocidad&
        \tiny \cellcolor{gray!20}\ding{51}&
        \tiny \cellcolor{gray!20}\ding{51}&
        \tiny \cellcolor{gray!20}\ding{51} \\ \hline
        %--------------------------
        \tiny \cellcolor{gray!20}\citeauthor{RACER2022}[\citenum{RACER2022}]&
        \tiny \cellcolor{gray!20}Exploración&
        \tiny \cellcolor{gray!20}Octomap HGrid&
        \tiny \cellcolor{gray!20}Next Best View Planner (NBVP)&
        \tiny \cellcolor{gray!20}Control directo de velocidad&
        \tiny \cellcolor{gray!20}\ding{51}&
        \tiny \cellcolor{gray!20}\ding{51}&
        \tiny \cellcolor{gray!20}\ding{51} \\ \hline
        %--------------------------
        %\scriptsize \cite{WESTHEIDER2023}[\citenum{WESTHEIDER2023}]&
        %\scriptsize Mapa de cuadrícula&
        %\scriptsize Deep Reinforcement Learning&
        %\scriptsize Control directo de velocidad \\ \hline
        %--------------------------
        \tiny \cellcolor{gray!20}\citeauthor{BARTOLOMEI2023}[\citenum{BARTOLOMEI2023}]&
        \tiny \cellcolor{gray!20}Exploración&
        \tiny \cellcolor{gray!20}Octomap&
        \tiny \cellcolor{gray!20}Basado en fronteras&
        \tiny \cellcolor{gray!20}Control directo de velocidad&
        \tiny \cellcolor{gray!20}\ding{51}&
        \tiny \cellcolor{gray!20}\ding{51}&
        \tiny \cellcolor{gray!20}\ding{51} \\ \hline
        %--------------------------
      \end{tabular}
    }
  \end{table*}
\end{landscape}

%\section{Conclusiones}

%The motivation for this work relies on the lack of detailed knowledge about the search behavior of particle swarm optimization and differential evolution on multi-objective optimization. The ability to identify the mechanisms that impact on the search of these two MOEAs will allow for the design of new evolutionary approaches that use a reduced number of objective function evaluations to solve a selection of MOPs.

  
  \chapter{Marco Teórico}

\lipsum[2-4]

%%%%%%%%%%%%%%%%%%%%%%%
%%%%%%%%%%%%%%%%%%%%%%%
%%%%%%%%%%%%%%%%%%%%%%%
%%%%%%%%%%%%%%%%%%%%%%%
%%%%%%%%%%%%%%%%%%%%%%%

\lipsum[2-4]

\section{Conceptos fundamentales} \label{}

\lipsum[2-4]

\subsection{Sistema de ejes coordenados}

\lipsum[2-4]

\subsection{Coordenadas homogéneas}

\lipsum[2-4]

\subsection{Transformaciones geométricas}

\lipsum[2-4]

\section{Funcionamiento de un VANT}

\lipsum[2-4]

\section{Control de un VANT}

\lipsum[2-4]

\section{Estimación de posición}

\lipsum[2-4]



  \chapter{Enfoque propuesto}

El enfoque propuesto para la tesis de exploración coordinada multi-VANT se centra en el desarrollo de un sistema autónomo que emplea varios Vehículos Aéreos No Tripulados (VANTS) para explorar y mapear entornos desconocidos de manera coordinada. Después de identificar técnicas y algoritmos relevantes, seguido del diseño detallado del sistema y su implementación en un entorno simulado.

El diseño e implementación de un sistema autónomo que integre estas tecnologías avanzadas permite abordar los desafíos inherentes a la exploración coordinada con múltiples VANT y complementado con una distrubución de tareas descentralizado.

En resumen, el enfoque propuesto aborda el desafío de la exploración coordinada multi-VANT mediante la integración de algoritmos avanzadas y la evaluación práctica de su desempeño en diferentes ambientes.


\section{Revisión de la literatura}

Realizar una revisión exhaustiva de la literatura sobre técnicas y algoritmos existentes para la exploración coordinada con múltiples VANT. Esto incluiría investigar enfoques de planificación de trayectorias, técnicas de mapeo y localización simultáneos (SLAM), y métodos de coordinación y comunicación entre VANT.

Dentro de la revisión expuesta en el estado del arte se localizaron los siguientes algoritmos que tomaron nuestro interes.

A*, E-scaling, diagramas de voronoi, ?? 

\section{Diseño del sistema}

Desarrollar un diseño detallado del sistema que incluya la arquitectura general, los componentes individuales y los algoritmos específicos que se utilizarán para la planificación de trayectorias, el mapeo del entorno y la coordinación entre los VANT.

\subsection*{Odometría VANT}


\end{doublespace}

\appendix

\begin{doublespace}
%\chapter{Publicaciones}
%Dar el formato que considere apropiado. Presentar, por ejemplo: Conferencias, Revistas internacionales, etc...
\begin{itemize}
\item Jorge Sebastian Hernández Domínguez and Gregorio Toscano Pulido. \emph{A Comparison on the Search of Particle Swarm Optimization and Differential Evolution on Multi-Objective Optimization}, in IEEE Congress on Evolutionary Computation (CEC 2011), New Orleans, LA, USA, June 2011.
\item 
Jorge Sebastian Hernández Domínguez, Gregorio Toscano Pulido, and Carlos A. Coello Coello, \emph{A Multi-objective Particle Swarm Optimizer Enhanced with a Differential Evolution Scheme}, in International Conference on Artificial Evolution (EA 2011), Angers, France, October 2011.

\end{itemize}



%Las publicaciones generadas por este trabajo de investigación deben publicarse en la sección de 'Anexos'

%\include{appendixA}

%\include{appendixB}

%\bibliographystyle{apalike}
%\setcitestyle{numbers}
%\bibliographystyle{unsrtnat}%{plainnat}%{abbrvnat}
\bibliography{EMOO,protocolo}

\end{doublespace}

\end{document}
