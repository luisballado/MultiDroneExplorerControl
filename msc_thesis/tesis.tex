%% LyX 1.6.7 created this file.  For more info, see http://www.lyx.org/.
%% Do not edit unless you really know what you are doing.

\documentclass[spanish,ESP,MSc]{cinvestav} %si quieren en inglés, usen english, ENG
\usepackage{cite}
\usepackage{slashbox}
\usepackage[T1]{fontenc}
\usepackage[utf8]{inputenc}
\setcounter{secnumdepth}{3}
\setcounter{tocdepth}{3}
\usepackage{graphics,epsfig,latexsym,amssymb}
\usepackage{tabularx}
\usepackage{titlesec}
\usepackage{amsmath}
\usepackage{blindtext}
\usepackage{hyperref}
%\usepackage[hypertex]{hyperref}
\usepackage{array}
\usepackage{float}
\usepackage{rotfloat}
\usepackage{textcomp}
\usepackage{graphicx}
\usepackage{color}
\usepackage{xcolor}
\usepackage{setspace}
\usepackage[TABBOTCAP]{subfigure}
\usepackage{algorithm}
\usepackage{captcont}
\usepackage{afterpage}
\usepackage{lscape}
\usepackage{algorithmic}
\usepackage{longtable}
%\usepackage{caption}
\usepackage{caption}[=v1]
\usepackage[authoryear,square]{natbib}

%\renewcommand{\thesubfigure}{\thefigure.\arabic{subfigure}} 
%\makeatletter 
%\renewcommand{\p@subfigure}{} 
%\renewcommand{\@thesubfigure}{\thesubfigure:\hskip\subfiglabelskip} 
%\makeatother 

%\usepackage[nonamebreak]{natbib}


\makeatletter

%%%%%%%%%%%%%%%%%%%%%%%%%%%%%% LyX specific LaTeX commands.
%% Because html converters don't know tabularnewline
\providecommand{\tabularnewline}{\\}
%% A simple dot to overcome graphicx limitations
\newcommand{\lyxdot}{.}

\floatstyle{ruled}
\newfloat{algorithm}{tbp}{loa}
\floatname{algorithm}{Algorithm}

%%%%%%%%%%%%%%%%%%%%%%%%%%%%%% User specified LaTeX commands.

%\usepackage{apacite}
%\usepackage{natbib}
%\let\citep=\citeA
%\let\cite=\citeA
\bibpunct{[}{]}{,}{n}{,}{,} 

%\usepackage[lined,boxed]{algorithm2e}
%\decimalpoint 

\title{Estrategias para la exploración coordinada multi-VANT}
\titleen        {A Study on the Mechanisms of two Multi-objective Evolutionary Algorithms}
\author         {Luis Alberto Ballado Aradias}
\department     {Cinvestav Unidad Tamaulipas}
\departmenten   {Cinvestav Unidad Tamaulipas}
\degreein       {Ciencias en Ingeniería y Tecnologías Computacionales}
\degreeinen     {Computer engineering and technology}
\city           {Cd. Victoria, Tamaulipas, M\'{e}xico.}
\date           {\today}
\degreeday      {8}
\degreeyear     {2024}
\degreemonth  {Septiembre}
\degreemonthen  {September}

%A partir de 2015, este fondo ya no se utiliza
%\acknowledgmenttoproject{
%This research was partially funded by project number 51623 from ``Fondo Mixto
%Conacyt-Gobierno del Estado de Tamaulipas''
%}

%%%%%%%%%%%%%%%%%%%%%%%%%%%%%%%%%%%%%%%%%%%%%%%%%%%%%%%%%%%%%%%%%%%%%%%%

\chair          {Dr. José Gabriel Ramirez-Torres}
\chair          {Dr. Eduardo Arturo Rodriguez-Tello}
\member         {Dr. Mario Garza-Fabre}
\member         {Dr. Ricardo Landa-Becerra}

%%%%%%%%%%%%%%%%%%%%%%%%%%%%%%%%%%%%%%%%%%%%%%%%%%%%%%%%%%%%%%%%%%%%%%%%

\dedication     {A mi familia}

%%%%%%%%%%%%%%%%%%%%%%%%%%%%%%%%%%%%%%%%%%%%%%%%%%%%%%%%%%%%%%%%%%%%%%%%

\abstract{
\vspace*{-7mm}
La \emph{Optimización Mediante Cúmulos de Partículas (PSO)} y la \emph{Evolución Diferencial (DE)} son dos \emph{Algoritmos Evolutivos (EAs)} simples de conceptualizar que han sido exitosamente utilizados para resolver problemas mono-objetivo. Dicha simplicidad y éxito han promovido su uso en problemas multi-objetivo. Aun cuando a la fecha existen varias propuestas \emph{PSO Multi-objetivo (MOPSO)} y \emph{DE Multi-objetivo (MODE)}, el conocimiento sobre el proceso de búsqueda que realizan estas dos metaheurísticas es escaso dado que solo existen algunos trabajos teóricos y únicamente mono-objetivo. Como resultado, no se conoce claramente el comportamiento de estos algoritmos evolutivos en problemas multi-objetivo. Esta tesis presenta un estudio empírico sobre estos dos \emph{Algoritmos Evolutivos Multi-objetivo (MOEAs)}. Dicho estudio consta de una serie de experimentos que comparan diferentes variantes en DE y fórmulas de vuelo en PSO. Después, se evalúa la manera en que estos dos MOEAs generan nuevas soluciones y se identifican características de dichas soluciones y su relación con los mecanismos presentes en ambos enfoques. Estos experimentos permitieron concluir que MOPSO se mueve agresivamente hacia regiones prometedoras lo cual puede deteriorar la búsqueda. MODE por otro lado, realiza una búsqueda pasiva basada en pasos pequeños que a la larga le permiten seguir moviéndose hacia el frente de Pareto. El conocimiento obtenido fue usado para diseñar dos nuevos MOEAs que mostraron ser competitivos al ser comparados con tres algoritmos (OMOPSO, NSGA-II y DEMO) representativos del estado del arte. 
}

\abstracten{
\vspace*{-7mm}
\emph{Particle Swarm Optimization (PSO)} and \emph{Differential Evolution (DE)} are two \emph{Evolutionary Algorithms (EAs)} which are very simple to conceptualize and have shown excellent results on single-objective optimization problems. As expected, this simplicity and success have promoted their migration to multi-objective optimization. Even when several \emph{Multi-Objective Particle Swarm Optimizers (MOPSOs)} and \emph{Multi-objective Differential Evolution (MODE)} algorithms are available to this date, knowledge about the search performed by these two metaheuristics is limited in regards to multi-objective optimization since only some theoretical single-objective studies have been developed. As a result, there is uncertainty in regards to the search behavior of these \emph{Multi-objective Evolutionary Algorithms (MOEAS)}. This thesis work performs an empirical study about the search of these two MOEAs. The performed analysis develops a series of experiments that compare several DE variants and PSO flight formulas. Thereafter, the manner in which these two MOEAs generate new solutions is evaluated and certain characteristics of these solutions and their relationship to the mechanisms found on a MOEA are identified. These experiments allowed to conclude that MOPSO performs an aggressive search towards promissory regions which might result in stagnation of the search. On the other hand, MODE performs a more passive search taking small steps which on the long run allowed it to continue improving towards the true Pareto front. This knowledge was further used to design two new MOEAs. Results indicate that both algorithms are very competitive with respect to three algorithms (OMOPSO, NSGA-II, and DEMO) representative of the state of the art. 
}

%Las publicaciones generadas por este trabajo de investigación deben publicarse en la sección de 'Anexos'

\acknowledgments {
\begin{itemize}
\item { Le agradezco a spiderman y pongan m\'as agradecimientos como los siguientes:}
\item { I also thank the administrative personnel at CINVESTAV-Tamaulipas for their help during my stay}
\item { I thank CONACyT for the provided economic support which allowed me to concentrate in my studies and CINVESTAV-Tamaulipas for the opportunity to pursue graduate studies}
\item { I also acknowledge support from CONACyT through project 105060 ``Uso de técnicas evolutivas híbridas para resolver problemas de optimización multiobjetivo dinámicos y con más de tres objetivos'' under the lead of Dr. Gregorio Toscano Pulido}
\end{itemize}
}

\nomenclature {
\begin{longtable}{ll}
\textbf{BBDE} & Bare Bones Differential Evolution\tabularnewline
\textbf{BBPSO} & Bare Bones Particle Swarm Optimization\tabularnewline
\textbf{DelMiDE} & Delayed Micro Differential Evolution\tabularnewline
\textbf{DTLZ} & Deb-Thiele-Leumman-Zitzler test suite\tabularnewline
\textbf{EA} & Evolutionary Algorithm\tabularnewline
\textbf{EC} & Evolutionary Computation\tabularnewline
\textbf{ES} & Evolutionary Strategy\tabularnewline
\textbf{EP} & Evolutionary Programming\tabularnewline
\textbf{DE} & Differential Evolution\tabularnewline
\textbf{DEMO} & An specific implementation of Multi-objective Differential Evolution\tabularnewline
\textbf{GA} & Genetic Algorithm\tabularnewline
\textbf{GD} & Generational Distance\tabularnewline
\textbf{IFF} & If and only if\tabularnewline
\textbf{IGD} & Inverted Generational Distance\tabularnewline
\textbf{MODE} & Multi-objective Differential Evolution\tabularnewline
\textbf{MOEA} & Multi-objective Evolutionary Algorithm\tabularnewline
\textbf{MOP} & Multi-objective Optimization Problem\tabularnewline
\textbf{MOPEDS} & Multi-objective Particle Swarm Optimization Enhanced with a Differential Evolution Scheme\tabularnewline
\textbf{MOPSO} & Multi-objective Particle Swarm Optimizer\tabularnewline
\textbf{NSGA-II} & Non-dominated Sorting Genetic Algorithm\tabularnewline
\textbf{PF} & Pareto Optimal Front\tabularnewline
\textbf{PS} & Pareto Optimal Set\tabularnewline
\textbf{PSO} & Particle Swarm Optimization\tabularnewline
\textbf{SMPSO} & Speed constrained Multi-objective Particle Swarm Optimization\tabularnewline
\textbf{WRT} & With Respect To\tabularnewline
\textbf{ZDT} & Zitzler-Deb-Thiele test suite\tabularnewline
\tabularnewline
\tabularnewline
\end{longtable}
}

 \floatname{algorithm}{Algorithm}
%\renewcommand{\algorithmicrequire}{\textbf{Entrada:}}
%\renewcommand{\algorithmicensure}{\textbf{Salida:}}

%\@ifundefined{showcaptionsetup}{}{%
 %\PassOptionsToPackage{caption=false}{subfig}}
%\usepackage{subfig}
\makeatother

%\usepackage{babel}
%\addto\shorthandsspanish{\spanishdeactivate{~<>}}

\setcounter{lofdepth}{1} 
\setcounter{lotdepth}{1} 

\begin{document}
%\maketitle
\makeintropages

%\renewcommand{\tablename}{Tabla}

\begin{doublespace}
\chapter{Introducción}

\lipsum[2-4]

\section{Antecedentes y motivación} 

% usar --> para resaltar cosas \textbf{\emph{}}

\lipsum[2-4]

\section{Planteamiento del problema} 

\lipsum[2-4]

\section{Hipótesis y preguntas de investigación}

\lipsum[1]

\section{Objetivos}

The main objective of this thesis work can be defined as follows: \emph{``To contribute to the state of the art by identifying and understanding the mechanisms that promote good performance on multi-objective particle swarm optimization and multi-objective differential evolution and that allow a multi-objective evolutionary algorithm to reduce the number of objective function evaluations needed to solve an arbitrary MOP''}. 

This main goal has been divided in the following specific objectives: 
\begin{itemize}\setlength{\itemsep}{-1mm}
	\item To evaluate different strategies to generate solutions in DE and to move particles in PSO on multi-objective problems.
	\item To perform a series of experiments that allow to better understand the behavior of PSO and DE on multi-objective problems.
	\item To identify the mechanisms that impact (either enhance it or deter it) in the search of the metaheuristics particle swarm optimization and differential evolution when attacking multi-objective optimization problems. 
	\item To contribute to the state of the art with at least one multi-objective evolutionary algorithm that utilizes knowledge derived from this work to enhance the search process on diversity and convergence. 
	\item To validate the performed experiments using performance measures and test problems taken from the specialized literature.
\end{itemize}

\section{Solución propuesta}

\lipsum[2-4]

\section{Resumen}

\lipsum[2-4]

%The rest of this document is organized as follows: Chapter~\ref{chapter2} presents basic concepts and background in the field of optimization. Then, Chapter~\ref{ch:PSODE} introduces particle swarm optimization and differential evolution which are the two metaheuristics on which this thesis work focusses. In order to introduce these two metaheuristics, EAs is general are also described in this chapter. Afterwards, Chapter~\ref{experiments} presents a series of experiments that were developed and that allowed to obtain further information about the search performed by PSO and DE in multi-objective optimization. This knowledge was used to develop two new MOEAs which are presented and evaluted in Chapter~\ref{proposals}. Finally, Chapter~\ref{conclusion} concludes this thesis work. 


\chapter{Estado del Arte}

En este capítulo, se presenta un panorama de las distintas propuestas en exploración que involucran el uso de múltiples vehículos aéreos no tripulados (VANTs), evaluando tanto sus ventajas como sus limitaciones. Nuestra atención se centra especialmente en las propuestas relacionadas con la exploración multi-robot. % y la exploración multi-robot en entornos con incertidumbre en la posición.

%Este capítulo ofrece un panorama general de las principales propuestas en exploración empleando múltiples vehículos aéreos no tripulados (VANTS), discutiendo sus fortalezas y sus debilidades. Nos enfococamos principalmente en las propuestas de exploración multi-robot, SLAM multi-robot y exploración multi-robot con incertidumbre en la posición.

La exploración multi-robot se enfoca principalmente en guiar a cada robot de forma eficiente con la finalidad de obtener una representación del medio ambiente asumiendo, que la localización es perfecta y conocida en todo momento. El problema de localización y mapeo simultáneo (SLAM) integra la localización y el aprendizaje de mapas como el problema de construir un mapa mientras se trata de localizar al robot dentro del mismo.

Para lograr una exploración efectiva, los robots deben coordinar sus acciones, compartir información sobre el entorno y planificar sus movimientos de manera inteligente, teniendo en cuenta la distribución de tareas, la comunicación entre robots y la detección de obstáculos.%Finalmente, el problema de exploración multi-robot con incertidumbre en la posición representa el problema de generar un mapa mientras se estima la localización y se guía al robot hacia objetivos definidos con la finalidad de generar el mapa de forma eficiente.

%A continuación se analizan las principales propuestas para la generación de mapas con múltiples robots, que comparten rasgos comunes con el enfoque desarrollado en este trabajo de tesis.

\section{Introducción}

Las aplicaciones de la rob\'{o}tica industrial se han centrado en realizar tareas simples y repetitivas. La necesidad de robots con capacidad de identificar cambios en su entorno y reaccionar sin la intervenci\'{o}n humana, da origen a los robots inteligentes. Aunado a ello, si deseamos que el robot se mueva libremente, los cambios en su entorno pueden aumentar r\'{a}pidamente y complicar el problema de desarrollar un robot que muestre un comportamiento inteligente.

El despliegue r\'{a}pido de robots en situaciones de riesgo, b\'{u}squeda y rescate ha sido un \'{a}rea ampliamente estudiada en la rob\'{o}tica m\'{o}vil. Recientemente, concursos como el DARPA Subterranean (SubT) Challenge, buscan acelerar la investigación y desarrollo de tecnología en escenarios complejos subterráneos, donde factores como exploraci\'{o}n, planeaci\'{o}n y coordinaci\'{o}n son clave para lograr los objetivos del Reto [\citenum{DARPA2022}], la creación de una estategia de exploración es pieza clave para el cumplimiento del reto.

%%%%%%%%%%%%%%%%%%%%%%%
%%%%%%%%%%%%%%%%%%%%%%%
%%%%%%%%%%%%%%%%%%%%%%%
%%%%%%%%%%%%%%%%%%%%%%%
%%%%%%%%%%%%%%%%%%%%%%%

La exploraci\'{o}n de un ambiente desconocido empleando multi-VANT es un \'{a}rea relativamente nueva y con mucho crecimiento en los \'{u}ltimos a\~{n}os. Se han abordado una variedad de temas para lograr la exploraci\'{o}n aut\'{o}noma, desde la planificaci\'{o}n de rutas para m\'{u}ltiples robots terrestres en tareas de exploraci\'{o}n [\citenum{Sharma2016}], estrategias para la coordinaci\'{o}n y protocolos de comunicaci\'{o}n [\citenum{Gielis2022}]. Diversos estudios de aplicaciones multi-VANT se han realizado para resolver tareas como el monitoreo ambiental [\citenum{SurveyCollab2019}], la agricultura de precisi\'{o}n [\citenum{10011226}] y operaciones de b\'{u}squeda y rescate [\citenum{SHAKHATREH2019}]. A pesar de que los VANTS comerciales modernos cuenten con avanzadas integraciones en los problemas de navegación. Siguen sin incorporar una solución completa de autonomía.

La direcci\'{o}n en que apunta el estado del arte actualmente se puede atribuir a los avances en tecnolog\'{i}a en la \'{u}ltima d\'{e}cada que nos han permitido dotar de autonomía a VANTS de menor tamaño. Investigadores de diversas \'{a}reas, que incluyen las ciencias computacionales y la ingeniería, han contribuido al progreso de este campo.

Sin embargo, la necesidad de una arquitectura de software descentralizada que engloble los retos de exploración, junto con un mecanismo de coordinación para múltiples VANTS, surge como una área de investigación prometedora que garantice el funcionamiento correcto de los componentes necesarios para realizar tareas de exploración coordinada multi-VANT.

En esta revisión del estado del arte se citan trabajos relacionados a la coordinación multi-VANT para tareas de exploración, que sirven de guía para la realización de este trabajo.

%\section{Conceptos básicos} \label{}

La exploración es una tarea fundamental en robótica, el objetivo es construir un mapa de un espacio desconocido con ayuda de un algoritmo de planificación de trayectoria que guiará al robot donde moverse. Una representación del ambiente (mapa) es necesaria para tomar la decisión de la mejor ruta hacia un mayor conocimiento del espacio que lo rodea.

%Para poder construir una buena representación del ambiente, debemos comprender como realizar una exploración. En una planificación para una exploración, después de una observación, el robot debe decidir donde moverse.

%Buscando acelerar la exploración, el robot debe percibir el ambiente desconocido lo más rápido posible. En la literatura existen dos principales técnicas:

\section{Trabajos relacionados} \label{}

El presente texto se centra en el estado del arte de las estrategias para la exploración coordinada con múltiples Vehículos Aéreos No Tripulados (VANT), abordando diversos temas fundamentales en este campo de investigación. La exploración, la representación del medio ambiente en 3D, la planificación de trayectorias, la generación de trayectorias y la coordinación multirobot son aspectos cruciales que influyen en el diseño y la implementación de sistemas autónomos para la exploración coordinada. En este contexto, se analizan las tendencias actuales, los enfoques más relevantes y los desafíos existentes.


\subsection{Exploración}

Para poder construir una buena representación del ambiente, debemos comprender como realizar una exploración. En una planificación para una exploración, después de una observación, el robot debe decidir donde moverse.

Buscando acelerar la exploración, el robot debe percibir el ambiente desconocido lo más rápido posible. En la literatura existen dos principales técnicas:

\begin{itemize}\setlength{\itemsep}{-1mm}
\item Next Best View (NBV) planning - El planificador elige la siguiente posición en base a la percepción y una función de utilidad que maximice la ganancia de información. 
\item Basado en fronteras - La estrategia es simple y eficiente al asignar rutas al robot en base a la frontera más cercana.
\end{itemize}

%Las bases para la exploraci\'{o}n aut\'{o}noma e innovaciones son heredadas de algoritmos ya empleados en la rob\'{o}tica m\'{o}vil (tabla \ref{pathsummary}). Uno de los primeros trabajos en la exploraci\'{o}n con robots, es la propuesta de fronteras por \citeauthor{YAMAUCHI1997}[\citenum{YAMAUCHI1997}], donde define como una frontera a la l\'{i}nea entre las zonas exploradas y las no exploradas de un \'{a}rea de inter\'{e}s. Durante la navegaci\'{o}n la informaci\'{o}n percibida por el robot crece, moviendo las fronteras hasta que no existan m\'{a}s fronteras.

%En el trabajo de \citeauthor{Faria2019}[\citenum{Faria2019}], se combina la estrategia basada en fronteras con t\'{e}cnicas de planificaci\'{o}n de trayectorias Lazy Theta* en un VANT.\\  

%\cite{FRAUNDORFER2012}[\citenum{FRAUNDORFER2012}] hacen uso de una exploraci\'{o}n con fronteras a partir de una navegaci\'{o}n aut\'{o}noma aplicando un algoritmo tipo bug para el seguimiento de una pared, empleando campos de potencial artificial para una planificaci\'{o}n local en un mapa de ocupaci\'{o}n tipo grid. Estos trabajos demostraron la navegaci\'{o}n aut\'{o}noma de veh\'{i}culos a\'{e}reos no tripulados y que estos pueden seguir puntos de referencia en el mapa, evitar obst\'{a}culos y llevar a cabo tareas de exploraci\'{o}n en entornos complejos.\\

Los trabajos de \citeauthor{CIESLEWSKI2017}[\citenum{CIESLEWSKI2017}], \citeauthor{BARTOLOMEI2023}[\citenum{BARTOLOMEI2023}] se enfocan en la exploración haciendo uso de estrategias basadas en fronteras.

Mientras que los trabajos de \citeauthor{PAPACHRISTOS2017}[\citenum{PAPACHRISTOS2017}],\citeauthor{SELIN2019}[\citenum{SELIN2019}], \citeauthor{RACER2022}[\citenum{RACER2022}], que también trabajan en el problema de exploración autónoma. Utilizan la estrategia de Next Best View Planner guiando al robot donde puedan aportar más información en la construcción del medio ambiente (mapa). Por el contrario \citeauthor{BUG2019}[\citenum{BUG2019}], al tener un comportamiento tipo bug hace uso de su propuesta con Swarm Gradient Bug Algorithm.

\subsection{Representación medio ambiente 3D}

%Los primeros trabajos de VANT autónomos se encuentran en las aportaciones de \citeauthor{SHEN2011}[\citenum{SHEN2011}], que hacen uso de un VANT con la propuesta de dos planificadores de trayectorias con un control proporcional con retroalimentaci\'{o}n y basados en RRT*, para conseguir una representaci\'{o}n del mundo en 2D empleando un sensor tipo LiDAR. Por otra parte, los trabajos de \citeauthor{GRZONKA2012}[\citenum{GRZONKA2012}] tambi\'{e}n hacen una representaci\'{o}n del entorno en 2D, haciendo uso de algoritmos que trabajan en mapas densos tipo grid, y del algoritmo D* lite para la planificaci\'{o}n de trayectorias.\\

Con la llegada de las primeras c\'{a}maras capaces de obtener valores de profundidad (RGB-D), y con mayores capacidades de almacenamiento en menos espacio, nos permiten tratar el medio ambiente a través de representaciones tridimensionales. Podemos citar por ejemplo la propuesta de estructura de datos basada en grafos octrees por \citeauthor{DONALD1982}[\citenum{DONALD1982}] con una baja complejidad en el orden logarítmico, por consiguiente en el año 2013 se introdujo un nuevo concepto para la representaci\'{o}n de mapas 3D basados en esos principios, haciendo que la representaci\'{o}n de entornos 3D se realice de manera eficiente para aplicaciones en rob\'{o}tica donde se necesitan algoritmos r\'{a}pidos. Los trabajos de \citeauthor{ARMIN2013}[\citenum{ARMIN2013}] introducen los Octomaps, que se utilizan para representar mapas tridimensionales como subdivisiones marcadas como ocupadas, desocupadas y desconocidas, para aplicaciones de navegaci\'{o}n.

En recientes trabajos \citeauthor{min2020accelerating}[\citenum{min2020accelerating}] proponen dar soluci\'{o}n a los cuellos de botella que se presentan en el Octomap buscando acelerar los tiempos de cómputo en la construcci\'{o}n de mapas a partir de la implementaci\'{o}n de Aceleradores Gr\'{a}ficos GPU.

Los trabajos de \citeauthor{CIESLEWSKI2017}[\citenum{CIESLEWSKI2017}] hacen uso de la representaci\'{o}n del entorno por medio de una rejilla tridimensional (voxel grids), la versión tridimensional de la rejilla de ocupación empleada en robótica móvil terrestre, para planificar trayectorias de exploraci\'{o}n. \citeauthor{USENKO2017}[\citenum{USENKO2017}] proponen el uso del mapa centrando al robot en un círculo tridimensional de tama\~{n}o fijo, por su parte \citeauthor{MOHTA2017}[\citenum{MOHTA2017}] hacen uso de un mapa h\'{i}brido formado con la combinaci\'{o}n de un mapa local 3D con un mapa global en 2D. \citeauthor{FLORENCE2018}[\citenum{FLORENCE2018}] propone un framework para el manejo de datos para mapas 3D a partir de la información de una cámara de profundidad (RGB-D). La propuesta hace uso de una estructura con cuadrículas de ocupación y datos de profundidad.

Diversos trabajos en los que se incluyen a \citeauthor{GAO2018}[\citenum{GAO2018}], \citeauthor{LIN2017}[\citenum{LIN2017}], \cite{OLEYNIKOVA2018}[\citenum{OLEYNIKOVA2018}], añaden una estructura adicional a su representación del medio ambiente, usando distancias que permiten la evasión de obstáculos de forma segura. Sin embargo estas soluciones pueden ser costosas requiriendo un mayor procesamiento de computo.

\citeauthor{COLLINS2019}[\citenum{COLLINS2019}] usan una representaci\'{o}n local del mapa con ayuda de una estructura de datos KD-Tree. Usa un mapa representado en voxels, mientras que un grafo topol\'{o}gico representa todo el entorno explorado. Por otra parte el trabajo de \citeauthor{BUG2019}[\citenum{BUG2019}] no hace uso de un mapa. Al generar movimientos reactivos tipo bug, logra generar una navegación autónoma con odometría visual.

Los trabajos de \citeauthor{PAPACHRISTOS2017}[\citenum{PAPACHRISTOS2017}], \citeauthor{SELIN2019}[\citenum{SELIN2019}], \citeauthor{CINVES2021}[\citenum{CINVES2021}], \citeauthor{RACER2022}[\citenum{RACER2022}], \citeauthor{BARTOLOMEI2023}[\citenum{BARTOLOMEI2023}], optan por el uso de OctoMaps. Haciendo de la representación de ambientes en 3D con octomaps un estandard en la robótica moderna.

\subsection{Planificación de trayectorias}

Uno de los desaf\'{i}os clave en la colaboraci\'{o}n de m\'{u}ltiples VANTS es la planificaci\'{o}n de rutas. Se han desarrollado diversos algoritmos para optimizar la planificaci\'{o}n de rutas dentro de la rob\'{o}tica m\'{o}vil, minimizando los riesgos de colisi\'{o}n y mejorando la eficiencia en sus misiones. Estos algoritmos tienen en cuenta varios factores como las restricciones del robot y la ubicación del objetivo, para generar trayectorias seguras.

El objetivo principal de los algoritmos de planificación de trayectorias, es el de guiar al robot desde el punto de inicio al punto destino. Los trabajos por \citeauthor{Lumelsky1987}[\citenum{Lumelsky1987}], dieron respuesta a problemáticas de navegaci\'{o}n eficiente, que no requieren de una representación del medio ambiente y emplean, por lo tanto, pocos recursos computacionales y de memoria (algoritmos tipo bug).

%Siendo el v\'{e}rtice  la posici\'{o}n del robot y las aristas un camino donde encontramos los trabajos de

Matem\'{a}ticamente, el problema de planificación de trayectorias es resuelto a través del modelado del medio ambiente utilizando grafos, siendo un grafo una representaci\'{o}n matem\'{a}tica de v\'{e}rtices y aristas. \citeauthor{4082128}[\citenum{4082128}], al mejorar el algoritmo de Dijkstra para el robot Shakey, logr\'{o} navegar en una habitaci\'{o}n que conten\'{i}a obst\'{a}culos fijos. El objetivo principal del algoritmo A* es la eficiencia en la planificaci\'{o}n de rutas al incorporar una heurística. A su vez, el algoritmo D*, propuesto por \citeauthor{351061}[\citenum{351061}], ha demostrado operar de manera eficiente ante obst\'{a}culos din\'{a}micos; en comparaci\'{o}n con el algoritmo A* que vuelve a ejecutarse al encontrarse con un obst\'{a}culo no previsto inicialmente, el algoritmo D* usa la informaci\'{o}n previa para buscar una nueva ruta hacia el objetivo.

%\begin{table}[h!]
%\centering
%\scalebox{0.65}{
%\begin{tabular}[t]{l|c|c|c|p{9cm}}
%\hline\hline
%M\'{e}todo&Completez&\'{O}ptimo&Escalable&Notas\\
%\hline\hline
%Grafo de visibilidad [\citenum{8798322}]& \ding{51} & \ding{51} & ding{55} & \begin{itemize}[left=0pt,topsep=0pt]
%  \setlength\itemsep{0.1em}
%\item Mucho espacio libre
%\item Mala escalabilidad
%\item El robot pasa cerca de obstáculos
%\end{itemize}\nointerlineskip\\
%\hline
%Diagramas de Voronoi [\citenum{7978644}] & \ding{51} & \ding{55} &\ding{55} & \begin{itemize}[left=0pt,topsep=0pt]
%\item Espacio libre m\'{a}ximo
%\item Rutas conservadoras
%\item Mala escalabilidad
%\end{itemize}\nointerlineskip\\
%\hline
%Campos de potencial artificial [\citenum{9836159}]& \ding{51} & \dng{55} & Depende del ambiente & \begin{itemize}[left=0pt,topsep=0p]
%\item F\'{a}cil de implementar
%\item Susceptible a m\'{i}nimos locales
%\end{itemize}\nointerlineskip\\
%\hline
%Dijkstra/A* [\citenum{Wang2022}]& \ding{51} & Grafo & \ding{55} & begin{itemize}[left=0pt,topsep=0pt]
%\item M\'{a}s r\'{a}pido que la b\'{u}squeda desinformada
%\item A* usa una funci\'{o}n heur\'{i}stica para impulsar la b\'{usqueda de manera eficiente
%\item Mala escalabilidad
%\end{itemize}\nointerlineskip\\
%\hline
%PRM [\citenum{6491122}]& \ding{51} & Grafo & \ding{51} & \begin{itmize}[left=0pt,topsep=0pt]
%\item Eficiente para problemas con consultas m\'{u}ltiples
%\item Completez probabil\'{i}stica
%\item  Camino irregular
%\end{itemize}\nointerlineskip\\
%\hline
%RRT [\citenum{9018916}]& \ding{51} & \ding{55} & \ding{51} & \begi{itemize}[left=0pt,topsep=0pt]
%\item Eficiente para problemas de consulta \'{u}nica
%\item Completez probabil\'{i}stica
%\item Camino irregular
%\end{itemize}\nointerlineskip\\
%\hline
%\end{tabular}
%}
%\setlength\tabcolsep{0pt}
%\caption{\label{pathsummary}M\'{e}todos para planificaci\'{o}n de rayectorias usados en rob\'{o}tica m\'{o}vil}
%\end{table}%

Por otra parte, el algoritmo RRT (Rapidly Exploring Random Trees), propuesto por \citeauthor{LaValle1998RapidlyexploringRT}[\citenum{LaValle1998RapidlyexploringRT}], es ampliamente usado para la planificaci\'{o}n de rutas en robots modernos. El algoritmo construye de forma incremental una estructura de \'{a}rbol mediante un muestreo aleatorio en el espacio de configuraciones, uniendo aleatoriamente nuevas posiciones al \'{a}rbol existente hasta alcanzar la posición final. Las modificaciones realizadas al algoritmo RRT por \citeauthor{Karaman2011}[\citenum{Karaman2011}], incorporando una heur\'{i}stica de costo por recorrer, permite encontrar rutas casi \'{o}ptimas de manera eficiente.

En recientes trabajos de \citeauthor{yang2022far}[\citenum{yang2022far}], muestran la capacidad de implementaci\'{o}n de algoritmos clásicos de planificación de trayectorias, como los grafos de visibilidad, para tareas en entornos conocidos y no conocidos, utilizando una representaci\'{o}n del ambiente a base de polígonos, logrando un r\'{a}pido planificador que tambi\'{e}n resuelve obst\'{a}culos nuevos en el ambiente, logrando resultados comparables a las estrategias más recientes como D* e inclusive RRT*.

%Trabajos como el de \citeauthor{CERBERUS2022}[\citenum{CERBERUS2022}] en el que han logrado optimizar problemas de alta dimensionalidad como el control de navegaci\'{o}n para un robot con cuatro patas, haciendo uso de aprendizaje por refuerzo y con ayuda de simulaciones, logrando obtener un esquema de control que le permiten al robot resolver el problema de navegaci\'{o}n. Sin embargo, al momento de probar el esquema en un robot real, el robot no pudo efectuar un paso correcto. Este problema se debe a la distancia que existe entre la simulaci\'{o}n y la realidad, en particular al no considerar las incertidumbres en las lecturas de los sensores. \\

%Las simulaciones permiten demostrar el correcto funcionamiento de los esquemas de control. A través de simulaciones híbridas e introduciendo ruidos estocásticos en las simulaciones, es posible lograr resultados muy prometedores, como en el caso de éxito en el DARPA Subterranean Challenge [\citenum{DARPA2022}], que utiliza una exploraci\'{o}n basada en grafos y un mapa de ocupaci\'{o}n (Octomap) para simular el entorno tridimensional.\\

Un enfoque muy recurrente para abordar el problema de planificación de trayectorias, es el uso de las metaheur\'{i}sticas bio-inspiradas (Algoritmos Genéticos (GA), Particle Swarm Optimization (PSO), Ant Colony Optimization (ACO), Firefly Algorithm (FA)). Estas estrategias se inspiran en sistemas y procesos biol\'{o}gicos para resolver problemas complejos de optimizaci\'{o}n.
%Existen varios tipos de metaheur\'{i}sticas bio-inspiradas:

%\begin{enumerate}
%  \item \textbf{Algoritmos Gen\'{e}ticos (GA)}. Propuestos por J. Holland, se basan en los principios de selecci\'{o}n natural, usando operadores como la cruza, mutaci\'{o}n y selecci\'{o}n. Manteniendo una poblaci\'{o}n de las posibles soluciones iterando para encontrar la soluci\'{o}n cercana a la soluci\'{o}n \'{o}ptima.
%  \item \textbf{Particle Swarm Optimization (PSO)}. Propuestos por Eberhart y Kennedy, inspirado en el comportamiento de parvadas de p\'{a}jaros y cardumen de peces, el algoritmo involucra una poblaci\'{o}n de part\'{i}culas que se mueven en un espacio de b\'{u}squeda. Cada part\'{i}cula ajusta su posici\'{o}n seg\'{u}n su propia soluci\'{o}n y la soluci\'{o}n de toda la poblaci\'{o}n.
%  \item \textbf{Ant Colony Optimization (ACO)}. Propuesto por M. Dorigo, inspirado en el comportamiento de b\'{u}squeda de alimento de las hormigas, imita la comunicaci\'{o}n y toma de decisiones colectiva de las hormigas, puede ser usado para encontrar caminos dentro de un grafo. 
%  \item \textbf{Firefly Algorithm (FA)}. Propuesto por X. Yang, sigue el modelo de los patrones intermitentes de las luci\'{e}rnagas, el algoritmo emula el comportamiento de atracci\'{o}n y repulsi\'{o}n de las luci\'{e}rnagas.
%  \end{enumerate}

  Las metaheur\'{i}sticas han demostrado ser efectivas para resolver una amplia gama de problemas de optimizaci\'{o}n; sin embargo, su adopci\'{o}n en el campo de la rob\'{o}tica móvil se ve limitada por las restricciones de ejecución en tiempo real. La rob\'{o}tica a menudo implica la toma de decisiones en tiempo real, donde los robots deben responder r\'{a}pidamente a entornos cambiantes. Las metaheur\'{i}sticas suelen requerir extensos recursos computacionales y temporales para converger en una soluci\'{o}n \'{o}ptima, lo que puede no ser factible en aplicaciones de rob\'{o}tica en tiempo real, particularmente en vehículos aéreos con limitado poder de cómputo. El control y la planificaci\'{o}n en tiempo real en rob\'{o}tica a menudo requieren algoritmos de baja complejidad computacional, como la planificaci\'{o}n cl\'{a}sica o los enfoques de control reactivo.
  
  %\begin{itemize}
  %\item \textbf{Complejidad y restricciones en tiempo real:} la rob\'{o}tica a menudo implica la toma de decisiones en tiempo real, donde los robots deben responder r\'{a}pidamente a entornos cambiantes. Las metaheur\'{i}sticas suelen requerir extensos recursos computacionales y temporales para converger en una soluci\'{o}n \'{o}ptima, lo que puede no ser factible en aplicaciones de rob\'{o}tica en tiempo real, particularmente en vehículos aéreos con limitado poder de cómputo. El control y la planificaci\'{o}n en tiempo real en rob\'{o}tica a menudo requieren algoritmos de baja complejidad computacional, como la planificaci\'{o}n cl\'{a}sica o los enfoques de control reactivo.
    
  %\item \textbf{Soluciones rápidas:} en aplicaciones de rob\'{o}tica, especialmente las que involucran tareas cr\'{i}ticas para la seguridad o control preciso, se prefieren las soluciones deterministas y predecibles a las soluciones estoc\'{a}sticas que ofrecen las metaheur\'{i}sticas. Las metaheur\'{i}sticas brindan soluciones aproximadas con diversos grados de optimizaci\'{o}n, que pueden no ser adecuadas para tareas que requieren un control preciso o garant\'{i}as de seguridad.

%  \item \textbf{Optimizaci\'{o}n basada en modelos:} muchos problemas de rob\'{o}tica se pueden resolver de manera efectiva utilizando t\'{e}cnicas de optimizaci\'{o}n basadas en modelos. Con modelos din\'{a}micos conocidos y restricciones ambientales, los m\'{e}todos basados en modelos, como el control \'{o}ptimo o la optimizaci\'{o}n de la trayectoria, pueden proporcionar soluciones anal\'{i}ticas o num\'{e}ricas con un rendimiento garantizado. Estos enfoques pueden explotar la estructura del problema y las restricciones espec\'{i}ficas, lo que lleva a soluciones eficientes y confiables en comparaci\'{o}n con las metaheur\'{i}sticas de prop\'{o}sito general.

  %\item \textbf{Algoritmos de tareas específicas:} la rob\'{o}tica a menudo implica tareas y dominios espec\'{i}ficos que se han estudiado ampliamente, lo que da como resultado algoritmos espec\'{i}ficos de tareas adaptados a esos dominios. Estos enfoques personalizados a menudo son m\'{a}s eficientes y efectivos para resolver los problemas espec\'{i}ficos abordados en rob\'{o}tica, lo que hace que las metaheur\'{i}sticas de prop\'{o}sito general sean menos atractivas.
    
  %\item \textbf{Limitaciones de hardware y energ\'{i}a}: los sistemas de rob\'{o}tica suelen tener recursos de hardware limitados y, a menudo, est\'{a}n limitados por el consumo de energ\'{i}a. Las metaheur\'{i}sticas, que a menudo requieren una mayor cantidad de recursos o extensos tiempos de ejecución para alcanzar la convergencia, pueden no ser adecuadas para plataformas rob\'{o}ticas con recursos limitados.
  %\end{itemize}

Sin embargo, es importante tener en cuenta que ciertamente hay \'{a}reas dentro de la rob\'{o}tica donde las metaheur\'{i}sticas se han aplicado con \'{e}xito, como la planificaci\'{o}n de rutas de robots en entornos complejos, la rob\'{o}tica de enjambres o la asignaci\'{o}n de tareas en sistemas de m\'{u}ltiples robots. Los enfoques h\'{i}bridos que combinan metaheur\'{i}sticas con optimizaci\'{o}n basada en modelos o algoritmos espec\'{i}ficos de tareas pueden aprovechar las fortalezas de ambos y proporcionar soluciones efectivas para aplicaciones en la rob\'{o}tica.

En lugar de planificar trayectorias, los autores en \citeauthor{CIESLEWSKI2017}[\citenum{CIESLEWSKI2017}], emplean un enfoque reactivo que genera comandos de velocidad instantáneos hacia las fronteras descubiertas. \citeauthor{PAPACHRISTOS2017}[\citenum{PAPACHRISTOS2017}], \citeauthor{CINVES2021}[\citenum{CINVES2021}] construye un árbol de exploración rápido y aleatorio (RRT) con un costo relacionado al número de nuevos voxels para identificar el próximo objetivo, y un segundo RRT para trazar una ruta hacia el punto de vista seleccionado minimizando la incertidumbre en la posición y puntos de referencia del robot. Por su parte, \citeauthor{SELIN2019}[\citenum{SELIN2019}] introduce nodos con un alto potencial de ganancia de información en un RRT para utilizarlos como objetivos de planificación después de que el agente ha explorado su entorno cercano.

\citeauthor{OLEYNIKOVA2018}[\citenum{OLEYNIKOVA2018}] que también se ocupa del problema de exploración, incorpora un objetivo adicional de alcanzar una meta para abordar de manera explícita el problema de quedarse atrapado en mínimos locales. Eligen la próxima meta al seleccionarla con cierta probabilidad desde el objetivo global.

\citeauthor{MOHTA2017}[\citenum{MOHTA2017}], \citeauthor{GAO2018}[\citenum{GAO2018}] y \citeauthor{LIN2017}[\citenum{LIN2017}] emplean la información generada en la etapa de planificación para establecer un problema de optimización que produzca trayectorias seguras y dinámicamente viables. Todos buscan generar un corredor seguro para restringir la optimización. \citeauthor{MOHTA2017}[\citenum{MOHTA2017}], \citeauthor{LIN2017}[\citenum{LIN2017}], y \citeauthor{FLORENCE2018}[\citenum{FLORENCE2018}] emplean un algoritmo A* para planificar una trayectoria para buscar una distancia mínima hacia la siguiente frontera.

\cite{BUG2019}[\citenum{BUG2019}] presenta una soluci\'{o}n de navegaci\'{o}n para enjambres de peque\~{n}os multi-VANTS que exploran entornos desconocidos sin se\~{n}al de GPS de forma centralizada. \'{E}ste trabajo propone el algoritmo Swarm Gradient Bug (SGBA), que maximiza la cobertura al hacer que los robots se muevan en diferentes direcciones lejos del punto de partida. Los robots navegan por el entorno y enfrentan obst\'{a}culos est\'{a}ticos sobre la marcha mediante la odometr\'{i}a visual y algoritmos tipo BUG para el seguimiento de paredes. Adem\'{a}s, se comunican entre s\'{i} para evitar colisiones y maximizar la eficiencia de la b\'{u}squeda. Para regresar al punto de partida, los robots realizan una b\'{u}squeda de gradiente hacia una se\~{n}al Bluetooth de baja potencia.

%Se estudiaron los aspectos colectivos de SGBA, demostrando que permite que un grupo de cuadric\'{o}pteros comerciales est\'{a}ndar de 33 gramos explore con \'{e}xito un entorno del mundo real. El potencial de aplicaci\'{o}n se ilustra mediante una misi\'{o}n de b\'{u}squeda y rescate de prueba en la que los robots capturaron im\'{a}genes para encontrar v\'{i}ctimas en un entorno de oficina. Los algoritmos desarrollados se generalizan a otros tipos de robots y sientan las bases para abordar misiones igualmente complejas con enjambres de robots en el futuro.\cite{BUG2019}.\\

Independientemente del enfoque utilizado para generar la ruta, un aspecto crítico para la navegación libre de colisiones en entornos desconocidos es imponer restricciones en el plan de movimiento para navegar dentro del campo de visión actual hacia el siguiente punto de referencia, utilizada por los métodos en \citeauthor{RACER2022}[\citenum{RACER2022}] y \citeauthor{BARTOLOMEI2023}[\citenum{BARTOLOMEI2023}]

%En el trabajo de \citeauthor{MOHTA2017}[\citenum{MOHTA2017}] usan un planificador A* en un grafo h\'{i}brido con la informaci\'{o}n 3D y 2D, formulan un problema de programaci\'{o}n cuadrática para la generaci\'{o}n de trayectorias agregando un t\'{e}rmino en la funci\'{o}n de costo sobre el error entre la trayectoria y los segmentos de l\'{i}nea del camino. La trayectoria se representa como un polinomio de s\'{e}ptimo orden, utilizando un perfil de velocidad trapezoidal. \\

%\cite{LIN2017}[\citenum{LIN2017}] hacen uso de un planificador global offline para generar rutas, en la navegaci\'{o}n usan un planificador local seleccionando las nuevas rutas y un algoritmo A* para buscar la distancia m\'{i}nima hacia esas nuevas rutas. Utilizan un polinomio por partes de octavo orden para la representaci\'{o}n de la trayectoria.\\

%\cite{PAPACHRISTOS2017}[\citenum{PAPACHRISTOS2017}] presentan algoritmos para la exploraci\'{o}n aut\'{o}noma, construyendo un \'{a}rbol aleatorio de exploraci\'{o}n r\'{a}pida (RRT), buscando el camino que minimice la incertidumbre del robot con los puntos de referencia del mapa, mientras una segunda ejecuci\'{o}n del algoritmo RRT encuentra el camino hacia el punto de vista seleccionado minimizando la incertidumbre del robot y los puntos de referencia.\\

%\cite{OLEYNIKOVA2018}[\citenum{OLEYNIKOVA2018}] aborda el problema de m\'{i}nimos locales de la función de potencial empleada para la navegación, agregando objetivos secundarios para escapar de dichos mínimos.\\ %Los autores hacen uso de tablas hash que proporcionan una representaci\'{o}n del entorno con r\'{a}pidos tiempos de inserci\'{o}n y consulta de complejidad constante.\\

%Reescribir. No entendí.
%\cite{GAO2018}[\citenum{GAO2018}] propone un algoritmo para la generación de trayectorias utilizando marcha rápida y los polinomios en la base de Bernstein. Al considerar la dinámica del VANT, garantizan la prevención de colisiones manteniendo movimientos suaves y continuos.\\

%\cite{SELIN2019}[\citenum{SELIN2019}] presenta un enfoque para la planificación de una exploración autónoma en entornos tridimensionales a gran escala. Hacen uso del algoritmo RRT insertando valores altos a los vertices con mayores ganancias de informaci\'{o}n y que son usados como objetivos de planificaci\'{o}n de rutas. \\
%La planificación combina la estrategia basada en fronteras con el enfoque de ganancia de información para dirigir al robot a áreas inexploradas, logrando evitar obstáculos a su paso. Mediante un ajuste del plan de exploración en función del tamaño y complejidad del ambiente, muestran su eficiencia maximizando la cobertura de la exploración a medida que minimizan el tiempo de procesamiento.

%\cite{WESTHEIDER2023}\\

%\cite{BARTOLOMEI2023}\\

\subsection{Generación de trayectoria}
%\citeauthor{}[\citenum{}]

La propuesta de \citeauthor{GAO2018}[\citenum{GAO2018}] busca confinar toda la trayectoria dentro del espacio libre. Plantean un programa cuadrático (QP, por sus siglas en inglés) con restricciones, representando la trayectoria en forma de curvas de Bezier por tramos. \citeauthor{MOHTA2017}[\citenum{MOHTA2017}] formulan un QP para la generación de trayectorias donde, además de las restricciones habituales de velocidad, aceleración y jerk, agregan un término en la función de costo proporcional al cuadrado de la distancia entre la trayectoria y los segmentos de línea de la trayectoria modificada. Para asignar tiempo a cada segmento de spline, lo cual es crucial para la viabilidad del QP y la calidad de la trayectoria resultante, utilizan los tiempos obtenidos ajustando un perfil de velocidad trapezoidal a través de los segmentos.

\citeauthor{USENKO2017}[\citenum{USENKO2017}] plantean un problema de replanificación local como la optimización de una función de costo compuesta por un término que penaliza las desviaciones de posición y velocidad al final de la trayectoria, así como un costo por colisión. La trayectoria local se representa a través de un B-spline cúbico uniforme, lo cual simplifica el cálculo de los términos de costo. Por otro lado, \citeauthor{LIN2017}[\citenum{LIN2017}] formulan un problema de optimización no lineal utilizando polinomios de octavo orden para representar la trayectoria.

\citeauthor{CIESLEWSKI2017}[\citenum{CIESLEWSKI2017}] emplea un modo reactivo para generar comandos de velocidad instantáneos basados en las fronteras que se observan en el momento. Para aquellas fronteras dentro del alcance del sensor de profundidad, la velocidad deseada será la máxima y estará orientada hacia el volumen desconocido. En contraste, para las fronteras más cercanas al robot, la velocidad deseada será menor. Por otro lado, el algoritmo presentado por \citeauthor{FLORENCE2018}[\citenum{FLORENCE2018}], busca un movimiento primitivo 3D que maximice el progreso euclidiano hacia el objetivo global, teniendo en cuenta las probabilidades de colisión para trayectorias completas en entornos con obstáculos.

%Un aspecto importante de la navegación en el campo de visión es asegurar que el eje de la cámara esté alineado con la dirección del movimiento. Oleynikova, Taylor, Siegwart y Nieto (2018), y Cieslewski et al. (2017a) intentaron abordar esta restricción implementando un enfoque de seguimiento de velocidad en dirección de cabeceo, pero no garantizan que las trayectorias de cabeceo generadas cumplan con las restricciones dinámicas angulares.

Adicionalmente, es crucial abordar de manera explícita las restricciones dinámicas del movimiento lineal del robot para garantizar su permanencia en áreas seguras. En específico, los trabajos de \citeauthor{CIESLEWSKI2017}[\citenum{CIESLEWSKI2017}], \citeauthor{USENKO2017}[\citenum{USENKO2017}], \citeauthor{SELIN2019}[\citenum{SELIN2019}], \citeauthor{LIN2017}[\citenum{LIN2017}], \citeauthor{COLLINS2019}[\citenum{COLLINS2019}] y \citeauthor{BUG2019}[\citenum{BUG2019}] no gestionan las restricciones dinámicas de forma explícita. En contraste, los algoritmos propuestos en \citeauthor{FLORENCE2018}[\citenum{FLORENCE2018}], \citeauthor{GAO2018}[\citenum{GAO2018}], \citeauthor{OLEYNIKOVA2018}[\citenum{OLEYNIKOVA2018}], \citeauthor{PAPACHRISTOS2017}[\citenum{PAPACHRISTOS2017}], \citeauthor{MOHTA2017}[\citenum{MOHTA2017}], \citeauthor{CINVES2021}[\citenum{CINVES2021}], \citeauthor{RACER2022}[\citenum{RACER2022}] y \citeauthor{BARTOLOMEI2023}[\citenum{BARTOLOMEI2023}] que abordan de manera explícita las restricciones dinámicas del robot.

\subsection{Coordinación multi-robot}

Adem\'{a}s de la planificaci\'{o}n de rutas, la coordinaci\'{o}n de m\'{u}ltiples robots para la exploración requiere de una estrategia de comunicaci\'{o}n efectiva, que garantice la coherencia del mapa que se va generando. Se han investigado diferentes protocolos de comunicaci\'{o}n y estrategias de intercambio de informaci\'{o}n para permitir la colaboraci\'{o}n. Algunos enfoques utilizan comunicaci\'{o}n directa entre los robots, mientras que otros emplean una arquitectura de red donde los m\'{u}ltiples robots se comunican a trav\'{e}s de una infraestructura descentralizada \citeauthor{10120943}[\citenum{10120943}], \citeauthor{BARTOLOMEI2023}[\citenum{BARTOLOMEI2023}], \citeauthor{RACER2022}[\citenum{RACER2022}], \citeauthor{BARTOLOMEI2023}[\citenum{BARTOLOMEI2023}], mostrando la tolerancia a fallas en equipos para tareas de b\'{u}squeda y rescate.

En recientes trabajos \cite{CIESLEWSKI2021}[\citenum{CIESLEWSKI2021}] ha demostrado descentralizar la tarea de SLAM para la creaci\'{o}n de mapas en tareas de exploraci\'{o}n multi-VANT eliminando el bloque de optimizaci\'{o}n, haciendo uso de t\'{e}cnicas de machine learning (teach and repeat).

La dirección que apunta el estado del arte, es en la repartición inteligente de tareas para un problema multi-agente en tareas de exploración.

%La elecci\'{o}n del enfoque depende de las caracter\'{i}sticas de la aplicaci\'{o}n y las restricciones del sistema.\\

%La colaboraci\'{o}n de m\'{u}ltiples VANTS tambi\'{e}n puede implicar la formaci\'{o}n de formaciones o la realizaci\'{o}n de tareas coordinadas. Para ello, se han desarrollado algoritmos de control distribuido que permiten a los VANTS mantener posiciones relativas estables y realizar movimientos coordinados. 

%En t\'{e}rminos de validaci\'{o}n y evaluaci\'{o}n, se utilizan simulaciones y pruebas reales para verificar el rendimiento y la eficacia de los sistemas de colaboraci\'{o}n de m\'{u}ltiples VANTS. Las simulaciones permiten evaluar diferentes escenarios y ajustar los par\'{a}metros del sistema antes de las pruebas reales. Los casos de prueba reales proporcionan informaci\'{o}n sobre la implementaci\'{o}n y la eficiencia en situaciones del mundo real, y pueden ayudar a identificar desaf\'{i}os adicionales que deben abordarse.\\

%\noindent\hrulefill ((IDEAS INICIO)) \noindent\hrulefill\\
  
%\textcolor{red}{Estos problemas ya se han resulto en varios robots terrestres llegando a tener soluciones distribuidas o resultos los problemas de colisi\'{o}n, navegaci\'{o}n, mapeo y se han propuestos buenos algoritmos que formar\'{a}n parte de la arquitectura de software para resolver.}\\
 
%\noindent\hrulefill ((IDEAS FIN)) \noindent\hrulefill\\

%Multirobot\\
%Multi-robot exploration is a popular area of research in robotics, with applications in search and rescue, planetary exploration, and more. The papers present different algorithms and approaches to multi-robot exploration. Pandey 2012 presents an algorithm that takes into account communication constraints between robots and allocates target points to maximize the area explored while minimizing time and distance traveled. Pal 2011 proposes a modification to the A* algorithm for optimal path planning and target allocation strategy. Zlot 2002 presents an approach that uses a market architecture to maximize information gain while minimizing costs, which is reliable and robust to dynamic changes in team members and communication interruptions. Overall, the papers collectively suggest that multi-robot exploration is a complex problem that requires careful consideration of communication constraints, path planning, and target allocation strategies.\\

En el Centro de Investigaci\'{o}n y Estudios Avanzados del Instituto Polit\'{e}cnico Nacional Unidad Tamaulipas se han realizado investigaciones en el \'{a}rea de exploraci\'{o}n multi-robot y dise\~{n}o de prototipos de VANTS, lo cual sirve como antecedente para este trabajo. Este relevante desarrollo, propuesto por \citeauthor{CINVESTAM2013}[\citenum{CINVESTAM2013}], tiene como objetivo principal el despliegue de una estrategia de coordinaci\'{o}n para m\'{u}ltiples robots m\'{o}viles basado en un enfoque de auto-ofertas (método húngaro).

\newpage

\begin{landscape}
  \vspace{1cm}
  \begin{table*}[htbp]
    \centering
    \caption[Trabajos relacionados]{Trabajos relacionados}\label{tab:summary}
    \vspace{0.5cm}
    \scalebox{1.20}{
      \begin{tabular}{ | p{2.5cm} | p{1.5cm} | p{2.3cm} | p{3cm} | p{3cm} | p{0.8cm} | p{1cm} | p{0.7cm} | }
        \hline    
        \tiny REFERENCIA&
        \tiny APLICACIÓN&
        \tiny GENERACIÓN MAPA&
        \tiny PLANIFICACIÓN DE TRAYECTORIA&
        \tiny GENERACIÓN TRAYECTORIA&
        \tiny SENSOR RGB-D&
        \tiny DINÁMICA VANT&
        \tiny multi-VANT\\
        \hline
        %--------------------------
        %\tiny \cellcolor{gray!20}\cite{cinvestav2016}[\citenum{cinvestav2016}]&
        %\tiny \cellcolor{gray!20}Exploración&
        %\tiny \cellcolor{gray!20}Octomap&
        %\tiny \cellcolor{gray!20}Basado en fronteras&
        %\tiny \cellcolor{gray!20}Control directo de velocidad&
        %\tiny \cellcolor{gray!20}\ding{51} &
        %\tiny \cellcolor{gray!20}\ding{55} &
        %\tiny \cellcolor{gray!20}\ding{51} \\ \hline
        %--------------------------
        \tiny \cellcolor{gray!20}\citeauthor{CIESLEWSKI2017}[\citenum{CIESLEWSKI2017}]&
        \tiny \cellcolor{gray!20}Exploración&
        \tiny \cellcolor{gray!20}3D Grid&
        \tiny \cellcolor{gray!20}Basado en fronteras&
        \tiny \cellcolor{gray!20}Control directo de velocidad&
        \tiny \cellcolor{gray!20}\ding{51} &
        \tiny \cellcolor{gray!20}\ding{55} &
        \tiny \cellcolor{gray!20}\ding{55} \\ \hline
        %--------------------------
        \tiny \citeauthor{USENKO2017}[\citenum{USENKO2017}]&
        \tiny Punto Objetivo&
        \tiny Cuadr\'{i}cula egoc\'{e}ntrica Voxel 3D&
        \tiny Offline RRT*&
        \tiny Curvas de Bezier&
        \tiny \ding{51} &
        \tiny \ding{55} &
        \tiny \ding{55} \\ \hline
        %--------------------------
        \tiny \citeauthor{MOHTA2017}[\citenum{MOHTA2017}]&
        \tiny Punto Objetivo&
        \tiny 3D-Local y 2D-Global&
        \tiny A*&
        \tiny Programaci\'{o}n cuadr\'{a}tica&
        \tiny \ding{55} &
        \tiny \ding{51} &
        \tiny \ding{55} \\ \hline
        %--------------------------
        \tiny \citeauthor{LIN2017}[\citenum{LIN2017}]&
        \tiny Punto Objetivo&
        \tiny 3D voxel array TSDF&
        \tiny A*&
        \tiny Optimizaci\'{o}n cuadr\'{a}tica&
        \tiny \ding{55}&
        \tiny \ding{55} &
        \tiny \ding{55} \\ \hline
        %--------------------------
        \tiny \cellcolor{gray!20}\citeauthor{PAPACHRISTOS2017}[\citenum{PAPACHRISTOS2017}]&
        \tiny \cellcolor{gray!20}Exploración&
        \tiny \cellcolor{gray!20}Octomap&
        \tiny \cellcolor{gray!20}Next Best View Planner (NBVP)&
        \tiny \cellcolor{gray!20}Control directo de velocidad&
        \tiny \cellcolor{gray!20}\ding{55}&
        \tiny \cellcolor{gray!20}\ding{51}&
        \tiny \cellcolor{gray!20}\ding{55} \\ \hline
        %--------------------------
        \tiny \citeauthor{OLEYNIKOVA2018}[\citenum{OLEYNIKOVA2018}]&
        \tiny Punto Objetivo&
        \tiny Voxel Hashing TSDF&
        \tiny Next Best View Planner (NBVP)&
        \tiny Optimizaci\'{o}n cuadr\'{a}tica&
        \tiny \ding{51}&
        \tiny \ding{51}&
        \tiny \ding{55} \\ \hline
        %--------------------------
        \tiny \citeauthor{GAO2018}[\citenum{GAO2018}]&
        \tiny Punto Objetivo&
        \tiny Mapa de cuadr\'{i}cula&
        \tiny M\'{e}todo de marcha r\'{a}pida&
        \tiny Optimizaci\'{o}n cuadr\'{a}tica&
        \tiny \ding{55}&
        \tiny \ding{51}&
        \tiny \ding{55} \\ \hline
        %--------------------------
        \tiny \citeauthor{FLORENCE2018}[\citenum{FLORENCE2018}]&
        \tiny Punto Objetivo&
        \tiny Busqueda basada en visibilidad&
        \tiny 2D A*&
        \tiny Control predictivo por modelo (MPC)&
        \tiny \ding{51}&
        \tiny \ding{51}&
        \tiny \ding{55} \\ \hline
        %--------------------------
        \tiny \cellcolor{gray!20}\citeauthor{SELIN2019}[\citenum{SELIN2019}]&
        \tiny \cellcolor{gray!20}Exploración&
        \tiny \cellcolor{gray!20}Octomap&
        \tiny \cellcolor{gray!20}Next Best View Planner (NBVP)&
        \tiny \cellcolor{gray!20}Control directo de velocidad&
        \tiny \cellcolor{gray!20}\ding{55}&
        \tiny \cellcolor{gray!20}\ding{55}&
        \tiny \cellcolor{gray!20}\ding{55} \\ \hline
        %--------------------------
        \tiny \cellcolor{gray!20}\citeauthor{BUG2019}[\citenum{BUG2019}]&
        \tiny \cellcolor{gray!20}Exploración&
        \tiny \cellcolor{gray!20}NA&
        \tiny \cellcolor{gray!20}Swarm Gradient Bug Algorithm (SGBA)&
        \tiny \cellcolor{gray!20}Control directo de velocidad&
        \tiny \cellcolor{gray!20}\ding{55}&
        \tiny \cellcolor{gray!20}\ding{55}&
        \tiny \cellcolor{gray!20}\ding{51} \\ \hline
        %--------------------------
        \tiny \citeauthor{COLLINS2019}[\citenum{COLLINS2019}]&
        \tiny Punto Objetivo&
        \tiny KD Tree $+$ Mapa en Voxel&
        \tiny B\'{u}squeda en Grafo&
        \tiny Movimientos suaves&
        \tiny \ding{51}&
        \tiny \ding{55}&
        \tiny \ding{55} \\ \hline
        %--------------------------
        %\tiny \cite{Zhang2020}[\citenum{Zhang2020}]&
        %\tiny Punto Objetivo&
        %\tiny N/A&
        %\tiny Rapidly Exploring Random Trees (RRT)&
        %\tiny Movimientos suaves&
        %\tiny \ding{51}&
        %\tiny \ding{51}&
        %\tiny \ding{51} \\ \hline
        %--------------------------
        \tiny \citeauthor{CINVES2021}[\citenum{CINVES2021}]&
        \tiny Punto Objetivo&
        \tiny Octree&
        \tiny Rapidly Exploring Random Trees (RRT)&
        \tiny Basado en contornos&
        \tiny \ding{51}&
        \tiny \ding{51}&
        \tiny \ding{55} \\ \hline
        %--------------------------
        \tiny \cellcolor{gray!20}\citeauthor{CIESLEWSKI2021}[\citenum{CIESLEWSKI2021}]&
        \tiny \cellcolor{gray!20}Exploración&
        \tiny \cellcolor{gray!20}Octomap&
        \tiny \cellcolor{gray!20}Basado en fronteras&
        \tiny \cellcolor{gray!20}Control directo de velocidad&
        \tiny \cellcolor{gray!20}\ding{51}&
        \tiny \cellcolor{gray!20}\ding{51}&
        \tiny \cellcolor{gray!20}\ding{51} \\ \hline
        %--------------------------
        \tiny \cellcolor{gray!20}\citeauthor{RACER2022}[\citenum{RACER2022}]&
        \tiny \cellcolor{gray!20}Exploración&
        \tiny \cellcolor{gray!20}Octomap HGrid&
        \tiny \cellcolor{gray!20}Next Best View Planner (NBVP)&
        \tiny \cellcolor{gray!20}Control directo de velocidad&
        \tiny \cellcolor{gray!20}\ding{51}&
        \tiny \cellcolor{gray!20}\ding{51}&
        \tiny \cellcolor{gray!20}\ding{51} \\ \hline
        %--------------------------
        %\scriptsize \cite{WESTHEIDER2023}[\citenum{WESTHEIDER2023}]&
        %\scriptsize Mapa de cuadrícula&
        %\scriptsize Deep Reinforcement Learning&
        %\scriptsize Control directo de velocidad \\ \hline
        %--------------------------
        \tiny \cellcolor{gray!20}\citeauthor{BARTOLOMEI2023}[\citenum{BARTOLOMEI2023}]&
        \tiny \cellcolor{gray!20}Exploración&
        \tiny \cellcolor{gray!20}Octomap&
        \tiny \cellcolor{gray!20}Basado en fronteras&
        \tiny \cellcolor{gray!20}Control directo de velocidad&
        \tiny \cellcolor{gray!20}\ding{51}&
        \tiny \cellcolor{gray!20}\ding{51}&
        \tiny \cellcolor{gray!20}\ding{51} \\ \hline
        %--------------------------
      \end{tabular}
    }
  \end{table*}
\end{landscape}

%\section{Conclusiones}

%The motivation for this work relies on the lack of detailed knowledge about the search behavior of particle swarm optimization and differential evolution on multi-objective optimization. The ability to identify the mechanisms that impact on the search of these two MOEAs will allow for the design of new evolutionary approaches that use a reduced number of objective function evaluations to solve a selection of MOPs.


\chapter{Marco Teórico}

La exploración coordinada con múltiples vehículos aéreos no tripulados es un área de investigación actual en la robótica móvil, con aplicaciones en campos como la vigilancia, la exploración de áreas de difícil acceso y la respuesta a emergencias. Para comprender plenamente este tema, es crucial tener una sólida comprensión de los fundamentos teóricos que sustentan tanto la representación del movimiento de los vehículos aéreos no tripulados como los algoritmos para planificación de trayectorias y coordinación que facilitan su trabajo conjunto para alcanzar objetivos comunes mientras evitan colisiones y construyen una representación del medio ambiente.

%El marco teórico de esta tesis abarca desde los conceptos básicos de transformaciones homogéneas, que son fundamentales para describir la cinemática y la posición de los VANT en un espacio tridimensional, hasta los algoritmos de coordinación, que permiten a múltiples VANT colaborar de manera efectiva para alcanzar objetivos comunes mientras evitan colisiones y optimizan la eficiencia.

%Comenzaremos explorando las transformaciones homogéneas, que son herramientas matemáticas poderosas para representar la posición y orientación de los VANT en un sistema de coordenadas global. Estas transformaciones son esenciales para la planificación de trayectorias, la navegación y la coordinación de múltiples VANT en entornos complejos y dinámicos.

%A continuación, examinaremos los algoritmos de coordinación, que abarcan desde enfoques simples basados en reglas hasta métodos más avanzados basados en técnicas de optimización y aprendizaje automático. Estos algoritmos permiten a los VANT colaborar de manera inteligente, compartiendo información sobre el entorno, planificando rutas eficientes y ajustando su comportamiento en tiempo real para adaptarse a cambios en las condiciones del entorno.

Al comprender estos aspectos teóricos fundamentales, estaremos preparados para abordar los desafíos de la exploración coordinada con múltiples vehículos aéreos no tripulados.

%%%%%%%%%%%%%%%%%%%%%%%
%%%%%%%%%%%%%%%%%%%%%%%
%%%%%%%%%%%%%%%%%%%%%%%
%%%%%%%%%%%%%%%%%%%%%%%
%%%%%%%%%%%%%%%%%%%%%%%

\section{Conceptos fundamentales} \label{}

\subsection*{Exploración o localización y mapeo simultáneo}

Una diferencia clave entre el problema de localización y mapeo simultáneo (SLAM) y la exploración radica en sus objetivos y enfoques:

El \textbf{objetivo principal del SLAM} es estimar simultáneamente la trayectoria del robot (o vehículo) y construir un mapa del entorno desconocido en el que se encuentra.

El \textbf{objetivo principal de la exploración} es recorrer y mapear un área desconocida de manera eficiente y completa, sin necesariamente enfocarse en la localización precisa del robot en tiempo real.

Mientras el \textbf{enfoque del SLAM} se centra en la fusión de datos de sensores (como cámaras, LIDAR, etc.) con algoritmos de estimación de estado (como filtros de Kalman extendidos o métodos basados en grafos) para actualizar la estimación de la posición del robot y construir un mapa del entorno.

Por otra parte, el \textbf{enfoque de la exploración} se centra en la planificación de trayectorias y la toma de decisiones para guiar al robot a través del entorno de manera que se maximice la cobertura del área a explorar y se minimice el tiempo y los recursos necesarios.

%En resumen, mientras que el SLAM se enfoca en la estimación simultánea de la posición del robot y la construcción del mapa del entorno, la exploración se centra en recorrer y mapear un área desconocida de manera eficiente, sin necesidad de una localización precisa en tiempo real.

\subsection*{Sistema de ejes coordenados}

El sistema de ejes coordenados en tres dimensiones (3D) es una herramienta fundamental en la representación y comprensión del espacio tridimensional. Consiste en tres ejes: x, y, z, que se intersectan en un punto de origen común. Estos ejes proporcionan un marco de referencia para describir la posición y orientación de objetos en el espacio 3D, facilitando el análisis de información espacial. Un entendimiento de este sistema es importante para la planificación de trayectorias, la navegación, la representación del medio ambiente para el diseño de sistemas autónomos, incluyendo los vehículos aéreos no tripulados (VANTS).

\textit{\textbf{Desarrollar..}}

\subsection*{Coordenadas homogéneas}

Las coordenadas homogéneas son una herramienta matemática esencial en la representación y manipulación de la posición y orientación de vehículos aéreos no tripulados (VANTS) en un entorno tridimensional. Este enfoque permite expresar las transformaciones geométricas de manera compacta. Al utilizar coordenadas homogéneas, se pueden representar tanto las traslaciones como las rotaciones en un único marco de referencia, lo que simplifica los cálculos. Este enfoque es fundamental para el desarrollo de sistemas autónomos con habilidades de exploración coordinada, ya que proporciona una base matemática sólida para la comprensión y manipulación del movimiento de los vehículos aéreos no tripulados (VANTS).

\textit{\textbf{Desarrollar..}}

\subsection*{Transformaciones geométricas}

En el contexto de los vehículos aéreos no tripulados (VANTS), las transformaciones geométricas son herramientas fundamentales para comprender y manipular la posición y orientación de los vehículos en un espacio tridimensional. Estas transformaciones se utilizan para describir cambios en la posición, la rotación y el tamaño de los objetos en un sistema de coordenadas. Comprenden operaciones como la traslación y la rotación, que permiten modelar y simular el movimiento de los VANTS. 

\textit{\textbf{Desarrollar..}}

\section{Funcionamiento de un VANT}

Un vehículo aéreo no tripulado (VANT) es un robot móvil aéreo que gracias a su privilegiada perspectiva aérea han captado la atención a más de una persona.

Su funcionamiento se basa en una serie de componentes clave, que incluyen una plataforma de vuelo equipada con sistemas de propulsión, sensores, actuadores y una unidad de procesamiento central. Estos dispositivos pueden variar en tamaño, forma y capacidad.

%El VANT utiliza sus sensores para recopilar información del entorno, como datos de posición, imágenes, video y mediciones de distancia. Esta información se procesa a bordo o en tierra mediante algoritmos de percepción y planificación de rutas, permitiendo al VANT tomar decisiones autónomas sobre su navegación, evasión de obstáculos y cumplimiento de objetivos de misión.

El vuelo de un VANT se controla mediante comandos enviados desde una estación terrestre o mediante algoritmos de control autónomo a bordo. Estos comandos ajustan la velocidad, dirección y altitud del VANT, permitiéndole moverse de manera precisa y eficiente en el espacio aéreo.

El funcionamiento de un VANT se ve influenciado por una serie de factores, incluyendo las condiciones meteorológicas, las regulaciones gubernamentales, la autonomía de la batería y la calidad de los datos de los sensores. La seguridad y la fiabilidad son consideraciones críticas en el diseño y operación de estos sistemas, lo que requiere un enfoque cuidadoso en la planificación, desarrollo y prueba de cada componente.

\textit{\textbf{Desarrollar..}}

%En resumen, el funcionamiento de un VANT implica una integración compleja de hardware, software y sistemas de control para permitir operaciones autónomas en entornos aéreos variados. El entendimiento detallado de estos principios es fundamental para el diseño efectivo, la operación segura y el aprovechamiento máximo del potencial de los VANT en una amplia gama de aplicaciones.

\section{Control de un VANT}

El control es un aspecto crucial en la investigación y desarrollo de sistemas autónomos aéreos. Este proceso implica el diseño y la implementación de algoritmos y estrategias que permiten a los VANTs realizar tareas específicas de manera eficiente y segura. Se abordan diversos aspectos del control, como la estabilización, la navegación, la planificación de trayectorias y la interacción con el entorno. La investigación en este campo se centra en mejorar la precisión, y la capacidad de adaptación de los algoritmos de control, así como en la integración de tecnologías de la inteligencia artificial clásica. El control efectivo de los VANTs es fundamental para una amplia gama de aplicaciones.

\textit{\textbf{Desarrollar..}}

\section{Estimación de posición}

El proceso de estimación de posición en vehículos aéreos no tripulados (VANTs) es fundamental para su operación autónoma. Este proceso implica el uso de sensores y algoritmos para determinar la posición y orientación del VANT en relación con un sistema de coordenadas global. Los VANTs suelen utilizar una combinación de sensores, como IMU, cámaras y LIDAR, para recopilar información sobre su entorno y su movimiento. A partir de estos datos, se aplican algoritmos de estimación, como el filtro de Kalman extendido (EKF) o el filtro de partículas, para integrar la información de los sensores y calcular la posición estimada del VANT con respecto al tiempo.

\textit{\textbf{Desarrollar..}}

\subsection{Odometría visual inercial (Visual Inertial Odometry - VIO)}

Es una tecnología avanzada utilizada en vehículos aéreos no tripulados (VANTs) que combina datos visuales y de sensores inerciales para estimar la posición y orientación del vehículo en tiempo real. Este enfoque aprovecha la información visual de cámaras a bordo y la información de acelerómetros y giroscopios para calcular con precisión la trayectoria del VANT en entornos dinámicos. Al fusionar la información de múltiples fuentes sensoriales, la odometría visual inercial (VIO por sus siglas en inglés) supera las limitaciones individuales de cada sensor y proporciona una estimación confiable del movimiento del VANT.

\textit{\textbf{Desarrollar..}}

\section{Planificación de trayectoria}

El tema de la planificación de trayectorias es de vital importancia, abarcando un conjunto diverso de técnicas y enfoques para garantizar la navegación segura y eficiente de estos dispositivos en entornos complejos y dinámicos. Este proceso implica la generación de rutas que permitan a los VANTs alcanzar sus destinos mientras evitan obstáculos y cumplen con restricciones operativas, como minimizar el consumo de energía o maximizar la cobertura de área explorada. Desde métodos clásicos basados en grafos, hasta enfoques más avanzados que integran técnicas de aprendizaje automático y optimización.

\textit{\textbf{Desarrollar..}}

\subsection{Evasión de obstáculos}

El desarrollo de estrategias efectivas para la evasión de obstáculos es un componente crítico en el diseño de sistemas autónomos, especialmente en el contexto de los vehículos aéreos no tripulados (VANTs).

\textit{\textbf{Desarrollar..}}
%Además, se investigan estrategias de planificación de trayectorias y control de vuelo que permiten a los VANTs calcular rutas alternativas y tomar decisiones evasivas en función de la información del entorno.

\section{Representación medio ambiente}

%El mapeo del entorno es una tarea crítica en la operación de Vehículos Aéreos No Tripulados (VANTs), ya que proporciona una representación precisa y detallada del medio ambiente. Estos mapas son esenciales para la planificación de rutas, la navegación autónoma y la toma de decisiones inteligentes durante las misiones de exploración.

Estas representaciones espaciales permiten a los robots construir y mantener mapas detallados y precisos del entorno que los rodea, lo que les permite tomar decisiones inteligentes y seguras en tiempo real. Los Octomaps, que se basan en estructuras de datos de tipo octree, permiten una representación eficiente de la información de obstáculos y espacio libre en 3D, lo que facilita la planificación de trayectorias y la evasión de obstáculos. Por otro lado, los HGrids son una alternativa que utiliza una cuadrícula híbrida para representar el entorno, combinando una cuadrícula regular con una estructura de árbol para adaptarse a la complejidad del terreno. Ambos enfoques son ampliamente utilizados en robótica móvil y sistemas autónomos para mejorar la percepción del entorno y la navegación en entornos 3D, lo que resulta crucial en aplicaciones como la exploración de terrenos desconocidos.

%En la tesis se examinan diferentes enfoques y técnicas para la generación y actualización de mapas, que van desde métodos basados en sensores como cámaras y LIDAR hasta técnicas de fusión de datos y aprendizaje automático. Se analizan tanto las ventajas como las limitaciones de cada método en términos de precisión, eficiencia computacional y robustez en diferentes entornos y condiciones. Además, se exploran estrategias para la representación eficiente y la gestión de la incertidumbre en los mapas, con el objetivo de mejorar la fiabilidad y la utilidad de los datos para la planificación de misiones y la toma de decisiones. El estudio contribuye al avance en el campo de la robótica móvil y la exploración autónoma, ofreciendo nuevas perspectivas y soluciones innovadoras para el mapeo del entorno con VANTs.

\subsection{Ray casting}

El raycasting es una técnica fundamental en la percepción que permite detectar obstáculos y mapear el entorno tridimensional de manera eficiente. Consiste en emitir rayos desde el robot móvil en direcciones específicas y detectar las intersecciones con objetos o superficies en el entorno. Estas intersecciones proporcionan información crucial sobre la geometría del entorno, que se utiliza para generar mapas tridimensionales y planificar trayectorias seguras y eficientes para la navegación autónoma.

\textit{\textbf{Desarrollar..}}

\subsection{Point clouds}

El uso de vehículos aéreos no tripulados (VANTs) para la captura y procesamiento de nubes de puntos (point clouds) ha emergido como una herramienta poderosa en una amplia gama de aplicaciones, desde la cartografía y la topografía hasta la inspección de infraestructuras y la monitorización medioambiental. Las nubes de puntos, que consisten en conjuntos masivos de puntos tridimensionales que representan la superficie de objetos y entornos, proporcionan información detallada y precisa sobre la estructura y la geometría de los mismos.

\textit{\textbf{Desarrollar..}}

\subsection{Octomapas vs. HGrid}

\textit{\textbf{Desarrollar..}}

%En esta tesis, exploramos los avances en la adquisición y procesamiento de nubes de puntos utilizando VANTs, centrándonos en el desarrollo de algoritmos y técnicas para la planificación de trayectorias, la navegación y la captura eficiente de datos. Además, investigamos estrategias para la fusión y la interpretación de múltiples nubes de puntos obtenidas de diferentes perspectivas y momentos temporales, con el objetivo de generar modelos tridimensionales precisos y completos del entorno. Nuestro trabajo contribuye al avance en la capacidad de los VANTs para la recopilación de datos geoespaciales y la generación de información útil para una variedad de aplicaciones, impulsando así el potencial de esta tecnología en el ámbito de la percepción y la inteligencia espacial.

%\section{Exploración}

%El uso de Vehículos Aéreos No Tripulados (VANTs) para la exploración ha ganado prominencia en una variedad de campos, desde la vigilancia ambiental hasta la respuesta a desastres. Esta tesis se centra en el desarrollo y la implementación de técnicas avanzadas para la exploración coordinada con múltiples VANTs. Se investigan y se proponen algoritmos de coordinación eficientes que permiten a los VANTs colaborar de manera inteligente, compartiendo información del entorno y optimizando rutas para una exploración efectiva. Además, se aborda la planificación de trayectorias y la navegación autónoma para garantizar la eficacia y seguridad de las misiones de exploración. A través de experimentos simulados y pruebas de campo, se evalúan y validan las soluciones propuestas, con el objetivo de avanzar en el estado del arte en la exploración con VANTs y contribuir al desarrollo de sistemas autónomos más robustos y eficientes.

\section{Exploración multi-robot}

El concepto de exploración multi-robot con vehículos aéreos no tripulados (VANTs) es una área de investigación que aprovecha las capacidades colectivas de múltiples robots para explorar y mapear entornos desconocidos de manera eficiente y precisa. Esta estrategia ofrece ventajas significativas sobre la exploración con un solo robot, incluyendo una mayor cobertura del área, tiempos de exploración reducidos y una mayor robustez ante fallos individuales.

%En esta tesis, se investiga y se propone un conjunto de algoritmos y técnicas para la coordinación efectiva de múltiples VANTs en entornos complejos y dinámicos. Se explora el diseño de sistemas de planificación de trayectorias, la asignación de tareas, la comunicación entre robots y la toma de decisiones distribuida para lograr una exploración colaborativa y eficiente. Además, se llevan a cabo experimentos simulados y en entornos reales para validar y evaluar el desempeño de los algoritmos propuestos. Los resultados obtenidos demuestran el potencial de la exploración multirobot con VANTs para aplicaciones prácticas en campos como la vigilancia, la cartografía de áreas de difícil acceso y la respuesta a emergencias. Este trabajo contribuye al avance del conocimiento en el campo de la robótica móvil y sienta las bases para futuras investigaciones en el desarrollo de sistemas autónomos colaborativos para la exploración de entornos desconocidos.

\subsection{Asignación de tareas}

El desarrollo de sistemas multi-robot ha generado un interés creciente en la coordinación y asignación de tareas, especialmente en aplicaciones que requieren la colaboración de múltiples VANT para lograr objetivos comunes de la mejor manera posible.

%En esta tesis, se examina en detalle el diseño y la implementación de algoritmos de coordinación y asignación de tareas para equipos de VANT. Se abordan aspectos teóricos y prácticos relacionados con la planificación de rutas, la optimización de recursos, la comunicación entre VANT y la adaptación a cambios en el entorno. Se exploran diferentes enfoques, desde métodos basados en reglas simples hasta algoritmos más sofisticados que incorporan técnicas de aprendizaje automático y optimización. Además, se presentan estudios de caso y experimentos para validar la efectividad y la escalabilidad de los enfoques propuestos en una variedad de escenarios y aplicaciones prácticas. Esta investigación contribuye al avance del campo de la coordinación y asignación de tareas en sistemas de VANT, proporcionando herramientas y metodologías para mejorar la eficiencia y el rendimiento de los equipos de VANT en una amplia gama de aplicaciones, desde la vigilancia y la exploración hasta la logística y el transporte.

\section{Arquitectura de un robot}

La arquitectura de un robot integra diversos componentes y subsistemas que trabajan en conjunto para permitir al robot percibir su entorno, planificar y ejecutar acciones, y comunicarse con su entorno y otros sistemas, lo que le permite realizar tareas específicas de manera autónoma.

\begin{itemize}\setlength{\itemsep}{-1mm}
\item \textbf{Arquitectura de control:} La arquitectura de control consiste en la implementación de tres capas que permiten organizar los algoritmos en grupos específicos de acuerdo al tipo de acceso requerido. Las capas se listan a continuación.
  \begin{itemize}\setlength{\itemsep}{-1mm}
  \item Capa de planificación: Es la capa de más alto nivel, contiene algoritmos que no son de tiempo real y que pueden demandar altos recursos computacionales. Módulos de planificación de trayectoria, representación del medio ambiente son los incorporados en esta capa. Su función es evaluar los planes para alcanzar objetivos establecidos por la exploración.
  \item Capa de secuenciador: Es la capa mediadora entre la capa de planificación y la capa de habilidades. Activa los módulos solicitados por el planificador proporcionando los argumentos necesarios para cada habilidad y reporta a la capa superior eventos que pueden generar cambios en los planes establecidos.
  \item Capa de habilidades: Es la capa que ejecutará las instrucciones que modifican el estado del vehículo accediendo a su hardware, se conoce como capa de ejecución reactiva y se encarga de las respuestas en tiempo real de los eventos en el medio ambiente. En esta capa se localizan las habilidades reactivas como los módulos de evasión de obstáculos, seguir contrornos entre otros. 
  \end{itemize}

%\item \textbf{Comportamientos VANT:} La arquitectura integra diversos comportamientos que incrementan la funcionalidad del VANT en tareas de exploración.
%  \begin{itemize}\setlength{\itemsep}{-1mm}
%  \item Navegación hacia ciertos puntos.
%  \item Evitar colisones.
%  \item Construcción y mantenimiento de un mapa de ocupación 3D.
%  \end{itemize}
%\item \textbf{Coordinación entre múltiples VANTS:} La coordinación descentralizada se realizará siguiendo el enfoque de auto-ofertas presentada por \citeauthor{CINVESTAM2013}, ya que su portabilidad hace que pueda ejecutarse sin necesidad de un módulo central.
\end{itemize}

\section{El uso del sistema operativo robótico ROS}

El desarrollo de soluciones en robótica no es una tarea fácil, la escalabilidad y las capacidades de los robot son cada vez mayores, debido a la miniaturización de componentes que han permitido el desarrollo en robots con recursos limitados.

El uso de ROS (Robot Operating System) emerge como un componente esencial en el diseño y desarrollo de robótica, incluidos los vehículos aéreos no tripulados (VANTs). ROS proporciona una plataforma flexible y modular que facilita la integración de hardware, la implementación de algoritmos de control y la comunicación entre componentes en un entorno distribuido. Este sistema operativo robótico ofrece una amplia gama de herramientas y recursos que agilizan el desarrollo de aplicaciones, permitiendo concentrarse en la implementación de soluciones innovadoras en lugar de abordar problemas de infraestructura.

%En esta tesis, se destaca la importancia y los beneficios del uso de ROS en el diseño, implementación y evaluación de sistemas autónomos, subrayando su papel fundamental en la mejora de la interoperabilidad de estos diversos componentes que componen a un sistema autónomo.

El Sistema Operativo Robótico (ROS) se basa en programas modulares o procesos tratados como grafos de un nodo. Cada nodo en ejecución procesa información de forma paralela y la comparte con otros procesos con los que se comunica dentro del sistema.

A continuación se describen las principales características del Sistema Operativo Robótico (ROS):

\begin{itemize}\setlength{\itemsep}{-1mm}
\item Capaz de construir sistemas basados en múltiples procesos en diferentes equipos anfitriones conectados a través de una red, permitiendo realizar ejecuciones paralelas en distintas computadoras a bordo.
\item La codificación se puede llevar en diversos lenguajes de programación como: C/C++ y python.
\end{itemize}

Los principales conceptos en que ROS trabaja es bajo el paradigma publicador-suscriptor. A continuación se listan diversos tecnicismos usados en ROS.

\begin{itemize}\setlength{\itemsep}{-1mm}
\item \textbf{rosmaster}: Módulo de control, en el que cada nodo existente en el sistema debe registrarse, este módulo mantiene una lista de todos los nodos y tópicos en el sistema.
\item \textbf{Nodo}: Son módulos de programa que efectúan funciones específicas en la aplicación y comparten información a través de tópicos, servicios o por medio del servidor de parámetros. Su desarrollo es modular, cada nodo es un proceso independiente permitiendo que sea tolerante a fallos. La información generada por cada nodo puede ser publicada a través del intercambio de mensajes y otros nodos pueden suscribirse a ese tópico.
\item \textbf{Tópico}: Es el canal de comunicación usado para el intercambio de información entre los nodos. Un nodo recibirá los mensajes de un tópico previo a una suscripción al tópico.
\item \textbf{Mensaje}: Estructura de datos definida que puede contener tipos datos como enteros, flotantes, booleanos, puntos, entre otros.
\item \textbf{Servicios}: Responde a peticiones de un cliente, los servicios son contenidos dentro de un nodo (servicio) que responde con información de respuesta a la solicitud de otros nodos (clientes). 
\item \textbf{Paquetes}: Contiene los archivos de configuración, y los programas (nodos) para la funcionalidad del robot.
  
\end{itemize}

\textit{\textbf{Desarrollar..}}



\end{doublespace}

\appendix

\begin{doublespace}
\chapter{Publicaciones}
%Dar el formato que considere apropiado. Presentar, por ejemplo: Conferencias, Revistas internacionales, etc...
\begin{itemize}
\item Jorge Sebastian Hernández Domínguez and Gregorio Toscano Pulido. \emph{A Comparison on the Search of Particle Swarm Optimization and Differential Evolution on Multi-Objective Optimization}, in IEEE Congress on Evolutionary Computation (CEC 2011), New Orleans, LA, USA, June 2011.
\item 
Jorge Sebastian Hernández Domínguez, Gregorio Toscano Pulido, and Carlos A. Coello Coello, \emph{A Multi-objective Particle Swarm Optimizer Enhanced with a Differential Evolution Scheme}, in International Conference on Artificial Evolution (EA 2011), Angers, France, October 2011.

\end{itemize}



%Las publicaciones generadas por este trabajo de investigación deben publicarse en la sección de 'Anexos'

%\include{appendixA}

%\include{appendixB}

%\bibliographystyle{apalike}
\bibliography{EMOO,protocolo}
\end{doublespace}

\end{document}
