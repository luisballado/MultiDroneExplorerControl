\documentclass[24pt,aspectratio=169]{beamer}
\graphicspath{{Images/}{./}} % Specifies where to look for included images (trailing slash required)
\usetheme{metropolis}

\usepackage[spanish]{babel} % Cargar el paquete babel con la opción spanish
\usepackage[utf8]{inputenc} % Asegurar la codificación UTF-8
\usepackage{appendixnumberbeamer}
\usepackage{booktabs}
\usepackage[scale=2]{ccicons}
\usepackage{pgfplots}
\usepackage{xcolor} % Cargar el paquete xcolor
\usepackage{xspace}
\usepackage[round]{natbib}
\usepackage{ragged2e}
\usepackage{bbding} %palomitas checkmark
\usepgfplotslibrary{dateplot}

\usepackage[ruled,vlined]{algorithm2e}
\usepackage{dirtree}
\usepackage{xcolor}

\usepackage{rotating}
\usepackage{tikz}

\usepackage{array} % needed for \arraybackslash
\usepackage{graphicx}
\usepackage{adjustbox} % for \adjincludegraphics

\usepackage{subcaption}
\usepackage{bibentry}
%\bibliographystyle{apalike}
\usepackage{chngcntr}
\usepackage{lipsum}% http://ctan.org/pkg/lipsum
\usepackage{hanging}% http://ctan.org/pkg/hanging

\usepackage{xcolor,colortbl}
\usepackage{multirow}

\usepackage{animate}
\usepackage{multicol}
\usepackage{tabularx,booktabs}
\usepackage{forloop}
\usepackage{ragged2e}

\usepackage{bbding} %palomitas checkmark
\usepackage{pifont}
\usepackage{lipsum,tabularx}

\usepackage{tikz}
\usetikzlibrary{arrows.meta, positioning, shapes.multipart}

\usepackage{tikz}
\usetikzlibrary{shapes,arrows}
\usepackage{amsmath,bm,times}
\newcommand{\mx}[1]{\mathbf{\bm{#1}}} % Matrix command
\newcommand{\vc}[1]{\mathbf{\bm{#1}}} % Vector command

%\usepackage{annotate-equations}
\newcounter{loopcntr}

\definecolor{verde_cinves}{HTML}{009383} % Color de primer plano (foreground) de la barra de progreso
\definecolor{azulrey_cinves}{HTML}{005179} % Color de primer plano (foreground) de la barra de progreso
\definecolor{amarillo_cinves}{HTML}{FFDA72} % Color de primer plano (foreground) de la barra de progreso

\newcommand{\themename}{\textbf{\textsc{metropolis}}\xspace}

%separador titulo inicio
\setbeamercolor{title separator}{use=structure,fg=azulrey_cinves!50, bg=azulrey_cinves!10}

%barra que anuncia el nombre de la seccion
\setbeamercolor{progress bar in section page}{use=structure,fg=azulrey_cinves!100, bg=azulrey_cinves!10}

%barra de titulo en cada section page
\setbeamercolor{frametitle}{use=structure,fg=azulrey_cinves!5, bg=azulrey_cinves!100}

\title{\large{Estrategias para la exploración coordinada multi-VANT}}
%\subtitle{A modern beamer theme}
\author{Tesista: Luis Alberto Ballado Aradias\\[\baselineskip]
  \small{{Asesores:} 
    \and\\Dr. José Gabriel Ramírez-Torres
    \and\\Dr. Eduardo Rodriguez-Tello\\ }}
\date{\today}
\institute{Centro de Investigación y de Estudios Avanzados - Unidad Tamaulipas}
\titlegraphic{\hfill\includegraphics[height=1.5cm]{cinvestav_negro}}

\newcommand{\on}[1][1]{
  \forloop{loopcntr}{0}{\value{loopcntr}<#1}{&\cellcolor{lightgray}}
}
\newcommand{\onok}[1][1]{
  \forloop{loopcntr}{0}{\value{loopcntr}<#1}{&\cellcolor{green}}
}
\newcommand{\ondelay}[1][1]{
  \forloop{loopcntr}{0}{\value{loopcntr}<#1}{&\cellcolor{red}}
}
\newcommand{\off}[1][1]{
  \forloop{loopcntr}{0}{\value{loopcntr}<#1}{&\cellcolor{white}}
}

\addtolength{\textheight}{90pt}

\newcommand{\I}{\mathbb{I}}
\newcommand{\K}{\mathbb{K}}
\newcommand{\N}{\mathbb{N}}
\newcommand{\Q}{\mathbb{Q}}
\newcommand{\R}{\mathbb{R}}
\newcommand{\Z}{\mathbb{Z}}

\newcommand{\specialcell}[2][c]{%
  \begin{tabular}[#1]{@{}c@{}}#2\end{tabular}}

\begin{document}

\maketitle

\begin{frame}{Contenido}
  \setbeamertemplate{section in toc}[sections numbered]
  \tableofcontents[hideallsubsections]
\end{frame}

\section{Introducción}

\begin{frame}[fragile]{Antecedentes}
  \bigskip % Vertical whitespace
  \centering
  \includegraphics[width=0.45\textwidth,height=0.35\textheight]{DJI_B1}$^\dag$
  \hfil
  \includegraphics[width=0.45\textwidth,height=0.35\textheight]{survey_chart.jpg}\footnotemark
  \vspace{2pt}\\
  
  \begin{itemize}
    \item \textbf{UAV} {\tiny(\textit{Vehículo Aéreo No Tripulado - VANT})}  $\implies$ \textbf{UAS} {\tiny(\textit{Sistemas Aéreos No Tripulado - SANT})}
  \item \textbf{Aplicaciones} en lugares inaccesibles o peligrosos.
  \item \textbf{Múltiples VANT} pueden aumentar la confianza del sistema.
  \item \textbf{Limitaciones} en carga, procesamiento y batería influyen en el tiempo de vuelo.
  \end{itemize}

  \tiny{
  \footnotetext{UAV in the advent of the twenties: Where we stand and what is next [\cite{Nex2022}]}}
\end{frame}

\begin{frame}[fragile]{Antecedentes}
  
  %\centering
  Principales preguntas que un robot autónomo debe responder \footnotemark\\
  \begin{itemize}
  \item ¿Dónde estoy? $\implies$ Localización 
  \item ¿A dónde voy? $\implies$ Cognición
  \item ¿Cómo llego hasta ahí? $\implies$ Planificación de trayectoria
  \end{itemize}
  \pause
  %\bigskip % Vertical whitespace
  Para resolver estas preguntas, el robot debe:\\
  \begin{itemize}
  \item Tener un modelo del ambiente (dado, o autónomamente construido)
  \item Localizarse dentro del ambiente
  \item Planear y ejecutar los movimientos
  \end{itemize}

  \footnotetext{Visual map making for a mobile robot [\cite{1087348}]}
  
\end{frame}

\begin{frame}{Antecedentes}
  
  %Exploración es una tarea fundamental en robots autónomos.\\
  %El objetivo es crear un mapa de un ambiente desconocido.\\
  %\bigskip % Vertical whitespace
  \centering
  \bigskip % Vertical whitespace
  \includegraphics[width=9cm]{exploracion}\\
  
  \begin{itemize}
  \item \textbf{Sensar}
  \item \textbf{Creación Mapa} 
  \item \textbf{Localización en Mapa}
  \item \textbf{\textcolor{blue}{Exploración}} (Aumentar la base de conocimiento del mapa)
  \item \textbf{\textcolor{blue}{Planificación trayectoria}} (Trayectorias hacia nuevas fronteras) 
  \item \textbf{Control} (Toma de decisiones y ejecución de trayectorias)
  \end{itemize}
  
  \alert{- El ciclo se repite hasta completar la exploración -}

\end{frame}

\begin{frame}[fragile]{Motivación del proyecto}
  %https://ccc.inaoep.mx/~emorales/Papers/2009/eduardo.pdf
  %https://www.mdpi.com/2504-446X/7/1/62
  %Hablar de los usos e importancia del trabajo .. de la exploracion .. de la importancia de investigacion en robotica para el pais.
  %El potencial del uso de UAVs en tareas de búsqueda y rescate, inspección, mapeo, vigilancia, entre otras, es de gran interés a explorar, debido a las habilidades de vuelo que presentan en favor de la realización de estas tareas, y en especial situaciones que podrían poner en riesgo a personas.\\
  %Enviar personal de rescate dentro de un edificio parcialmente colapsado por un terremoto en busca de sobrevivientes, es poner a más personas en un gran riesgo, pues no se sabe qué es lo que les espera en el interior del edificio; esto limita la capacidad de tomar buenas decisiones acerca de si es seguro seguir cierto camino.
  \begin{figure}[ht!]
    \centering
    \begin{minipage}{0.48\textwidth}
      \centering
      \includegraphics[width=\linewidth,height=3.5cm]{turquia1.jpg} % BUSQUEDA Y RESCATE
      %\caption{Primera Imagen}
    \end{minipage}\hfill
    \begin{minipage}{0.48\textwidth}
      \centering
      \includegraphics[width=\linewidth,height=3.5cm]{drone_agriculture.jpg} % INSPECCION Y SEGURIDAD
      %\caption{Segunda Imagen}
    \end{minipage}
    \vspace{-0.2cm} % Espacio vertical entre imágenes
    \begin{minipage}{0.48\textwidth}
      \centering
      \includegraphics[width=\linewidth,height=3.5cm]{pipe_drone.jpg} % DRONES EN CIUDADES CON GPS DENY
      %\caption{Tercera Imagen}
    \end{minipage}\hfill
    \begin{minipage}{0.48\textwidth}
      \centering
      \includegraphics[width=\linewidth,height=3.5cm]{drone_city.jpg} % 
      %\caption{Cuarta Imagen}
    \end{minipage}
  \end{figure}
\end{frame}

\section{Planteamiento del problema}

\begin{frame}[fragile]{Planteamiento del problema}
  \vspace{1mm}
  \centering
  \includegraphics[width=9.5cm]{problema}\\
  \justifying
  \footnotesize Dado un volumen de interés desconocido en un espacio cerrado que se desea explorar denotado como $\mathcal{W}$, tal que $\mathcal{W} \subset \mathbb{R}^{3}$.\\
%el volumen se discretiza en un mapa de ocupación M consistente en voxeles cúbicos $m \exists M$ con una longitud de arista r.
  \begin{itemize}
  \item El volumen se discretiza usando unidades cúbicas tridimencionales (voxel) $v_{libre}$, $v_{ocup}$, $v_{desc}$.
    %La representación del volumen a explorar se obtiene dividiendo el volumen de interés en unidades cúbicas tridimencionales (voxel) que puede tomar los valores de libre $v_{libre}$, ocupado $v_{ocup}$ y desconocido $v_{desc}$ con lecturas a partir de los valores de una cámara RGB-D basada en un modelo de ocupación probabilístico.\\
  \item Un conjunto de VANTS con una cámara RGB-D embarcadas denotado como $\mathcal{V} = \{\mathcal{V}_{1},\mathcal{V}_{2},\mathcal{V}_{3},...,\mathcal{V}_{n}\}$, comenzando cada uno en un estado inicial conocido $q = \{q_{1},q_{2},q_{3},...,q_{n}\}$, y terminando en una configuración que maximice la construcción de un mapa.\\
  \item \textcolor{blue}{Coordinar el conjunto de VANTs para reducir el tiempo total de exploración.}
  \end{itemize}
      
  %Crear una Coordinar un conjunto de vehiculos aereos no tripulados
  %Explorar un volumen desconocido tal que $V \subset R^{3}$ con un conjunto de robots aéreos autónomos $R ={R_1,R_2,...,R_n}$, siendo n la cardinalidad del conjunto de robots aéreos. Cada robot cuenta con capacidades de generar una representación del medio ambiente, localizarse y generar trayectorias que minimicen la incertidumbre de la localización y el mapeo, evaluada a través de una métrica sobre la creencia probabilística de la pose del robot y la posición de los puntos de referencia. El problema de exploración puede formularse globalmente como el de partir de una configuración inicial libre de colisiones y derivar puntos de vista que cubran el volumen previamente desconocido, determinando qué partes del espacio inicialmente inexplorado están libres o ocupadas. Para este proceso, el volumen se discretiza en un mapa de ocupación M consistente en voxeles cúbicos $m \exists M$ con una longitud de arista r. La operación está sujeta a las restricciones dinámicas del vehículo y a las limitaciones del sistema de sensores. Dado que las capacidades perceptivas de la mayoría de los sensores terminan en las superficies, los espacios huecos o bolsillos estrechos a veces pueden permanecer inexplorados, lo que lleva a un volumen residual.
\end{frame}

\begin{frame}{Hipótesis y preguntas de investigación}
  
  \begin{block}{Hipótesis}
    \vspace{1mm}
    \small Una estrategia que coordine y asigne tareas de exploración para múltiples VANTS de manera descentralizada, en combinación con una arquitectura de software diseñada para resolver problemas de localización, gestión de mapas y planificación de rutas, mejorará la eficiencia y cobertura de la exploración en interiores de un entorno desconocido.
  \end{block}
  \pause
  %\textcolor{blue}{Preguntas de investigación}
  \begin{block}{Preguntas de investigación}
    \small{
      \begin{enumerate}
      \item ¿Qué características de la dinámica del VANT son cruciales para lograr trayectorias suaves y continuas?
      \item ¿Podría un planificador de trayectorias que aproveche las regiones libres de obstáculos acelerar los desplazamientos de los VANTs y, consecuentemente, reducir los tiempos de exploración?
      \item ¿Qué mecanismos de coordinación existen dentro de la literatura que podrían ayudar en resolver el problema de exploración multi-VANT?
      \end{enumerate}
    }
  \end{block}
\end{frame}

\begin{frame}{Objetivos}
  
  \begin{block}{Objetivo General}
    %Desarrollar una estrategia de exploración descentralizada que permita resolver los problemas de coordinación para múltiples VANTS en ambientes desconocidos.
    %Desarrollar una estrategia de exploración descentralizada que integre técnicas de percepción y algoritmos de decisión para mejorar la autonomía de múltiples VANTs en la exploración de interiores.
    \vspace{2mm}
    Desarrollar una estrategia de exploración descentralizada que permita resolver los problemas de coordinación para múltiples VANTS en ambientes desconocidos.
  \end{block}
  %\pause
  \bigskip
  \begin{block}{Objetivos Particulares}
    \vspace{2mm}
    \begin{enumerate}
    \item Desarrollar una arquitectura de software que resuelva los problemas de autonomía para un VANT (localización, manejo de mapas y planificación de trayectorias).
    \item Implementar un mecanismo de coordinación descentralizado que asigne tareas de exploración.
    \item Realizar pruebas y simulaciones de la solución propuesta en diversos entornos, analizando la relación tiempo de exploración y cobertura del área de interés.
    \end{enumerate}
  \end{block}
\end{frame}

\begin{frame}{Estado del arte}
  \centering
  \includegraphics[width=10cm, height=6cm]{soa}
\end{frame}


\section{Plan de cierre del trabajo}

\begin{frame}{Graficos}
  \tiny{

    \begin{tikzpicture}[
  node distance=1.5cm and 2cm,
  every node/.style={rectangle, draw, rounded corners, align=center, minimum height=1cm},
  process/.style={rectangle, draw, rounded corners, align=center, minimum height=1cm},
  input/.style={rectangle, draw, rounded corners, align=center, minimum height=1cm, minimum width=2cm},
  arrow/.style={-{Latex[scale=1.5]}}
]

% Input nodes
\node[input] (depth) {Depth};
\node[input, below=1cm of depth] (odometry) {Odometry};

% Process nodes
\node[process, right=3cm of depth] (map) {Map};
\node[process, right=2cm of map] (frontiers_extraction) {Frontiers Extraction};
\node[process, below=1cm of frontiers_extraction] (frontiers_classification) {Frontiers Classification};

% Arrows
\draw[arrow] (depth) -- (map);
\draw[arrow] (odometry) -- (map);
\draw[arrow] (map) -- (frontiers_extraction);
\draw[arrow] (frontiers_extraction) -- (frontiers_classification);

% Group box boundaries
\coordinate (box-nw) at ([xshift=-1cm, yshift=1cm] depth.north west);
\coordinate (box-se) at ([xshift=1cm, yshift=-1cm] frontiers_classification.south east);

% Group box
\draw[dashed, rounded corners] (box-nw) rectangle (box-se);
\node[anchor=north west, font=\bfseries] at (box-nw) {Mapping};

\end{tikzpicture}
}
\end{frame}


  % We need layers to draw the block diagram
\pgfdeclarelayer{background}
\pgfdeclarelayer{foreground}
\pgfsetlayers{background,main,foreground}

% Define a few styles and constants
\tikzstyle{sensor}=[draw, fill=blue!20, text width=5em, 
    text centered, minimum height=2.5em]
\tikzstyle{ann} = [above, text width=5em]
\tikzstyle{naveqs} = [sensor, text width=6em, fill=red!20, 
    minimum height=12em, rounded corners]
\def\blockdist{2.3}
\def\edgedist{2.5}
\begin{frame}{Ejemplo}
\begin{tikzpicture}
    \node (naveq) [naveqs] {Navigation equations};
    % Note the use of \path instead of \node at ... below. 
    \path (naveq.140)+(-\blockdist,0) node (gyros) [sensor] {Gyros};
    \path (naveq.-150)+(-\blockdist,0) node (accel) [sensor] {Accelero-meters};
    
    % Unfortunately we cant use the convenient \path (fromnode) -- (tonode) 
    % syntax here. This is because TikZ draws the path from the node centers
    % and clip the path at the node boundaries. We want horizontal lines, but
    % the sensor and naveq blocks aren't aligned horizontally. Instead we use
    % the line intersection syntax |- to calculate the correct coordinate
    \path [draw, ->] (gyros) -- node [above] {$\vc{\omega}_{ib}^b$} 
        (naveq.west |- gyros) ;
    % We could simply have written (gyros) .. (naveq.140). However, it's
    % best to avoid hard coding coordinates
    \path [draw, ->] (accel) -- node [above] {$\vc{f}^b$} 
        (naveq.west |- accel);
    \node (IMU) [below of=accel] {IMU};
    \path (naveq.south west)+(-0.6,-0.4) node (INS) {INS};
    \draw [->] (naveq.50) -- node [ann] {Velocity } + (\edgedist,0) 
        node[right] {$\vc{v}^l$};
    \draw [->] (naveq.20) -- node [ann] {Attitude} + (\edgedist,0) 
        node[right] { $\mx{R}_l^b$};
    \draw [->] (naveq.-25) -- node [ann] {Horisontal position} + (\edgedist,0)
        node [right] {$\mx{R}_e^l$};
    \draw [->] (naveq.-50) -- node [ann] {Depth} + (\edgedist,0) 
        node[right] {$z$};
    
    % Now it's time to draw the colored IMU and INS rectangles.
    % To draw them behind the blocks we use pgf layers. This way we  
    % can use the above block coordinates to place the backgrounds   
    \begin{pgfonlayer}{background}
        % Compute a few helper coordinates
        \path (gyros.west |- naveq.north)+(-0.5,0.3) node (a) {};
        \path (INS.south -| naveq.east)+(+0.3,-0.2) node (b) {};
        \path[fill=yellow!20,rounded corners, draw=black!50, dashed]
            (a) rectangle (b);
        \path (gyros.north west)+(-0.2,0.2) node (a) {};
        \path (IMU.south -| gyros.east)+(+0.2,-0.2) node (b) {};
        \path[fill=blue!10,rounded corners, draw=black!50, dashed]
            (a) rectangle (b);
    \end{pgfonlayer}
\end{tikzpicture}
\end{frame}

\begin{frame}{Plan de cierre del trabajo}
\tiny{
  \begin{tikzpicture}[
  every node/.style={rectangle, draw, align=center, minimum width=3cm, minimum height=1cm, text width=3cm},
  concept/.style={rounded corners, fill=blue!20},
  subconcept/.style={rounded corners, fill=red!20},
  arrow/.style={-{Latex[scale=1.0]}, thick}
]

% Nodes
    \node[concept] (history) {Historia de la Robótica};

    \node[concept, below left=2cm and 3cm of history] (antiquity) {Antigüedad \\ \tiny{3000 AC - 1er siglo}};
\node[concept, below=of antiquity] (middle_ages) {Edad Media \\ \tiny{Siglo XII - XVI}};
\node[concept, below=of middle_ages] (modern_times) {Edad Moderna \\ \tiny{Siglo XVII - XIX}};
\node[concept, below=of modern_times] (20th_century) {Siglo XX};
\node[concept, below=of 20th_century] (21st_century) {Siglo XXI};

\node[subconcept, below right=1.5cm and 1cm of antiquity] (antikythera) {Mecanismo de Antikythera \\ \tiny{~100 AC}};
\node[subconcept, below right=1.5cm and 1cm of middle_ages] (al_jazari) {Automata de Al-Jazari \\ \tiny{1206}};
\node[subconcept, below right=1.5cm and 1cm of modern_times] (jacquard) {Telar de Jacquard \\ \tiny{1801}};
\node[subconcept, below right=1.5cm and 1cm of 20th_century] (turing) {Test de Turing \\ \tiny{1950}};
\node[subconcept, below right=1.5cm and 1cm of 21st_century] (sophia) {Sophia el robot \\ \tiny{2016}};

% Arrows
\draw[arrow] (history) -- (antiquity);
\draw[arrow] (history) -- (middle_ages);
\draw[arrow] (history) -- (modern_times);
\draw[arrow] (history) -- (20th_century);
\draw[arrow] (history) -- (21st_century);

\draw[arrow] (antiquity) -- (antikythera);
\draw[arrow] (middle_ages) -- (al_jazari);
\draw[arrow] (modern_times) -- (jacquard);
\draw[arrow] (20th_century) -- (turing);
\draw[arrow] (21st_century) -- (sophia);

\end{tikzpicture}
  }
\end{frame}


\section{Conclusión}
\begin{frame}{Conclusiones }
\tiny{
  \begin{tikzpicture}[
  node distance=0.5cm and 0.5cm,
  every node/.style={rectangle, draw, align=center, minimum width=1.5cm},
  milestone/.style={rounded corners, fill=blue!20},
  event/.style={fill=red!20},
  arrow/.style={-{Latex[scale=1.5]}}
]

% Nodes
\node[milestone] (antiquity) {Antiquity \\ \tiny{3000 BC - 1st Century}};
\node[milestone, right=of antiquity] (middle_ages) {Middle Ages \\ \tiny{12th - 16th Century}};
\node[milestone, right=of middle_ages] (modern_times) {Modern Times \\ \tiny{17th - 19th Century}};
\node[milestone, right=of modern_times] (20th_century) {20th Century};
\node[milestone, right=of 20th_century] (21st_century) {21st Century};

\node[event, below=of antiquity] (antikythera) {Antikythera mechanism \\ \tiny{~100 BC}};
\node[event, below=of middle_ages] (al_jazari) {Al-Jazari's automata \\ \tiny{1206}};
\node[event, below=of modern_times] (jacquard) {Jacquard loom \\ \tiny{1801}};
\node[event, below=of 20th_century] (turing) {Turing Test \\ \tiny{1950}};
\node[event, below=of 21st_century] (sophia) {Sophia the robot \\ \tiny{2016}};

% Arrows
\draw[arrow] (antiquity) -- (middle_ages);
\draw[arrow] (middle_ages) -- (modern_times);
\draw[arrow] (modern_times) -- (20th_century);
\draw[arrow] (20th_century) -- (21st_century);

\draw[arrow] (antiquity) -- (antikythera);
\draw[arrow] (middle_ages) -- (al_jazari);
\draw[arrow] (modern_times) -- (jacquard);
\draw[arrow] (20th_century) -- (turing);
\draw[arrow] (21st_century) -- (sophia);

\end{tikzpicture}
}
\end{frame}

%\begin{frame}[standout]
%  ¿Preguntas?
%\end{frame}

\appendix

\begin{frame}[fragile]{Apéndice}
  Sometimes, it is useful to add slides at the end of your presentation to
  refer to during audience questions.

  The best way to do this is to include the \verb|appendixnumberbeamer|
  package in your preamble and call \verb|\appendix| before your backup slides.

  \themename will automatically turn off slide numbering and progress bars for
  slides in the appendix.
\end{frame}

\begin{frame}[fragile]{Estado del arte}
  \centering
  \scalebox{0.70}{
  \begin{tabular}{ | p{3cm} | p{1.6cm} | p{2.5cm} | p{3cm} | p{3.1cm} | p{0.8cm} | p{0.9cm} | }
    \hline    
    \tiny REFERENCIA&
    \tiny APLICACIÓN&
    \tiny GENERACION MAPA&
    \tiny PLANIFICACION DE RUTA&
    \tiny GENERACION TRAYECTORIA&
    \tiny SENSOR RGB-D&
    \tiny DINAMICA VANT\\
    \hline
    %--------------------------
    \tiny \cellcolor{gray!20}\cite{CIESLEWSKI2017}[\citenum{CIESLEWSKI2017}]&
    \tiny \cellcolor{gray!20}Exploración&
    \tiny \cellcolor{gray!20}Octomap&
    \tiny \cellcolor{gray!20}Basado en fronteras&
    \tiny \cellcolor{gray!20}Control directo de velocidad&
    \tiny \cellcolor{gray!20}\ding{51} &
    \tiny \cellcolor{gray!20}\ding{55} \\ \hline
    %--------------------------
    \tiny \cite{USENKO2017}[\citenum{USENKO2017}]&
    \tiny Punto Objetivo&
    \tiny Cuadr\'{i}cula egoc\'{e}ntrica&
    \tiny Offline RRT*&
    \tiny Curvas de Bezier&
    \tiny \ding{55} &
    \tiny \ding{51} \\ \hline
    %--------------------------
    \tiny \cite{MOHTA2017}[\citenum{MOHTA2017}]&
    \tiny Punto Objetivo&
    \tiny mapa 3D-Local y 2D-Global&
    \tiny A*&
    \tiny Programaci\'{o}n cuadr\'{a}tica&
    \tiny \ding{55} &
    \tiny \ding{51} \\ \hline
    %--------------------------
    \tiny \cite{LIN2017}[\citenum{LIN2017}]&
    \tiny Punto Objetivo&
    \tiny 3D voxel array TSDF&
    \tiny A*&
    \tiny Optimizaci\'{o}n cuadr\'{a}tica&
    \tiny \ding{55}&
    \tiny \ding{55} \\ \hline
    %--------------------------
    \tiny \cellcolor{gray!20}\cite{PAPACHRISTOS2017}[\citenum{PAPACHRISTOS2017}]&
    \tiny \cellcolor{gray!20}Exploración&
    \tiny \cellcolor{gray!20}Octomap&
    \tiny \cellcolor{gray!20}Next Best View Planner (NBVP)&
    \tiny \cellcolor{gray!20}Control directo de velocidad&
    \tiny \cellcolor{gray!20}\ding{55}&
    \tiny \cellcolor{gray!20}\ding{55} \\ \hline
    %--------------------------
    \tiny \cite{OLEYNIKOVA2018}[\citenum{OLEYNIKOVA2018}]&
    \tiny Punto Objetivo&
    \tiny Voxel Hashing TSDF&
    \tiny Next Best View Planner (NBVP)&
    \tiny Optimizaci\'{o}n cuadr\'{a}tica&
    \tiny \ding{51}&
    \tiny \ding{51} \\ \hline
    %--------------------------
    \tiny \cite{GAO2018}[\citenum{GAO2018}]&
    \tiny Punto Objetivo&
    \tiny Mapa de cuadr\'{i}cula&
    \tiny M\'{e}todo de marcha r\'{a}pida&
    \tiny Optimizaci\'{o}n cuadr\'{a}tica&
    \tiny \ding{55}&
    \tiny \ding{51} \\ \hline
    %--------------------------
    \tiny \cite{FLORENCE2018}[\citenum{FLORENCE2018}]&
    \tiny Punto Objetivo&
    \tiny Busqueda basada en visibilidad&
    \tiny 2D A*&
    \tiny Control predictivo por modelo (MPC)&
    \tiny \ding{51}&
    \tiny \ding{51} \\ \hline
    %--------------------------
    \tiny \cellcolor{gray!20}\cite{SELIN2019}[\citenum{SELIN2019}]&
    \tiny \cellcolor{gray!20}Exploración&
    \tiny \cellcolor{gray!20}Octomap&
    \tiny \cellcolor{gray!20}Next Best View Planner (NBVP)&
    \tiny \cellcolor{gray!20}Control directo de velocidad&
    \tiny \cellcolor{gray!20}\ding{55}&
    \tiny \cellcolor{gray!20}\ding{55} \\ \hline
    %--------------------------
    \tiny \cellcolor{gray!20}\cite{BUG2019}[\citenum{BUG2019}]&
    \tiny \cellcolor{gray!20}Exploración&
    \tiny \cellcolor{gray!20}NA&
    \tiny \cellcolor{gray!20}Swarm Gradient Bug Algorithm (SGBA)&
    \tiny \cellcolor{gray!20}Control directo de velocidad&
    \tiny \cellcolor{gray!20}\ding{55}&
    \tiny \cellcolor{gray!20}\ding{55} \\ \hline
    %--------------------------
    \tiny \cite{COLLINS2019}[\citenum{COLLINS2019}]&
    \tiny Punto Objetivo&
    \tiny KD Tree $+$ Mapa en Voxel&
    \tiny B\'{u}squeda en Grafo&
    \tiny Movimientos suaves&
    \tiny \ding{51}&
    \tiny \ding{51} \\ \hline
    %--------------------------
    \tiny \cite{CINVES2021}[\citenum{CINVES2021}]&
    \tiny Punto Objetivo&
    \tiny Octree&
    \tiny Rapidly Exploring Random Trees (RRT)&
    \tiny Basado en contornos&
    \tiny \ding{51}&
    \tiny \ding{51} \\ \hline
    %--------------------------
    \tiny \cellcolor{gray!20}\cite{CIESLEWSKI2021}[\citenum{CIESLEWSKI2021}]&
    \tiny \cellcolor{gray!20}Exploración&
    \tiny \cellcolor{gray!20}Octomap&
    \tiny \cellcolor{gray!20}Basado en fronteras&
    \tiny \cellcolor{gray!20}Control directo de velocidad&
    \tiny \cellcolor{gray!20}\ding{51}&
    \tiny \cellcolor{gray!20}\ding{55} \\ \hline
    %--------------------------
    \tiny \cellcolor{gray!20}\cite{RACER2022}[\citenum{RACER2022}]&
    \tiny \cellcolor{gray!20}Exploración&
    \tiny \cellcolor{gray!20}HGrid&
    \tiny \cellcolor{gray!20}Next Best View Planner (NBVP)&
    \tiny \cellcolor{gray!20}Control directo de velocidad&
    \tiny \cellcolor{gray!20}\ding{51}&
    \tiny \cellcolor{gray!20}\ding{51} \\ \hline
    %--------------------------
    %\scriptsize \cite{WESTHEIDER2023}[\citenum{WESTHEIDER2023}]&
    %\scriptsize Mapa de cuadrícula&
    %\scriptsize Deep Reinforcement Learning&
    %\scriptsize Control directo de velocidad \\ \hline
    %--------------------------
    \tiny \cellcolor{gray!20}\cite{BARTOLOMEI2023}[\citenum{BARTOLOMEI2023}]&
    \tiny \cellcolor{gray!20}Exploración&
    \tiny \cellcolor{gray!20}HGrid&
    \tiny \cellcolor{gray!20}Basado en fronteras&
    \tiny \cellcolor{gray!20}Control directo de velocidad&
    \tiny \cellcolor{gray!20}\ding{51}&
    \tiny \cellcolor{gray!20}\ding{51} \\ \hline
    %--------------------------
  \end{tabular}
  }
\end{frame}

\begin{frame}[allowframebreaks]{Referencias}
  \bibliographystyle{abbrvnat}
  \bibliography{demo}
\end{frame}

\end{document}
