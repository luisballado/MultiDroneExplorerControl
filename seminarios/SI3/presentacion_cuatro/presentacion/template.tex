%%%%%%%%%%%%%%%%%%%%%%%%%%%%%%%%%%%%%%%%%
% Beamer Presentation
% LaTeX Template
% Version 2.0 (March 8, 2022)
%
% This template originates from:
% https://www.LaTeXTemplates.com
%
% Author:
% Vel (vel@latextemplates.com)
%
% License:
% CC BY-NC-SA 4.0 (https://creativecommons.org/licenses/by-nc-sa/4.0/)
%
%%%%%%%%%%%%%%%%%%%%%%%%%%%%%%%%%%%%%%%%%

%----------------------------------------------------------------------------------------
%	PACKAGES AND OTHER DOCUMENT CONFIGURATIONS
%----------------------------------------------------------------------------------------
\documentclass[
  24pt, % Set the default font size, options include: 8pt, 9pt, 10pt, 11pt, 12pt, 14pt, 17pt, 20pt
  %t, % Uncomment to vertically align all slide content to the top of the slide, rather than the default centered
  aspectratio=169, % Uncomment to set the aspect ratio to a 16:9 ratio which matches the aspect ratio of 1080p and 4K screens and projectors
]{beamer}

\graphicspath{{Images/}{./}} % Specifies where to look for included images (trailing slash required)

\usepackage{booktabs} % Allows the use of \toprule, \midrule and \bottomrule for better rules in tables

%----------------------------------------------------------------------------------------
%	SELECT LAYOUT THEME
%----------------------------------------------------------------------------------------

% Beamer comes with a number of default layout themes which change the colors and layouts of slides. Below is a list of all themes available, uncomment each in turn to see what they look like.

%\usetheme{default}
%\usetheme{AnnArbor}
%\usetheme{Antibes}
%\usetheme{Bergen}
%\usetheme{Berkeley}
%\usetheme{Berlin}
\usetheme{Boadilla} %me gusta
%\usetheme{CambridgeUS}
%\usetheme{Copenhagen}
%\usetheme{Darmstadt}
%\usetheme{Dresden}
%\usetheme{Frankfurt}
%\usetheme{Goettingen} %dos dos
%\usetheme{Hannover} %dos dos
%\usetheme{Ilmenau}
%\usetheme{JuanLesPins}
%\usetheme{Luebeck}
%\usetheme{Madrid}
%\usetheme{Malmoe}
%\usetheme{Marburg}
%\usetheme{Montpellier}
%\usetheme{PaloAlto}
%\usetheme{Pittsburgh}
%\usetheme{Rochester} %muy flat
%\usetheme{Singapore}
%\usetheme{Szeged}
%\usetheme{Warsaw}

%----------------------------------------------------------------------------------------
%	SELECT COLOR THEME
%----------------------------------------------------------------------------------------

% Beamer comes with a number of color themes that can be applied to any layout theme to change its colors. Uncomment each of these in turn to see how they change the colors of your selected layout theme.

%\usecolortheme{albatross}
%\usecolortheme{beaver}
%\usecolortheme{beetle}
%\usecolortheme{crane}
%\usecolortheme{dolphin}
%\usecolortheme{dove}
%\usecolortheme{fly}
%\usecolortheme{lily} %default
%\usecolortheme{monarca}
%\usecolortheme{seagull}
%\usecolortheme{seahorse}
%\usecolortheme{spruce}
%\usecolortheme{whale}
%\usecolortheme{wolverine}

%----------------------------------------------------------------------------------------
%	SELECT FONT THEME & FONTS
%----------------------------------------------------------------------------------------

% Beamer comes with several font themes to easily change the fonts used in various parts of the presentation. Review the comments beside each one to decide if you would like to use it. Note that additional options can be specified for several of these font themes, consult the beamer documentation for more information.

\usefonttheme{default} % Typeset using the default sans serif font
%\usefonttheme{serif} % Typeset using the default serif font (make sure a sans font isn't being set as the default font if you use this option!)
%\usefonttheme{structurebold} % Typeset important structure text (titles, headlines, footlines, sidebar, etc) in bold
%\usefonttheme{structureitalicserif} % Typeset important structure text (titles, headlines, footlines, sidebar, etc) in italic serif
%\usefonttheme{structuresmallcapsserif} % Typeset important structure text (titles, headlines, footlines, sidebar, etc) in small caps serif

%------------------------------------------------

%\usepackage{mathptmx} % Use the Times font for serif text
\usepackage{palatino} % Use the Palatino font for serif text

\usepackage[ruled,vlined]{algorithm2e}
%\usepackage{helvet} % Use the Helvetica font for sans serif text
\usepackage[default]{opensans} % Use the Open Sans font for sans serif text
\usepackage[spanish]{babel}
\usepackage{dirtree}
\usepackage{xcolor}
%\usepackage[default]{FiraSans} % Use the Fira Sans font for sans serif text
%\usepackage[default]{lato} % Use the Lato font for sans serif text

\usepackage[scaled]{helvet}
\usepackage[round]{natbib}
%\newcommand{\newblock}{}

\usepackage{rotating}

\newcommand\FourQuad[4]{%
  \begin{minipage}[b][.33\textheight][t] 
    {.48\textwidth}#1\end{minipage}\hfill%
    \begin{minipage}[b][.33\textheight][t] 
      {.48\textwidth}#2\end{minipage}\\[0.5em]
      \begin{minipage}[b][.33\textheight][t] 
        {.48\textwidth}#3\end{minipage}\hfill
        \begin{minipage}[b][.33\textheight][t] 
          {.48\textwidth}#4\end{minipage}%
}

\usepackage{tikz}
%\usetikzlibrary{arrows,shapes,positioning,shadows,trees,quotes}


%\tikzset{
%  basic/.style  = {draw, text width=2cm, drop shadow, font=\sffamily, rectangle},
%  root/.style   = {basic, rounded corners=2pt, thin, align=center,
%                   fill=green!30},
%  level 2/.style = {basic, rounded corners=6pt, thin,align=center, fill=green!60,
%                   text width=8em},
%  level 3/.style = {basic, thin, align=left, fill=pink!60, text width=6.5em}
%}

\usetikzlibrary{calc}

\tikzstyle{part} = [rectangle, rounded corners, minimum width=3cm, minimum height=1cm,     align=center, draw=black]
\tikzstyle{chapter} = [rectangle, rounded corners, minimum width=3cm, minimum height=1cm,     align=center, draw=black, text width=3.5cm]
\tikzstyle{arrow} = [thick, ->]

\usepackage{array} % needed for \arraybackslash
\usepackage{graphicx}
\usepackage{adjustbox} % for \adjincludegraphics

\usepackage{subcaption}
\usepackage{bibentry}
%\bibliographystyle{apalike}
\usepackage{chngcntr}
\usepackage{lipsum}% http://ctan.org/pkg/lipsum
\usepackage{hanging}% http://ctan.org/pkg/hanging

\usepackage{xcolor,colortbl}
\usepackage{multirow}

\usepackage{animate}
\usepackage{multicol}
\usepackage{tabularx,booktabs}
\usepackage{forloop}
\usepackage{ragged2e}

\usepackage{bbding} %palomitas checkmark
\usepackage{pifont}
\usepackage{lipsum,tabularx}

\newcounter{loopcntr}

%----------------------------------------------------------------------------------------
%	SELECT INNER THEME
%----------------------------------------------------------------------------------------

% Inner themes change the styling of internal slide elements, for example: bullet points, blocks, bibliography entries, title pages, theorems, etc. Uncomment each theme in turn to see what changes it makes to your presentation.

\useinnertheme{circles}

%----------------------------------------------------------------------------------------
%	SELECT OUTER THEME
%----------------------------------------------------------------------------------------

% Outer themes change the overall layout of slides, such as: header and footer lines, sidebars and slide titles. Uncomment each theme in turn to see what changes it makes to your presentation.

\setbeamertemplate{footline}[frame number] % Uncomment this line to replace the footer line in all slides with a simple slide count
\setbeamertemplate{navigation symbols}{} % Uncomment this line to remove the navigation symbols from the bottom of all slides

%----------------------------------------------------------------------------------------
%	PRESENTATION INFORMATION
%----------------------------------------------------------------------------------------

\title[PRESENTACION 4]{%\centering\includegraphics[width=10cm]{swarm_drones}\\
  Estrategias para la exploración coordinada multi-VANT} % The short title in the optional parameter appears at the bottom of every slide, the full title in the main parameter is only on the title page

%\subtitle{Optional Subtitle} % Presentation subtitle, remove this command if a subtitle isn't required

\author[]{Luis Alberto Ballado Aradias\\[\baselineskip]
  \small{{Asesores:} \\
    \and\\Dr. José Gabriel Ramírez-Torres
    \and\\Dr. Eduardo Rodriguez-Tello }}

\institute[CINVESTAV]{
  CINVESTAV UNIDAD TAMAULIPAS \\
  %\smallskip \textit{luis.ballado@cinvestav.mx}
} % Your institution, the optional parameter can be used for the institution shorthand and will appear on the bottom of every slide after author names, while the required parameter is used on the title slide and can include your email address or additional information on separate lines


\date[\today]{Cd. Victoria, Tamaulipas - 24 Abril 2024} % Presentation date or conference/meeting name, the optional parameter can contain a shortened version to appear on the bottom of every slide, while the required parameter value is output to the title slide

%\titlegraphic{\hspace*{8.75cm}~%
%   \includegraphics[width=0.8cm]{cinvestavlogo}
%}

%----------------------------------------------------------------------------------------

\counterwithin*{footnote}{page}
\newcommand\footcite[1]{\footnote{\bibentry{#1}}\label{\thepage:#1}}
\newcommand\secondcite[1]{\textsuperscript{\ref{\thepage:#1}}}

\newcommand{\rpt}[2][1]{%
  \forloop{loopcntr}{0}{\value{loopcntr}<#1}{#2}%
}
\newcommand{\on}[1][1]{
  \forloop{loopcntr}{0}{\value{loopcntr}<#1}{&\cellcolor{gray}}
}
\newcommand{\onok}[1][1]{
  \forloop{loopcntr}{0}{\value{loopcntr}<#1}{&\cellcolor{green}}
}
\newcommand{\off}[1][1]{
  \forloop{loopcntr}{0}{\value{loopcntr}<#1}{&\cellcolor{white}}
}

\addtolength{\textheight}{90pt}

\newcommand{\I}{\mathbb{I}}
\newcommand{\K}{\mathbb{K}}
\newcommand{\N}{\mathbb{N}}
\newcommand{\Q}{\mathbb{Q}}
\newcommand{\R}{\mathbb{R}}
\newcommand{\Z}{\mathbb{Z}}

\newcommand{\specialcell}[2][c]{%
  \begin{tabular}[#1]{@{}c@{}}#2\end{tabular}}


\begin{document}

%----------------------------------------------------------------------------------------
%	TITLE SLIDE
%----------------------------------------------------------------------------------------

\begin{frame}
  \titlepage % Output the title slide, automatically created using the text entered in the PRESENTATION INFORMATION block above
\end{frame}

%----------------------------------------------------------------------------------------
%	TABLE OF CONTENTS SLIDE
%----------------------------------------------------------------------------------------

% The table of contents outputs the sections and subsections that appear in your presentation, specified with the standard \section and \subsection commands. You may either display all sections and subsections on one slide with \tableofcontents, or display each section at a time on subsequent slides with \tableofcontents[pausesections]. The latter is useful if you want to step through each section and mention what you will discuss.
\AtBeginSection[]
{
  \begin{frame}
    \frametitle{Contenido} % Slide title, remove this command for no title
    \tableofcontents[currentsection] % Output the table of contents (all sections on one slide)
    %\tableofcontents[pausesections] % Output the table of contents (break sections up across separate slides)
  \end{frame}
}

\pretocmd{\tableofcontents}{\thispagestyle{empty}}{}{}

%----------------------------------------------------------------------------------------
%	PRESENTATION BODY SLIDES
%----------------------------------------------------------------------------------------

\section{Antecedentes y motivación del proyecto}
\begin{frame}{Antecedentes 1/3}
  %HACER UNA NUEVA SLIDE PARA EXPLICAR LOS ANTECEDENTES PARA LLEGAR AL PROBLEMA DE EXPLORACION MULTI AGENTE
  \bigskip % Vertical whitespace
  \centering
  \includegraphics[width=0.45\textwidth,height=0.35\textheight]{DJI_B4}
  \hfil
  \includegraphics[width=0.45\textwidth,height=0.35\textheight]{drones-07-00236-g001.png}\footnotemark
  \vspace{2pt}\\
  
  \begin{itemize}
  \item \textbf{UAV} {\tiny(\textit{Vehículo Aéreo No Tripulado - VANT})}  $\implies$ \textbf{UAS} {\tiny(\textit{Sistemas Aéreos No Tripulado - SANT})}
  \item \textbf{Aplicaciones} en lugares inaccesibles o peligrosos.
  \item \textbf{Múltiples VANT} pueden reducir el tiempo de exploración y aumentar la confianza del sistema.
  \item \textbf{Limitaciones} en carga, procesamiento y batería influyen en el tiempo de vuelo.
  \end{itemize}
  
  \footnotetext{UAV in the advent of the twenties: Where we stand and what is next [\cite{Nex2022}]}
  %\includegraphics[width=0.45\textwidth,height=0.35\textheight]{DJI_B5}$^\dag$ 
  %\hfil
  %\includegraphics[width=0.45\textwidth,height=0.35\textheight]{DJI_B4}$^\dag$\\
  %\rule{0in}{1.2em}$^\dag$ \small Inspecciones con VANT basadas en los mejores casos de uso\\
  %\tiny \url{https://enterprise-insights.dji.com/blog/complete-guide-to-drone-inspections}
\end{frame}

\begin{frame}{Antecedentes 2/3 - ROBOT AUTÓNOMO}
  \centering
  \includegraphics[width=\linewidth]{ANTECEDENTES}\\
  %HACER UNA NUEVA SLIDE PARA EXPLICAR LOS ANTECEDENTES PARA LLEGAR AL PROBLEMA DE EXPLORACION MULTI AGENTE
  %\begin{figure}[ht!]
   %     \centering
    %    \begin{minipage}{0.48\textwidth}
     %       \centering
      %      \includegraphics[width=\linewidth]{drone_taxo} % BUSQUEDA Y RESCATE
            %\caption{Primera Imagen}
       % \end{minipage}\hfill
        %\begin{minipage}{0.48\textwidth}
         %   \centering
          %  \includegraphics[width=\linewidth]{octomapas} % INSPECCION Y SEGURIDAD
            %\caption{Segunda Imagen}
       % \end{minipage}
       % \vspace{-0.2cm} % Espacio vertical entre imágenes
       % \begin{minipage}{0.48\textwidth}
       %     \centering
       %     \includegraphics[width=0.9\linewidth]{ORB_SLAM} % DRONES EN CIUDADES CON GPS DENY
            %\caption{Tercera Imagen}
       % \end{minipage}\hfill
       % \begin{minipage}{0.48\textwidth}
       %     \centering
       %     \includegraphics[width=0.8\linewidth]{path2} % 
            %\caption{Cuarta Imagen}
       % \end{minipage}
    %\end{figure}
\end{frame}

\begin{frame}{Antecedentes 3/3 - EXPLORACIÓN}
  
  %Exploración es una tarea fundamental en robots autónomos.\\
  %El objetivo es crear un mapa de un ambiente desconocido.\\
  %\bigskip % Vertical whitespace
  \centering
  \includegraphics[width=9cm]{exploracion}\\
  
  \begin{itemize}
  \item \textbf{Sensar}
  \item \textbf{Creación Mapa} 
  \item \textbf{Localización en Mapa}
  \item \textbf{Exploración} (Aumentar la base de conocimiento del mapa)
  \item \textbf{Planificación trayectoria} (Trayectorias hacia nuevas fronteras) 
  \item \textbf{Control} (Toma de decisiones y ejecución de trayectorias)
  \end{itemize}

  \alert{- El ciclo se repite hasta completar la exploración -}
  
\end{frame}

\begin{frame}{Motivación del proyecto}
  %https://ccc.inaoep.mx/~emorales/Papers/2009/eduardo.pdf
  %https://www.mdpi.com/2504-446X/7/1/62
  %Hablar de los usos e importancia del trabajo .. de la exploracion .. de la importancia de investigacion en robotica para el pais.
  %El potencial del uso de UAVs en tareas de búsqueda y rescate, inspección, mapeo, vigilancia, entre otras, es de gran interés a explorar, debido a las habilidades de vuelo que presentan en favor de la realización de estas tareas, y en especial situaciones que podrían poner en riesgo a personas.\\
  %Enviar personal de rescate dentro de un edificio parcialmente colapsado por un terremoto en busca de sobrevivientes, es poner a más personas en un gran riesgo, pues no se sabe qué es lo que les espera en el interior del edificio; esto limita la capacidad de tomar buenas decisiones acerca de si es seguro seguir cierto camino.
  \begin{figure}[ht!]
        \centering
        \begin{minipage}{0.48\textwidth}
            \centering
            \includegraphics[width=\linewidth,height=3.5cm]{turquia1.jpg} % BUSQUEDA Y RESCATE
            %\caption{Primera Imagen}
        \end{minipage}\hfill
        \begin{minipage}{0.48\textwidth}
            \centering
            \includegraphics[width=\linewidth,height=3.5cm]{drone_agriculture.jpg} % INSPECCION Y SEGURIDAD
            %\caption{Segunda Imagen}
        \end{minipage}
        \vspace{-0.2cm} % Espacio vertical entre imágenes
        \begin{minipage}{0.48\textwidth}
            \centering
            \includegraphics[width=\linewidth,height=3.5cm]{pipe_drone.jpg} % DRONES EN CIUDADES CON GPS DENY
            %\caption{Tercera Imagen}
        \end{minipage}\hfill
        \begin{minipage}{0.48\textwidth}
            \centering
            \includegraphics[width=\linewidth,height=3.5cm]{drone_city.jpg} % 
            %\caption{Cuarta Imagen}
        \end{minipage}
    \end{figure}
\end{frame}

\section{Planteamiento del problema}
\begin{frame}{Planteamiento del problema}

  
  %Crear una Coordinar un conjunto de vehiculos aereos no tripulados
  %Explorar un volumen desconocido tal que $V \subset R^{3}$ con un conjunto de robots aéreos autónomos $R ={R_1,R_2,...,R_n}$, siendo n la cardinalidad del conjunto de robots aéreos. Cada robot cuenta con capacidades de generar una representación del medio ambiente, localizarse y generar trayectorias que minimicen la incertidumbre de la localización y el mapeo, evaluada a través de una métrica sobre la creencia probabilística de la pose del robot y la posición de los puntos de referencia. El problema de exploración puede formularse globalmente como el de partir de una configuración inicial libre de colisiones y derivar puntos de vista que cubran el volumen previamente desconocido, determinando qué partes del espacio inicialmente inexplorado están libres o ocupadas. Para este proceso, el volumen se discretiza en un mapa de ocupación M consistente en voxeles cúbicos $m \exists M$ con una longitud de arista r. La operación está sujeta a las restricciones dinámicas del vehículo y a las limitaciones del sistema de sensores. Dado que las capacidades perceptivas de la mayoría de los sensores terminan en las superficies, los espacios huecos o bolsillos estrechos a veces pueden permanecer inexplorados, lo que lleva a un volumen residual.
  
\end{frame}

\section{Hipótesis y preguntas de investigación}
\begin{frame}%{Hipótesis y preguntas de investigación}
  %\textbf{Exploración}\\
  %\bigskip % Vertical whitespace

  \begin{block}{Hipótesis}
    Una estrategia que coordine y asigne tareas de exploración para múltiples VANTS de manera descentralizada, en combinación con una arquitectura de software diseñada para resolver problemas de localización, gestión de mapas y planificación de rutas, mejorará la eficiencia y cobertura de la exploración en interiores de un entorno desconocido.
  \end{block}

  \begin{block}{Preguntas de investigación}
    \begin{itemize}
    \item ¿Qué características de la dinámica del VANT se deben considerar en el simulador para lograr trayectorias suavizadas y continúas?
    \item ¿El uso de un planificador de trayectorias que explote las regiones libres de obstáculos contribuirá a acelerar los desplazamientos de los VANTs y, por ende, a reducir los tiempos de exploración?
    \item ¿Qué mecanismos de coordinación existen dentro de la literatura que podrían ayudar en resolver el problema de exploración multi-VANT?
    \end{itemize}
  \end{block}
\end{frame}

\section{Objetivos generales y particulares}
\begin{frame}{Objetivos generales y particulares}
  \begin{enumerate}
  \item Objetivo General:\\
    \bigskip
    %Desarrollar una estrategia de exploración descentralizada que permita resolver los problemas de coordinación para múltiples VANTS en ambientes desconocidos.
    %Desarrollar una estrategia de exploración descentralizada que integre técnicas de percepción y algoritmos de decisión para mejorar la autonomía de múltiples VANTs en la exploración de interiores.
    Desarrollar una estrategia de exploración descentralizada que permita resolver los problemas de coordinación para múltiples VANTS en ambientes desconocidos.
    \pause
    \bigskip
  \item Objetivos Particulares:\\
    \vspace{4mm}
    \begin{itemize}
    \item Desarrollar una arquitectura de software que resuelva los problemas de autonomía para un VANT (localización, manejo de mapas y planificación de trayectorias).
    \item Implementar un mecanismo de coordinación descentralizado que asigne tareas de exploración.
    \item Realizar pruebas y simulaciones de la solución propuesta en diversos entornos, analizando la relación tiempo de exploración y cobertura del área de interés.
    \end{itemize}
  \end{enumerate}
\end{frame}

\section{Estado del arte}
\begin{frame}{Estado del arte}
  \centering
  \scalebox{0.85}{
  \begin{tabular}{ | p{3cm} | p{1.6cm} | p{2.5cm} | p{3cm} | p{3.1cm} | p{0.8cm} | p{0.9cm} | }
    \hline    
    \tiny REFERENCIA&
    \tiny APLICACIÓN&
    \tiny GENERACION MAPA&
    \tiny PLANIFICACION DE RUTA&
    \tiny GENERACION TRAYECTORIA&
    \tiny SENSOR RGB-D&
    \tiny DINAMICA VANT\\
    \hline
    %--------------------------
    \tiny \cellcolor{gray!20}\cite{CIESLEWSKI2017}[\citenum{CIESLEWSKI2017}]&
    \tiny \cellcolor{gray!20}Exploración&
    \tiny \cellcolor{gray!20}Octomap&
    \tiny \cellcolor{gray!20}Basado en fronteras&
    \tiny \cellcolor{gray!20}Control directo de velocidad&
    \tiny \cellcolor{gray!20}\ding{51} &
    \tiny \cellcolor{gray!20}\ding{55} \\ \hline
    %--------------------------
    \tiny \cite{USENKO2017}[\citenum{USENKO2017}]&
    \tiny Punto Objetivo&
    \tiny Cuadr\'{i}cula egoc\'{e}ntrica&
    \tiny Offline RRT*&
    \tiny Curvas de Bezier&
    \tiny \ding{55} &
    \tiny \ding{51} \\ \hline
    %--------------------------
    \tiny \cite{MOHTA2017}[\citenum{MOHTA2017}]&
    \tiny Punto Objetivo&
    \tiny mapa 3D-Local y 2D-Global&
    \tiny A*&
    \tiny Programaci\'{o}n cuadr\'{a}tica&
    \tiny \ding{55} &
    \tiny \ding{51} \\ \hline
    %--------------------------
    \tiny \cite{LIN2017}[\citenum{LIN2017}]&
    \tiny Punto Objetivo&
    \tiny 3D voxel array TSDF&
    \tiny A*&
    \tiny Optimizaci\'{o}n cuadr\'{a}tica&
    \tiny \ding{55}&
    \tiny \ding{55} \\ \hline
    %--------------------------
    \tiny \cellcolor{gray!20}\cite{PAPACHRISTOS2017}[\citenum{PAPACHRISTOS2017}]&
    \tiny \cellcolor{gray!20}Exploración&
    \tiny \cellcolor{gray!20}Octomap&
    \tiny \cellcolor{gray!20}Next Best View Planner (NBVP)&
    \tiny \cellcolor{gray!20}Control directo de velocidad&
    \tiny \cellcolor{gray!20}\ding{55}&
    \tiny \cellcolor{gray!20}\ding{55} \\ \hline
    %--------------------------
    \tiny \cite{OLEYNIKOVA2018}[\citenum{OLEYNIKOVA2018}]&
    \tiny Punto Objetivo&
    \tiny Voxel Hashing TSDF&
    \tiny Next Best View Planner (NBVP)&
    \tiny Optimizaci\'{o}n cuadr\'{a}tica&
    \tiny \ding{51}&
    \tiny \ding{51} \\ \hline
    %--------------------------
    \tiny \cite{GAO2018}[\citenum{GAO2018}]&
    \tiny Punto Objetivo&
    \tiny Mapa de cuadr\'{i}cula&
    \tiny M\'{e}todo de marcha r\'{a}pida&
    \tiny Optimizaci\'{o}n cuadr\'{a}tica&
    \tiny \ding{55}&
    \tiny \ding{51} \\ \hline
    %--------------------------
    \tiny \cite{FLORENCE2018}[\citenum{FLORENCE2018}]&
    \tiny Punto Objetivo&
    \tiny Busqueda basada en visibilidad&
    \tiny 2D A*&
    \tiny Control predictivo por modelo (MPC)&
    \tiny \ding{51}&
    \tiny \ding{51} \\ \hline
    %--------------------------
    \tiny \cellcolor{gray!20}\cite{SELIN2019}[\citenum{SELIN2019}]&
    \tiny \cellcolor{gray!20}Exploración&
    \tiny \cellcolor{gray!20}Octomap&
    \tiny \cellcolor{gray!20}Next Best View Planner (NBVP)&
    \tiny \cellcolor{gray!20}Control directo de velocidad&
    \tiny \cellcolor{gray!20}\ding{55}&
    \tiny \cellcolor{gray!20}\ding{55} \\ \hline
    %--------------------------
    \tiny \cellcolor{gray!20}\cite{BUG2019}[\citenum{BUG2019}]&
    \tiny \cellcolor{gray!20}Exploración&
    \tiny \cellcolor{gray!20}NA&
    \tiny \cellcolor{gray!20}Swarm Gradient Bug Algorithm (SGBA)&
    \tiny \cellcolor{gray!20}Control directo de velocidad&
    \tiny \cellcolor{gray!20}\ding{55}&
    \tiny \cellcolor{gray!20}\ding{55} \\ \hline
    %--------------------------
    \tiny \cite{COLLINS2019}[\citenum{COLLINS2019}]&
    \tiny Punto Objetivo&
    \tiny KD Tree $+$ Mapa en Voxel&
    \tiny B\'{u}squeda en Grafo&
    \tiny Movimientos suaves&
    \tiny \ding{51}&
    \tiny \ding{51} \\ \hline
    %--------------------------
    \tiny \cite{CINVES2021}[\citenum{CINVES2021}]&
    \tiny Punto Objetivo&
    \tiny Octree&
    \tiny Rapidly Exploring Random Trees (RRT)&
    \tiny Basado en contornos&
    \tiny \ding{51}&
    \tiny \ding{51} \\ \hline
    %--------------------------
    \tiny \cellcolor{gray!20}\cite{CIESLEWSKI2021}[\citenum{CIESLEWSKI2021}]&
    \tiny \cellcolor{gray!20}Exploración&
    \tiny \cellcolor{gray!20}Octomap&
    \tiny \cellcolor{gray!20}Basado en fronteras&
    \tiny \cellcolor{gray!20}Control directo de velocidad&
    \tiny \cellcolor{gray!20}\ding{51}&
    \tiny \cellcolor{gray!20}\ding{55} \\ \hline
    %--------------------------
    \tiny \cellcolor{gray!20}\cite{RACER2022}[\citenum{RACER2022}]&
    \tiny \cellcolor{gray!20}Exploración&
    \tiny \cellcolor{gray!20}HGrid&
    \tiny \cellcolor{gray!20}Next Best View Planner (NBVP)&
    \tiny \cellcolor{gray!20}Control directo de velocidad&
    \tiny \cellcolor{gray!20}\ding{51}&
    \tiny \cellcolor{gray!20}\ding{51} \\ \hline
    %--------------------------
    %\scriptsize \cite{WESTHEIDER2023}[\citenum{WESTHEIDER2023}]&
    %\scriptsize Mapa de cuadrícula&
    %\scriptsize Deep Reinforcement Learning&
    %\scriptsize Control directo de velocidad \\ \hline
    %--------------------------
    \tiny \cellcolor{gray!20}\cite{BARTOLOMEI2023}[\citenum{BARTOLOMEI2023}]&
    \tiny \cellcolor{gray!20}Exploración&
    \tiny \cellcolor{gray!20}HGrid&
    \tiny \cellcolor{gray!20}Basado en fronteras&
    \tiny \cellcolor{gray!20}Control directo de velocidad&
    \tiny \cellcolor{gray!20}\ding{51}&
    \tiny \cellcolor{gray!20}\ding{51} \\ \hline
    %--------------------------
  \end{tabular}
  }
\end{frame}

\begin{frame}{Dirección estado del arte}
  \centering
  \includegraphics[width=10cm, height=6cm]{soa}
\end{frame}

\section{Enfoque propuesto}
\begin{frame}{Enfoque propuesto}
  El objetivo principal es efectuar una exploracion con multiples vants de forma coordinada y descentralizada.\\

  1.- Conocer los fundamentos matematicos que nos aproximen realizar la tarea de exploración autonoma.
  - Donde estoy, a donde voy y cómo llego ahí\\

  2.- Definir el uso de un simulador que nos permita validar la solución propuesta\\

  3.- Conocer los algoritmos que nos acerquen a la solucion\\

  4.- Coordinar\\

  - Crear la arquitectura para dotar de autonomia a los drones\\
  - Crear una exploración autonoma para un VANT\\
  - Evaluación de fronteras \\
  
\end{frame}

\section{Resultados}
\begin{frame}{Resultados}
  \begin{itemize}
  \item El uso de ROS que nos resuelve el problema de computo distribuido, y comprender su arquitectura nos ayuda en formular y validar nuestra arquitectura que es escalable
  \item La configuración de un archivo .launch en ROS nos permite crear nodos con la cardinalidad de los VANT que deseamos en nuestra simulación
  \item Al completar la exploración de un VANT de forma autonoma, l
  \end{itemize}
\end{frame}

\begin{frame}{Discusión}
  EL USO DEL SIMULADOR Y COMPRENSION DEL SISTEMA ROS EN EL LENGUAJE C++ Y PYTHON
  Conocer las herramientas que nos permitirá la validación de nuestro enfoque
\end{frame}

\section{Conclusiones}
\begin{frame}{Conclusiones}
  QUE PONEMOS
\end{frame}

\section{Cronograma de actividades}
\begin{frame}{Cronograma de actividades}
  \tiny 

\hspace{0.0cm}\begin{minipage}{7cm}
  \noindent\begin{tabular}{|p{0.8\textwidth}*{12}{|p{0.040\textwidth}}|}
  % The top line
  %\diagbox[width=12.5em]{Actividades}{Cuatrimestres}
  \hline
  & \multicolumn{4}{c|}{\color{teal!80}\textbf{Cuatrimestre 4}} 
  & \multicolumn{4}{c|}{\color{teal!80}\textbf{Cuatrimestre 6}} 
  & \multicolumn{4}{c|}{\textbf{Cuatrimestre 6}}\\
  \hline
  % The second line, with its five years of four quarters
  %\textcolor{black}{\textbf{Etapas}}
  \rpt[3]{& 1 & 2 & 3 & 4} \\
  \hline
  \rowcolor{teal!40}{\textbf{Etapa 1}}\\
  % using the on macro to fill in twenty cells as `on'
  %\specialcell{Actividad 1\\espacio}        \on[0] \off[12] \\
%Actividad 1    \on[0] \off[12]\\
  \hline
  \rowcolor{teal!15}{\textbf{E1.A1.} Revisi\'{o}n literatura relevante}
%en exploraci\'{o}n multi-VANT, estrategias de exploraci\'{o}n, algoritmos de coordinaci\'{o}n y evaci\'{o}n de obst\'{a}culos
  \onok[8] 
  \on[4]\\
\hline
%\textbf{E1.A2.} Evaluaci\'{o}n de aptitudes\footnote{Evaluaci\'{o}n a partir de los nuevos trabajos}     \on[12] \\
%\hline
\rowcolor{teal!15}{\textbf{E1.A2.} Selecci\'{o}n de algoritmos} \onok[2] \off[10] \\
\hline
\rowcolor{teal!15}{\textbf{E1.A3.} Dise\~{n}o de la arquitectura de software}
\off[1] \onok[3] \on[0] \off[8] \\
\hline
\rowcolor{teal!15}{\textbf{E1.A4.} Documentaci\'{o}n Etapa 1}
\onok[4] \on[0]  \off[8] \\
\hline
\rowcolor{teal!15}{\textbf{E1.A5.} Revisi\'{o}n de tesis Etapa 1}
\off[3] \onok[1]  \off[8] \\
\hline
% using the on macro followed by the off macro
\rowcolor{teal!40}{\textbf{Etapa 2}}\\
\hline
\rowcolor{teal!15}{\textbf{E2.A1.} Selecci\'{o}n Simulador}
\onok[1]  \off[11] \\
\hline
\rowcolor{teal!15}{\textbf{E2.A2.} Visualizaci\'{o}n de datos}
\off[1] \onok[2]  \off[9] \\
\hline
\rowcolor{teal!15}{\textbf{E2.A3.} Control de desplazamientos}
\off[2] \onok[1] \on[1]  \off[8] \\
\hline
\textbf{E2.A4.} Desarrollo de algoritmo de exploraci\'{o}n
\off[3] \on[2] \off[7] \\
\hline
\textbf{E2.A5.} Implementaci\'{o}n y simulaci\'{o}n
\off[3] \on[2] \off[7] \\
\hline
%\textbf{E2.A6.} Simulaci\'{o}n un solo VANT
%\off[4] \on[1] \off[7] \\
%\hline
\textbf{E2.A6.} Desarrollo de coordinaci\'{o}n
\off[4] \on[3] \off[5] \\
\hline
\textbf{E2.A7.} Implementaci\'{o}n y simulaci\'{o}n
\off[6] \on[2] \off[4] \\
\hline
%\textbf{E2.A9.} Simulaci\'{o}n multi-VANT
%\off[6] \on[1] \off[5] \\
%\hline
\textbf{E2.A8.} Documentaci\'{o}n Etapa 2
\off[4] \on[4] \off[4] \\
\hline
\textbf{E2.A9.} Revisi\'{o}n de tesis Etapa 2
\off[7] \on[1] \off[4] \\
\hline
\rowcolor{black!5}{\textbf{Etapa 3}}\\
\hline
\textbf{E3.A1.} Experimentaci\'{o}n de soluci\'{o}n
\off[7]  \on[3] \off[2] \\
\hline
\textbf{E3.A2.} Recopilaci\'{o}n resultados
\off[9]  \on[1] \off[2] \\
\hline
\textbf{E3.A3.} Documentaci\'{o}n Etapa 3
\off[8] \on[3] \off[1] \\
\hline
\textbf{E3.A4.} Revisi\'{o}n de tesis
\off[10] \on[2] \\
\hline
\textbf{E3.A5.} Divulgaci\'{o}n
\off[5]  \on[7] \\
\hline
\textbf{E3.A6.} Proceso de titulaci\'{o}n
\off[11] \on\\
\hline
\end{tabular}
\end{minipage}

\end{frame}

\begin{frame}{Cronograma actualizado}
  \tiny 

\hspace{0.0cm}\begin{minipage}{7cm}
  \noindent\begin{tabular}{|p{0.8\textwidth}*{12}{|p{0.040\textwidth}}|}
  % The top line
  %\diagbox[width=12.5em]{Actividades}{Cuatrimestres}
  \hline
  & \multicolumn{4}{c|}{\color{teal!80}\textbf{Cuatrimestre 4}} 
  & \multicolumn{4}{c|}{\textbf{Cuatrimestre 5}}
  & \multicolumn{4}{c|}{\textbf{Cuatrimestre 6}}\\
  \hline
  % The second line, with its five years of four quarters
  %\textcolor{black}{\textbf{Etapas}}
  \rpt[3]{& 1 & 2 & 3 & 4} \\
  \hline
  \rowcolor{teal!40}{\textbf{Etapa 1}}\\
  % using the on macro to fill in twenty cells as `on'
  %\specialcell{Actividad 1\\espacio}        \on[0] \off[12] \\
%Actividad 1    \on[0] \off[12]\\
  \hline
  \rowcolor{teal!15}{\textbf{E1.A1.} Revisi\'{o}n literatura relevante}
%en exploraci\'{o}n multi-VANT, estrategias de exploraci\'{o}n, algoritmos de coordinaci\'{o}n y evaci\'{o}n de obst\'{a}culos
  \onok[3] 
  \on[9]\\
\hline
%\textbf{E1.A2.} Evaluaci\'{o}n de aptitudes\footnote{Evaluaci\'{o}n a partir de los nuevos trabajos}     \on[12] \\
%\hline
\rowcolor{teal!15}{\textbf{E1.A2.} Selecci\'{o}n de algoritmos} \onok[2] \off[10] \\
\hline
\rowcolor{teal!15}{\textbf{E1.A3.} Dise\~{n}o de la arquitectura de software}
\off[1] \onok[2] \on[1] \off[8] \\
\hline
\rowcolor{teal!15}{\textbf{E1.A4.} Documentaci\'{o}n Etapa 1}
\onok[3] \on[1]  \off[8] \\
\hline
\textbf{E1.A5.} Revisi\'{o}n de tesis Etapa 1 \off[3] \on[1]  \off[8] \\
\hline
% using the on macro followed by the off macro
\rowcolor{teal!40}{\textbf{Etapa 2}}\\
\hline
\rowcolor{teal!15}{\textbf{E2.A1.} Selecci\'{o}n Simulador}
\onok[1]  \off[11] \\
\hline
\rowcolor{teal!15}{\textbf{E2.A2.} Visualizaci\'{o}n de datos}
\off[1] \onok[2]  \off[9] \\
\hline
\rowcolor{teal!15}{\textbf{E2.A3.} Control de desplazamientos}
\off[2] \onok[1] \on[1]  \off[8] \\
\hline
\textbf{E2.A4.} Desarrollo de algoritmo de exploraci\'{o}n
\off[3] \on[2] \off[7] \\
\hline
\textbf{E2.A5.} Implementaci\'{o}n y simulaci\'{o}n
\off[3] \on[2] \off[7] \\
\hline
%\textbf{E2.A6.} Simulaci\'{o}n un solo VANT
%\off[4] \on[1] \off[7] \\
%\hline
\textbf{E2.A6.} Desarrollo de coordinaci\'{o}n
\off[4] \on[3] \off[5] \\
\hline
\textbf{E2.A7.} Implementaci\'{o}n y simulaci\'{o}n
\off[6] \on[2] \off[4] \\
\hline
%\textbf{E2.A9.} Simulaci\'{o}n multi-VANT
%\off[6] \on[1] \off[5] \\
%\hline
\textbf{E2.A8.} Documentaci\'{o}n Etapa 2
\off[4] \on[4] \off[4] \\
\hline
\textbf{E2.A9.} Revisi\'{o}n de tesis Etapa 2
\off[7] \on[1] \off[4] \\
\hline
\rowcolor{black!5}{\textbf{Etapa 3}}\\
\hline
\textbf{E3.A1.} Experimentaci\'{o}n de soluci\'{o}n
\off[7]  \on[3] \off[2] \\
\hline
\textbf{E3.A2.} Recopilaci\'{o}n resultados
\off[9]  \on[1] \off[2] \\
\hline
\textbf{E3.A3.} Documentaci\'{o}n Etapa 3
\off[8] \on[3] \off[1] \\
\hline
\textbf{E3.A4.} Revisi\'{o}n de tesis
\off[10] \on[2] \\
\hline
\textbf{E3.A5.} Divulgaci\'{o}n
\off[5]  \on[7] \\
\hline
\textbf{E3.A6.} Proceso de titulaci\'{o}n
\off[11] \on\\
\hline
\end{tabular}
\end{minipage}

\end{frame}

\begin{frame}[allowframebreaks,noframenumbering]{Bibliografía}
  \tiny
  \bibliographystyle{abbrvnat}
  \bibliography{test}
\end{frame}

\end{document} 
